In the previous chapter we have briefly touched upon the notion of Lie algebras.
A strictly related notion, we will see in which sense, is the notion of Lie group.
There are mathematical objects that are pervasive in mathematics, even outside the realm of differential geometry, and in physics, where they play an important role in classical mechanics\footnote{You may have heard of the celebrated Noether's theorem, which states that every smooth symmetry has a corresponding conservation law}, and in high-energy physics\footnote{Does gauge theory ring any bell?}.

The theory of Lie groups and Lie algebras is vast, and in these lectures we will just briefly scratch the surface.

\begin{definition}
  A \emph{Lie group $G$} is a smooth manifold (without boundary) that is also an algebraic group, with the property that the multiplication map $G\times G \to G$, $(g,h)\mapsto gh$, and the inversion map $G\to G$, $g\mapsto g^{-1}$ are smooth.
\end{definition}

\begin{example}
  \begin{enumerate}
    \item $\R^n$ is a Lie group under addition.
    \item $\R^n\setminus\{0\}$ is a Lie group under multiplication.
    \item A manifold can be equipped with different Lie group structures. For example, the following map
    \begin{equation}
      m(x,y) = (x^1+y^1,\, x^2+y^2,\, x^3+y^+x^1y^2)
    \end{equation}
    induces\footnote{To see that this defines a group structure, identify $\R^3$ with upper triangular $3\times3$ matrices via
    \begin{equation}
      x = (x^1, x^2, x^3) \mapsto \begin{pmatrix}
        1 & x^1 & x^3\\
        0 & 1 & x^2 \\
        0 & - & 1
      \end{pmatrix}
    \end{equation}
    and observe that $m$ becomes the standard matrix multiplication.} an alternative structure of Lie group on $\R^n$ called \emph{Heisenberg group}.
    \item The set $GL(n)$ of invertible $n\times n$ matrices is a Lie group under matrix multiplication. Indeed, it is a manifold of dimension $n^2$, the product is smooth since each matrix entry is given a polynomial and the inversion is smooth thanks to Cramer's rule.
    \item The $n$-torus $\bT^n = \R^n/\Z^n$ is an abelian Lie group with the group structure induced by addition on $\R^n$.
    \item Given Lie groups $(G_1, \ldots, G_k)$, their direct product is the product manifold $G_1\times \cdots\times G_k$ with the group structure given by
    \begin{equation}
      (g_1, \ldots, g_k)(h_1,\ldots,h_k) = (g_1h_1, \ldots g_kh_k)
    \end{equation}
    is a Lie group (why?).
    \item Not all smooth manifolds can be equipped with a Lie group structure: for example, $\bS^n$ admits a Lie group structure only for $n=0,1,3$.
  \end{enumerate}
\end{example}

\begin{definition}
  A \emph{Lie group homomorphism} $F:G\to H$ is a smooth map which is also a group homomorphism. It is called \emph{Lie group isomorphism} if it is also a diffeomorphism, which implies that it has an inverse that is also a Lie group homomorphism. In this case we call $G$ and $H$ isomorphic Lie groups.
\end{definition}

\begin{example}
  It turns out that you know plenty of examples of Lie group homomorpisms.
  \begin{enumerate}
    \item The map $\exp:\R\to\R\setminus\{0\}$ is a Lie group homomorphism. The image of $\exp$ is the open subgroup $\R+$ and $\exp:\R\to\R_+$ is a Lie group isomorphim with inverse $\log:\R_+\to\R$.
    \item The map $\epsilon:\R\to\bS^1$ defined by $\epsilon(t)=e^{2\pi i t}$ is a Lie group homomorphism whose kernel is $\Z$. Similarly, the map $\epsilon^n:\R^n\to\bT^n$ defined by $\epsilon^n(x^1, \ldots, x^n)=(e^{2\pi i x^1}, \ldots, e^{2\pi i x^n})$ is a Lie group homomorphism whose kernel is $\Z^n$.
    \item The determinant function $\det:GL(n)\to\R\setminus\{0\}$ is smooth since $\det$ is a polynomial in the entries of the matrix and it is a Lie group homomorphism since $\det (AB) = \det(A) \det(B)$.
  \end{enumerate}
\end{example}

\begin{definition}
  If $G$ is a Lie group, for any element $g\in G$, we denote by $L_g:G\to G$ the \emph{left translation} and by $R_g:G\to G$ the \emph{right translation}, respectively defined
  \begin{equation}
    L_g(h) = gh
    \quad\mbox{and}\quad
    R_g(h) = hg.
  \end{equation}
\end{definition}

\marginnote{The fact that translations are diffeomorphisms of the groups onto itself is crucial, it implies that the group looks the same around any point. Indeed, they are \emph{homogeneous spaces}. To study the local structure of a Lie group, as we will see soon, it is enough to examine a neighbourhood of the identity element.}
These are both diffeomorphisms, since they can be described by a composition of smooth maps. For instance,
\begin{equation}
  \underset{h}{G} \underset{\mapsto}{\to} \underset{(g,h)}{G\times G} \underset{\mapsto}{\to} \underset{gh}{G}.
\end{equation}
Moreover, $L_{g^{-1}}$ is the inverse of $L_g$.
Similarly for $R_g$.

\begin{remark}
  For convenience, we will only consider left translations.
  There is nothing wrong with right translations and, in fact, you can reformulate all the results that follow in terms of them.
\end{remark}

The next theorem is important for understanding many of the properties of Lie group homomorphisms.

\begin{theorem}
  Every Lie group homomorphism has constant rank.
\end{theorem}
\begin{proof}
  Let $F:G\to H$ be a Lie group homomorphism and let $e$ denote the identity element of $G$.
  
  Fix $g\in G$.
  We will show that $F$ has the same rank at $g$ as its rank at $e$.
  Since $F$ is a homomorphism, for all $h\in G$ we have
  \begin{equation}
    F(L_g(h)) = F(gh) = F(g)F(h) = L_{F(g)}(F(h)),
  \end{equation}
  that is,
  \begin{equation}
    F\circ L_g = L_{F(g)}\circ F.
  \end{equation}
  Differentiating both sides at $e$ and using the chain rule, this reads
  \begin{equation}
    dF_g\circ d(L_g)_e = d(L_{F(g)})_{F(e)}\circ dF_e.
  \end{equation}
  Since the left translation is a diffeomorphism, both $d(L_g)_e$ and $d(L_{F(g)})_{F(e)}$ are isomorphisms, and as such they preserve the rank.
  From this, it follows that $dF_g$ and $dF_e$ have the same rank.
\end{proof}

The global rank theorem then immediately implies the following corollary.
\begin{corollary}
  A Lie group homomorphism is a Lie group isomorphism if and only if it is bijective.
\end{corollary}

\begin{definition}
  Let $G$ be a Lie group. A \emph{Lie subgroup} of $G$ is a subgroup $H\subset G$ endowed with a topology and a smooth structure that make it at the same time a Lie group and an immersed submanifold of $G$.
\end{definition}

\begin{example}
  This means for example that the set $GL^+(n)$ of invertible matrices with positive determinant is a Lie group.
\end{example}

It turns out that embedded submanifolds are automatically Lie groups. In fact more than that.

\begin{theorem}[Closed subgroup theorem]
  Let $G$ be a Lie group and suppose $H$ is any subgroup of $G$.
  The following are equivalent:
  \begin{enumerate}
    \item $H$ is a closed subgroup\footnote{That is, $H$ is a closed subset of $G$.};
    \item $H$ is an embedded submanifold of $G$;
    \item $H$ is an embedded Lie subgroup of $G$.
  \end{enumerate}
\end{theorem}

The proof of this theorem is not hard, but especially proving the equivalence of the first two claims is rather involved, so we will skip it.
For a proof, look at the cooresponding section in \cite[Chapter 20]{book:lee}.

\begin{example}
  \marginnote{Since the closed subgroups of $GL(n)$ play a special role in Lie groups theory, they have their own name: they are the called \emph{matrix Lie group}.}
  Let $O(n)\subset GL(n)$ denote the set of orthogonal matrices\footnote{That is, $A$ such that $AA^T = I$.}, then $O(n)$ is closed in $GL(n)$ and by the previous theorem is a Lie subgroup.
  You have proven this when you solved Exercise~\ref{exe:onsubmanifold}.
\end{example}

\section{Lie algebras}

We are finally ready to see how Lie groups and Lie algebras ended up being related.

\begin{definition}
  Let $G$ be a Lie group.
  We define the \emph{Lie algebra} of $G$, usually denoted $\fg$, as the tangent space to $G$ at the identity element $e$:
  \begin{equation}
    \fg := T_e G.
  \end{equation}
\end{definition}

Of course, for this definition not to be completely insane, the Lie algebra of a Lie group better be a Lie algebra also in the sense of Definition~\ref{def:LieAlgebra}.
We are going to prove this very soon, but let's first look at some examples.

\begin{example}
  \begin{enumerate}
    \item The Lie algebra of $GL(n)$ is $\mathfrak{gl}(n)\simeq \mathrm{Mat}(n)$.
    \item The Lie algebra of $O(n)$ is $\mathfrak{o}{n} = \{A \in \mathfrak{gl}(n) \mid A+A^T = 0\}$. You have shown it in Exercise~\ref{exe:onsubmanifold}.
  \end{enumerate}
\end{example}

\begin{exercise}
  The Lie algebra of $\bT^n$ is $\R^n$.
  
  \textit{\small Hint: using the fact that $T(M\times N) \simeq T(M)\times T(N)$ and look at what happens in the case $n=1$.}
\end{exercise}