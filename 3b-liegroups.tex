In the previous chapter we have briefly touched upon the notion of Lie algebras.
A strictly related notion, we will see in which sense, is the notion of Lie group.
There are mathematical objects that are pervasive in mathematics, even outside the realm of differential geometry, and in physics, where they play an important role in classical mechanics\footnote{You may have heard of the celebrated Noether's theorem, which states that every smooth symmetry has a corresponding conservation law}, and in high-energy physics\footnote{Does gauge theory ring any bell?}.

The theory of Lie groups and Lie algebras is vast, and in these lectures we will just briefly scratch the surface.

\section{Lie groups}

\begin{definition}
  A \emph{Lie group $G$} is a smooth manifold (without boundary) that is also an algebraic group, with the property that the multiplication map $\mu: G\times G \to G$, $\mu:(g,h)\mapsto gh$, and the inversion map $\iota:G\to G$, $\iota: g\mapsto g^{-1}$ are smooth.
\end{definition}

\begin{example}
  \begin{enumerate}
    \item $\R^n$ is a Lie group under addition.
    \item $\R^n\setminus\{0\}$ is a Lie group under multiplication.
    \item A manifold can be equipped with different Lie group structures. For example, the following map
          \begin{equation}
            \mu(x,y) = (x^1+y^1,\, x^2+y^2,\, x^3+y^3+x^1y^2)
          \end{equation}
          induces\footnote{To see that this defines a group structure, identify $\R^3$ with upper triangular $3\times3$ matrices via
            \begin{equation}\nonumber
              x = (x^1, x^2, x^3) \mapsto \begin{pmatrix}
                1 & x^1 & x^3 \\
                0 & 1   & x^2 \\
                0 & 0   & 1
              \end{pmatrix}
            \end{equation}
            and observe that $m$ becomes the standard matrix multiplication.} an alternative structure of Lie group on $\R^n$ called \emph{Heisenberg group}.
    \item The set $GL(n)$ of invertible $n\times n$ matrices is a Lie group under matrix multiplication. Indeed, it is a manifold of dimension $n^2$, the product is smooth since each matrix entry is given a polynomial and the inversion is smooth thanks to Cramer's rule~\cite[Proposition B.36]{book:lee}.
    \item The $n$-torus $\bT^n = \R^n/\Z^n$ is an abelian Lie group with the group structure induced by addition on $\R^n$.
    \item Given Lie groups $(G_1, \ldots, G_k)$, their direct product is the product manifold $G_1\times \cdots\times G_k$ with the group structure given by
          \begin{equation}
            (g_1, \ldots, g_k)(h_1,\ldots,h_k) = (g_1h_1, \ldots g_kh_k)
          \end{equation}
          is a Lie group (why?).
    \item Not all smooth manifolds can be equipped with a Lie group structure: for example, $\bS^n$ admits a Lie group structure only for $n=0,1,3$.
  \end{enumerate}
\end{example}

\begin{exercise}
  Prove that any Lie group is orientable.
\end{exercise}

\begin{definition}
  A \emph{Lie group homomorphism} $F:G\to H$ is a smooth map which is also a group homomorphism. It is called \emph{Lie group isomorphism} if it is also a diffeomorphism, which implies that it has an inverse that is also a Lie group homomorphism. In this case we call $G$ and $H$ isomorphic Lie groups.
\end{definition}

\begin{example}
  It turns out that you know plenty of examples of Lie group homomorphisms.
  \begin{enumerate}
    \item The map $\exp:\R\to\R_+$ is a Lie group homomorphism. The image of $\exp$ is the open subgroup $\R_+$ and $\exp:\R\to\R_+$ is a Lie group isomorphism with inverse $\log:\R_+\to\R$.
    \item The map $\epsilon:\R\to\bS^1$ defined by $\epsilon(t)=e^{2\pi i t}$ is a Lie group homomorphism whose kernel is $\Z$. Similarly, the map $\epsilon^n:\R^n\to\bT^n$ defined by $\epsilon^n(x^1, \ldots, x^n)=(e^{2\pi i x^1}, \ldots, e^{2\pi i x^n})$ is a Lie group homomorphism whose kernel is $\Z^n$.
    \item The determinant function $\det:GL(n)\to\R\setminus\{0\}$ is smooth since $\det$ is a polynomial in the entries of the matrix and it is a Lie group homomorphism since $\det (AB) = \det(A) \det(B)$.
  \end{enumerate}
\end{example}

\begin{definition}
  If $G$ is a Lie group, for any element $g\in G$, we denote by $L_g:G\to G$ the \emph{left translation} and by $R_g:G\to G$ the \emph{right translation}, respectively defined
  \begin{equation}
    L_g(h) = gh
    \quad\mbox{and}\quad
    R_g(h) = hg.
  \end{equation}
\end{definition}

\marginnote{The fact that translations are diffeomorphisms of the groups onto itself is crucial, it implies that the group looks the same around any point. Indeed, they are \emph{homogeneous spaces}. To study the local structure of a Lie group, as we will see soon, it is enough to examine a neighbourhood of the identity element.}
These are both diffeomorphisms, since they can be described by a composition of smooth maps. For instance,
\begin{equation}
  \underset{h}{G} \underset{\mapsto}{\to} \underset{(g,h)}{G\times G} \underset{\mapsto}{\to} \underset{gh}{G}.
\end{equation}
Moreover, $L_{g^{-1}}$ is the inverse of $L_g$.
Similarly for $R_g$.

\begin{remark}
  For convenience, we will only consider left translations.
  There is nothing wrong with right translations and, in fact, you can reformulate all the results that follow in terms of them.
\end{remark}

The next theorem is important for understanding many of the properties of Lie group homomorphisms.

\begin{theorem}
  Every Lie group homomorphism has constant rank.
\end{theorem}
\begin{proof}
  Let $F:G\to H$ be a Lie group homomorphism and let $e$ denote the identity element of $G$.

  Fix $g\in G$.
  We will show that $F$ has the same rank at $g$ as its rank at $e$.
  Since $F$ is a homomorphism, for all $h\in G$ we have
  \begin{equation}
    F(L_g(h)) = F(gh) = F(g)F(h) = L_{F(g)}(F(h)),
  \end{equation}
  that is,
  \begin{equation}
    F\circ L_g = L_{F(g)}\circ F.
  \end{equation}
  Differentiating both sides at $e$ and using the chain rule, this reads
  \begin{equation}
    dF_g\circ d(L_g)_e = d(L_{F(g)})_{F(e)}\circ dF_e.
  \end{equation}
  Since the left translation is a diffeomorphism, both $d(L_g)_e$ and $d(L_{F(g)})_{F(e)}$ are isomorphisms, and as such they preserve the rank.
  From this, it follows that $dF_g$ and $dF_e$ have the same rank.
\end{proof}

The global rank theorem then immediately implies the following corollary.
\begin{corollary}
  A Lie group homomorphism is a Lie group isomorphism if and only if it is bijective.
\end{corollary}

\begin{definition}
  Let $G$ be a Lie group. A \emph{Lie subgroup} of $G$ is a subgroup $H\subset G$ endowed with a topology and a smooth structure that make it at the same time a Lie group and an immersed submanifold of $G$.
\end{definition}

\begin{example}
  This means for example that the set $GL^+(n)$ of invertible matrices with positive determinant is a Lie subgroup of $GL(n)$.
\end{example}

It turns out that embedded submanifolds are automatically Lie groups. In fact more than that.

\begin{theorem}[Closed subgroup theorem]
  Let $G$ be a Lie group and suppose $H$ is any subgroup of $G$.
  The following are equivalent:
  \begin{enumerate}
    \item $H$ is a closed subgroup\footnote{That is, $H$ is a closed subset of $G$.};
    \item $H$ is an embedded submanifold of $G$;
    \item $H$ is an embedded Lie subgroup of $G$.
  \end{enumerate}
\end{theorem}

The proof of this theorem is not hard, but especially proving the equivalence of the first two claims is rather involved, so we will skip it.
For a proof, look at the corresponding section in~\cite[Chapter 20]{book:lee}.

\begin{example}
  \marginnote{Since the closed subgroups of $GL(n)$ play a special role in Lie groups theory, they have their own name: they are the called \emph{matrix Lie group}.}
  Let $O(n)\subset GL(n)$ denote the set of orthogonal matrices\footnote{That is, $A$ such that $AA^T = I$.}, then $O(n)$ is closed in $GL(n)$ and by the previous theorem is a Lie subgroup.
  You have proven this when you solved Exercise~\ref{exe:onsubmanifold}.
\end{example}

\begin{exercise}\label{ex:SL2LGA}
  For this exercise is useful to remember that we can identify the space $\mathrm{Mat}(2,\R)$ of $2\times 2$-matrices with $\R^4$ by associating the matrix $X = \begin{pmatrix}x_{11} & x_{12}\\ x_{21} & x_{22}\end{pmatrix}$ with the point $(x_{11}, x_{12}, x_{21}, x_{22})\in\R^4$.

  \begin{enumerate}
    \item Show that the set
          \begin{equation*}
            \mathrm{SL}(2) := \mathrm{SL}(2,\R) = \{ A\in \mathrm{Mat}(2,\R) \;\mid\; \det A = 1 \}
          \end{equation*}
          is a 3-dimensional smooth submanifold of $\mathrm{Mat}(2,\R)$.
    \item Let $e\in\mathrm{Mat}(2,\R)$ denote the identity matrix. Show that
          \begin{equation*}
            T_e \mathrm{SL}(2) = \{ A\in \mathrm{Mat}(2,\R) \;\mid\; \mathrm{tr}\, A = 0 \},
          \end{equation*}
          where $\mathrm{tr} A$ denotes the matrix trace, i.e., the sum of the diagonal entries of $A$.
    \item Let $\iota: \mathrm{SL}(2)\to \mathrm{SL}(2)$ be the map $\iota(A) = A^{-1}$. Show that $\iota$ is smooth.
    \item Show that $d\iota_e: T_e\mathrm{SL}(2)\to T_e\mathrm{SL}(2)$ is given by $d\iota_e(A) = -A$.
  \end{enumerate}
\end{exercise}

In fact, some parts of the Exercise~\ref{ex:SL2LGA} above are instances of a them more general statement of Exercise~\ref{ex:DiffGroupMaps}.

\begin{exercise}\label{ex:DiffGroupMaps}
  Let $G$ be a Lie group.
  \begin{enumerate}
    \item Let $\mu:G\times G\to G$ denote the multiplication map.
          Use the identification $T_{(e,e)}(G\times G) \simeq T_eG\times T_eG$ to show that $d\mu_{(e,e)}:T_eG\times T_eG\to T_eG$ is given by
          \begin{equation}
            d\mu_{(e,e)}(X,Y) = X + Y.
          \end{equation}
          \textit{\small Hint: compute $d\mu_{(e,e)}(X,0)$ and $d\mu_{(e,e)}(0,Y)$ separately.}
    \item Let $\iota: G\to G$ denote the inversion map. Show that $d\iota_e:T_eG\to T_eG$ is given by $d\iota_e(X) = -X$.
  \end{enumerate}
\end{exercise}

\section{Lie algebras}

We are finally ready to see how Lie groups and Lie algebras ended up being related.

\begin{definition}
  Let $G$ be a Lie group.
  We define the \emph{Lie algebra} of $G$, usually denoted\marginnote{Sometimes you find $\mathrm{Lie}(G)$.} $\fg$, as the tangent space to $G$ at the identity element $e$:
  \begin{equation}
    \fg := T_e G.
  \end{equation}
\end{definition}

Of course, for this definition not to be completely insane, the Lie algebra of a Lie group better be a Lie algebra also in the sense of Definition~\ref{def:LieAlgebra}.
We are going to prove this very soon, but let's first look at some examples.

\begin{example}
  \begin{enumerate}
    \item The Lie algebra of $GL(n)$ is $\mathfrak{gl}(n)\simeq \mathrm{Mat}(n, \R)$.
    \item The Lie algebra of $O(n)$ is $\mathfrak{o}(n) = \{A \in \mathfrak{gl}(n) \mid A+A^T = 0\}$. You have shown it in Exercise~\ref{exe:onsubmanifold}.
    \item Can you guess what is the Lie algebra of $SL(2)$ from Exercise~\ref{ex:SL2LGA}?
  \end{enumerate}
\end{example}

\begin{exercise}
  Show that the Lie algebra of $\bT^n$ is $\R^n$.\\
  \textit{\small Hint: using the fact that $T(M\times N) \simeq T(M)\times T(N)$ and look at what happens in the case $n=1$.}
\end{exercise}

Before proceeding we need to introduce some more notation.

\begin{definition}
  Let $G$ be a Lie group.
  A vector field $X\in\fX(G)$ is called \emph{left-invariant} if
  \marginnote{That is, if for all $g,h\in G$ we have \begin{equation}
      d(L_g)_h X_h = X_{gh}.
    \end{equation}
    Indeed, ${X = (L_g)_* X = d(L_g)\circ X \circ (L_{g})^{-1}}$ if and only if $X \circ L_{g} = d(L_g)\circ X$.}
  \begin{equation}
    (L_g)_*X = X \qquad \forall g\in G.
  \end{equation}
  We denote the set of left-invariant vector fields by $\fX_L(G)\subset \fX(G)$.
\end{definition}

% \begin{proposition}
%   Let $G$ be a Lie group and let $e$ denote its identity element.
%   Then $X\in\fX_L(G)$ if and only if $X$ is uniquely identified with its value at $e$.
% \end{proposition}
% \begin{proof}
%   If $X$ is left-invariant, the by definition, for all $g\in G$ we have
%   \begin{equation}
%     X_g = (L_g)_* X_e.
%   \end{equation}
%   On the other hand, a vector field defined by $X_g = (L_g)_* v$ for some $v\in T_eG$ is automatically left-invariant.
% \end{proof}

% In particular, this also mean that left-invariant vector fields are always complete.

\begin{proposition}
  Let $G$ be a Lie group and $X,Y\in\fX_L(G)$.
  Then $[X,Y]\in\fX_L(G)$ and, therefore, $\fX_L(G)$ is a Lie subalgebra of $\fX(G)$.
\end{proposition}
\begin{exercise}
  Prove the proposition.
\end{exercise}

\begin{remark}
  The Lie algebra of all smooth left-invariant vector fields on a Lie group $G$, which we denoted $\fX_L(G)$ and is a subalgebra of $\fX(G)$, is also called the \emph{Lie algebra of $G$}.
  In the next theorem we are going to see that this is isomorphic to the one defined above in terms of tangent at the identity.
\end{remark}

A fundamental difference with the Lie algebra of vector fields is that $\fX_L(G)$ is finite dimensional.

\begin{theorem}
  Let $G$ be a Lie group. The evaluation map
  \begin{equation}
    \mathrm{eval}:\fX_L(G)\to T_eG, \quad
    \mathrm{eval}(X) = X_e,
  \end{equation}
  is a vector space isomorphism.
  Thus, $\fX_L(G)$ is finite dimensional with the same dimension as $G$.
\end{theorem}
\begin{proof}
  \newthought{Linearity}. Immediate (why?).

  \newthought{Injectivity}. Follows immediately from the left-invariance: if $\mathrm{eval}(X) = X_e = 0$ for some $X\in\fX_L(G)$, then the left-invariance of $X$ implies that $X_g = d(L_g)_e(X_e) = 0$ for every $g\in G$, thus $X\equiv0$.

  \newthought{Surjectivity}.
  Fix an arbitrary $v\in \fg = T_eG$ and define the map $v^L:G\to TG$ by
  \begin{equation}
    v^L(g) := d(L_g)_e(v).
  \end{equation}
  By construction, $v^L$ satisfies the section property\footnote{If you don't know what we are talking about, have another look at Definition~\ref{def:vfield}.}, since $\d(L_g)_e:T_eG\to T_gG$.
  \begin{itemize}
    \item $v^L$ is a vector field: we will show that $v^L_g f := v^L(g) f$ is smooth for any $f\in C^\infty(G)$.
          To this end, pick a smooth curve $\gamma:(-\epsilon,\epsilon)\to G$ such that $\gamma(0)=e$ and $\gamma'(0) = v$. Then, for any $g\in G$ we have
          \begin{align}
            v^L_g f & = v^L(g) f                      \\
                    & = d(L_g)_e(v) f                 \\
                    & = v(f\circ L_g)                 \\
                    & = (f\circ L_g\circ \gamma)'(0).
          \end{align}
          If we define $\varphi:(-\delta, \delta)\times G \to \R$ by $\varphi(t,g) = f\circ L_g\circ \gamma(t) = f(g\gamma(t))$, the computation above shows that $v^L_gf = \frac{\partial \varphi}{\partial t}(0, g)$.
          Since $\varphi$ is the composition of smooth functions, it is smooth, and thus $v^L f$ is smooth.
    \item $v^L$ is left-invariant.
          Indeed, for any $g,h\in G$, we have
          \begin{align}
            d(L_g)_h(v^L(h)) & = d(L_g)_h\circ d(L_h)_e(v) \\
                             & = d(L_g\circ L_h)_e(V)      \\
                             & = d(L_{gh})_e(v)            \\
                             & = v^L(gh).
          \end{align}
  \end{itemize}
  Thus $v^L\in\fX_L(G)$.
  Since $\mathrm{eval}(v^L) = v^L(e) = v$, the map $\mathrm{eval}$ is surjective, concluding the proof.
\end{proof}

\begin{corollary}
  Let $G$ be a Lie group of dimension $n$.
  Then \emph{its} Lie algebra is \emph{a} Lie algebra of dimension $n$.
\end{corollary}
\begin{proof}
  We just need to define a Lie bracket on $\fg$.
  But this is easier done than said: using the notation of the previous theorem, set
  \begin{equation}
    [v,w] := \mathrm{eval}([v^L, w^L]), \quad \forall v,w\in\fg.
  \end{equation}
\end{proof}

Another immediate consequence of this proposition is that every left-invariant vector field is complete, which immediately makes them all parallelizable.

\begin{proposition}\label{prop:XLcomplete}
  Let $G$ be a Lie group and $v\in\fX_L(G)$.
  Then $v$ is complete.
\end{proposition}
\begin{exercise}
  Prove the proposition.\\
  \textit{\small Hint: extend a curve starting at $e$ to a curve starting at $g$.}
\end{exercise}

\begin{corollary}
  Every Lie group admits a smooth global frame of  left-invariant vector fields, and therefore every Lie group is parallelizable.
\end{corollary}
\begin{proof}
  Every basis for $\fX_L$ is a left-invariant smooth global frame for $G$.
\end{proof}

Just as we can view the tangent space as a ``linear model'' of a smooth manifold near a point, the Lie algebra of a Lie group provides a ``linear model'' of the group, which reflects many of the properties of the group.
Because Lie groups have more structure than ordinary smooth manifolds, it should come as no surprise that their linear models have more structure than ordinary vector spaces.
Since a finite dimensional Lie algebra is a purely linear-algebraic object, it is in many ways simpler to understand than the group itself.
Much of the progress in the theory of Lie groups has come from a careful analysis of Lie algebras.

\begin{proposition}
  If $F: G\to H$ is a Lie group homomorphism\footnote{See Definition~\ref{def:LAmorphism}.}, then there is a map $F_*:\fg\to\fh$ which is a Lie algebra homomorphism. We call this map, the \emph{induced Lie algebra homomorphism}.
\end{proposition}
\begin{proof}
  Let $v\in\fg$ and let $v^L\in\fX_L(G)$ denote the unique left-invariant vector field satisfying $v^L_e = v$.
  Let $w := dF_e(v) =: F^*v$ and $w^L\in\fX_L(H)$ as above.
  It is enough to show that $w^L_{F(g)} = (F^*v^L)_g = dF_g V^L_g$ for all $g\in G$.

  Indeed, we have
  \begin{align}
    dF_g(v^L_g) & = dF_g\circ d(L_g)_e(v)          \\
                & =d(L_{F(g)})_{F(e)}\circ dF_e(v) \\
                & =d(L_{F(g)})_{F(e)}(w)           \\
                & =w^L_{F(g)}.
  \end{align}
  The result then follows from Theorem~\ref{thm:liealgiso}.
  If $v_1,v_2 \in\fg$ and $w_i=F^*v_i$, $i=1,2$, then
  \begin{equation}
    dF_e[v_1,v_2] = [w_1, w_2].
  \end{equation}
\end{proof}

An immediate consequence of this proposition is the following.
\begin{corollary}
  Let $H\subset G$ be a Lie subgroup.
  Then $\fh$ is a Lie subalgebra of $\fg$.
\end{corollary}
\begin{proof}
  Use the inclusion $i:H\hookrightarrow G$ as the homomorphism, then $di_e:\fh = T_eH\to \fg = T_eG$ is the Lie algebra homomorphism.
\end{proof}

If we go back to the example of $GL(n)$, now we have two possibly different Lie brackets on $\mathfrak{gl}(n)=\mathrm{Mat}(n, \R)$: the one coming from the previous corollary and the matrix commutator.
The next result, which we will not prove, shows that they coincide.

\marginnote{See~\cite[Proposition 8.41]{book:lee} for reference.}
\begin{proposition}
  The Lie bracket on $\mathfrak{gl}(n)$ is given by the matrix commutator.
  Therefore, if $G$ is a matrix Lie group, the Lie bracket on $\fg$ is also the matrix commutator.
\end{proposition}

In fact, the correspondence between Lie subgroups and Lie subalgebras goes both ways.

\marginnote{See~\cite[Theorem 8.46]{book:lee} for reference.}
\begin{theorem}
  Let $G$ be a Lie group with Lie algebra $\fg$. If $\fh$ is a Lie subalgebra of $\fg$, then there is a unique connected Lie subgroup $H$ of $G$ whose Lie algebra is $\fh$.
\end{theorem}

We close this section by stating a deep algebraic result about Lie algebras, whose proof is way out of our reach.
\begin{theorem}[Ado's theorem]
  Let $\mathfrak{gl}(V)$ denote the Lie algebra of linear maps from a finite dimensional vector space $V$ to itself.
  Every finite-dimensional real Lie algebra $\fg$ admits a faithful finite-dimensional representation, that is, there exists an injective Lie algebra homomorphism $F: \fg \to \mathfrak{gl}(V)\simeq\mathfrak{gl}(n,\R)$ for some finite dimensional vector space $V$.
\end{theorem}

\section{The exponential map}

We have seen that there is a tight relation between flows and exponentials, so much so, that we started using formally the exponential notation to denote flows of vector fields.
With Lie groups and Lie algebras, we will bring the construction to the next level, properly formalising the construction.

\begin{definition}
  \marginnote{There is a strict relation between one-parameter groups of diffeomorphisms and one-parameter subgroups of a Lie group. We will not discuss it here, just be aware that -- in some sense -- it mimicks what we are exploring here in the setting of infinite-dimensional Lie groups.}
  Let $G$ be a Lie group with Lie algebra $\fg$.
  We call a \emph{one-parameter subgroup} of $G$ a Lie group homomorphism $\R\to G$.
\end{definition}

Given the introduction, the following theorem should not come as a surprise.
\begin{theorem}
  Let $G$ be a Lie group.
  The one-parameter subgroups of $G$ are precisely the maximal integral curves of left-invariant vector fields starting at the identity.
\end{theorem}
\begin{proof}
  \begin{itemize}
    \item[($\Longleftarrow$)]
          Suppose that $\gamma$ is the maximal integral curve for some $v\in\fX_L(G)$ starting at the identity $e$.
          Proposition~\ref{prop:XLcomplete} implies that $\gamma$ is defined on all $\R$. Since $L_g$ is a diffeomorphism for all $g\in G$ and $v$ is left-invariant, by Proposition~\ref{prop:conjpfX} $L_g$ maps integral curves of $v$ to integral curves of $v$ (why?).
          If $g=\gamma(s)$ for some $s$, the curve $t\mapsto L_{\gamma(s)}(\gamma(t))$ is an integral curve starting at $\gamma(s)$.
          By the group property of the flow, also $t\mapsto \gamma(t+s)$ is an integral curve starting at $\gamma(s)$, so they must be equal.
          That is, for all $s,t\in \R$,
          \begin{equation}
            \gamma(s+t) = \gamma(s)\gamma(t).
          \end{equation}
          Which implies that $\gamma:\R\to G$ is a one-parameter subgroup.
    \item[($\Longrightarrow$)] Let now $\gamma:\R\to G$ be a one-parameter subgroup and $v = \gamma'(0)\in \fg$.
          The claim is that $\gamma'(t) = v^L(\gamma(t))$.
          Since $\gamma(s)\gamma(t) = \gamma(s+t) = L_{\gamma(t)}(\gamma(s))$ we have
          \begin{align}
            \gamma'(t) & = \frac{d}{ds}\Big|_{s=0} \gamma(t+s)              \\
                       & = \frac{d}{ds}\Big|_{s=0} L_{\gamma(t)}(\gamma(s)) \\
                       & = (dL_{\gamma(t)})_{\gamma(0)}(\gamma'(0))         \\
                       & = (dL_{\gamma(t)})_{e}(v)
            = v^L(\gamma(t)).
          \end{align}
          Again, due to uniqueness of the integral curves, we obtain the claim.
  \end{itemize}
\end{proof}

If we write $\Theta_t^v:=\varphi^{v^L}_t:G\to G$ for the flow of $v^L$, then by definition $\gamma^v = \Theta_t^v(e)$.

Note that the trick employed in the proof above, can be used also to show the following.
\begin{lemma}
  For any $s,t\in\R$ one has $\gamma^v(st) = \gamma^{sv}(t)$, where we used the superscript to specify the generator of the subgroup.
\end{lemma}

\begin{proposition}\label{prop:liflow}
  Let $G$ be a Lie group with Lie algebra $\fg$.
  Let $\gamma:\R\to G$ be a smooth curve with $\gamma(0)=e$ and $\gamma'(0)=v\in\fg$. Then the following claims are equivalent:
  \begin{enumerate}[(i)]
    \item $\gamma$ is a one-parameter subgroup;
    \item $\gamma(t) = \gamma^v(t)$ is the one-parameter subgroup generated by $v$;
    \item the flow $\Theta_t^v$ of $v^L$ is given by $\Theta_t^v = R_{\gamma(t)}$.
  \end{enumerate}
\end{proposition}
\begin{proof}
  We have already seen $(i) \Leftrightarrow (ii)$.
  To see $(iii)\Rightarrow(ii)$ observe that the first implies $\gamma^v(t) = \Theta_t^v(e) = R_{\gamma(t)}e = \gamma(t)$.

  Finally, assume $(ii)$ holds and fix $g\in G$.
  Since $v^L$ is left-invariant, $g\gamma^v = L_g \circ \gamma^v$ is another integral curve of $v^L$ starting at $g$, thus, again by uniqueness of integral curves, we have $R_{\gamma(t)}(g) = g\gamma^v(t) = \Theta_t^v(g)$. Which implies $(iii)$ by the arbitrariness of $g$.
\end{proof}

Given $v\in\fX_L(G)$, the one-parameter subgroup $\gamma^v$ determined by $v$ in this way is called the \emph{one-parameter subgroup generated by $v$}.
Because left-invariant vector fields are uniquely determined by their values at the identity, it follows that each one-parameter subgroup is uniquely determined by its initial velocity in $T_eG$, and thus there are one-to-one correspondences:
\begin{equation}
  \{\mbox{one-parameter subgroups of }G\}
  \leftrightarrow
  \fX_L(G)
  \leftrightarrow
  T_e G.
\end{equation}

The exponential map, is the map that will allow us to dissipate some of the mystery around these isomorphisms.

\begin{definition}
  Let $G$ be a Lie group and $\fg$ its Lie algebra.
  We define the \emph{exponential map of $G$} as the map
  \begin{equation}
    \exp: \fg \to G , \quad X \mapsto \gamma(1),
  \end{equation}
  where $\gamma$ is the one-parameter subgroup generated by $X$ or, equivalently, the integral curve of $X$ starting at the identity.
\end{definition}

\begin{exercise}
  Let $G$ be a Lie group. For any $X\in\fX_L(G)$, $\gamma(s) =\exp(sX)$ is the one-parameter subgroup of $G$ generated by $X$.
\end{exercise}

\begin{example}
  Proposition~\ref{prop:expmat} shows that the exponential map of $GL(n)$ is given by $\exp A = e^A$.
  This is where its name originated.
\end{example}

This is a corollary of the properties of flows, of group properties and of the previous propositions.
\begin{proposition}
  The exponential map $\exp: \fg \to G$ satisfies the following.
  For all $s,t\in\R$ and $v\in\fg$
  \begin{enumerate}
    \item $\exp$ is smooth;
    \item $\exp((s+t)v) = \exp(sv)\exp(tv)$;
    \item $\exp(-v) = (\exp(v))^{-1}$;
    \item $\exp(tv) =\gamma^v(t)$;
    \item the flow $\Theta_t^v$ of $v^L$ is given by $\Theta_t^v=R_{\exp(tv)}$.
  \end{enumerate}
\end{proposition}

The following property, on the other hand, deserves a bit more care.
\begin{theorem}
  The exponential map $\exp:\fg\to G$ is smooth.
  Moreover, up to the canonical isomorphism $T_0\fg = \fg$, the differential $d\exp_0$ at $0\in\fg$ is the identity.
\end{theorem}
\begin{proof}
  \newthought{Smoothness}. We need to show that $\Theta_1^v(e)$ depends smoothly on $v$.
  This is not covered by our previous analysis of flows, but can be via the following trick.
  Define a vector field $\nu$ on $G\times\fg$ by
  \begin{equation}
    \nu_{(g,v)} = (v^L_g, 0)\in T_gG\times T_v\fg \sim T_{(g,v)}(G\times \fg).
  \end{equation}
  Clearly $\nu$ satisfies the section property, so for it to be a smooth vector field, we only need to show that it is smooth.
  Pick any basis $(X_1,\ldots,X_k)$ for $\fg$ and let $(x^i)$ be the corresponding global coordinates for $\fg$ defined by $(x^i)\mapsto x^i X_i$.
  For any $f\in C^\infty(G\times\fg)$ and given a $v\in\fg$, let $f_v:= f(\cdot, v):G\to\R$ denote the smooth function defined by regarding $v$ as fixed.
  Then $\nu_{(g,v)}f = v^L_g f_v(g)$.

  Since $v^L$ depends linearly (and thus smoothly) on $v$ and $f$ is smooth in both $g$ and $v$, the expression $(g,v)\mapsto  v^L_g f_v(g)$ is smooth in both arguments. This confirms that $\nu_{(g,v)}$ is a vector field and, therefore, its flow $\Theta^\nu$, which by Proposition~\ref{prop:liflow} is given by
  \begin{equation}
    \Theta^\nu_t(g) := (g \exp(tv), v), \quad (t,g,v)\in\R\times G\times\fg,
  \end{equation}
  is smooth. In particular, $\Theta^\cdot_1(e)= (\exp(\cdot), \cdot): \fg\to G\times\fg$ is smooth and therefore $\exp$ itself is.

  \newthought{The differential $d\exp_0$}. We want to show that the following diagram commutes
  \begin{equation}\nonumber
    \begin{tikzcd}[row sep=large, column sep=large]
      T_0\fg \arrow[rr, "d\exp_0" description] & & \fg \\
      & \fg \arrow[ul, "\cT_0" description] \arrow[ur, "\id" description] &
    \end{tikzcd}
  \end{equation}
  where $\cT_0:\fg\to T_o\fg$ is the map from Exercise~\ref{ex:tg_curve_iso}.
  Let $v\in\fg$, then $\cT_0(v) = \delta'(t)$ where $\delta(t) = tv$.
  Thus we have
  \begin{align}
    d\exp_0 (\cT_0 v) & = (\exp \circ \delta)'(0)          \\
                      & = \frac{d}{dt}\Big|_{t=0} \exp(tv)
    = \frac{d}{dt}\Big|_{t=0} \gamma^v(t)
    = v,
  \end{align}
  which completes the proof.
\end{proof}
\begin{corollary}
  The exponential map is a diffeomorphism of some neighbourhood of the origin in $\fg$ onto its image in $G$.
\end{corollary}

Finally, let's investigate how the exponential map behaves with respect to Lie group homomorphisms.

\begin{proposition}
  Let $G$ and $H$ be Lie groups with Lie algebras $\fg$ and $\fh$ respectively.
  If $F:G\to H$ is a Lie group homomorphism, the following diagram commutes:
  \begin{equation}\nonumber
    \begin{tikzcd}[row sep=large, column sep=large]
      \fg \arrow[r, "F_*"] \arrow[d, "\exp" left] & \fh \arrow[d, "\exp"] \\
      G \arrow[r, "F"] & H
    \end{tikzcd}
  \end{equation}
\end{proposition}
\begin{proof}
  We need to show that $\exp(F_*v) = F(\exp(v))$ for every $v\in \fg$.
  Instead, we will show the stronger result that for all $t\in\R$, $\exp(t F_*v) = F(\exp(tv))$.
  The left-hand side is the subgroup generated by $F_* v$, thus if we put $\gamma(t) = F(\exp(tv))$, it is enough to show that $\gamma:\R\to H$ is a Lie group homomorphism such that $\gamma'(0) = (F_*v)_e$.
  A composition of homomorphisms is a homomorphism, therefore $t\mapsto F(\exp(tv))$ is a one-parameter subgroup of $H$.
  For the initial velocity, observe that
  \begin{align}
    (F\circ\exp)'(0) & = \frac{d}{dt}\Big|_{t=0} F(\exp(tv))               \\
                     & =dF_0\left(\frac{d}{dt}\Big|_{t=0} \exp(tv) \right) \\
                     & = dF_0(v_e) = (F_*v)_e.
  \end{align}
\end{proof}

\begin{remark}
  \marginnote[1em]{The complete formula is called Baker-Campbell-Hausdorff formula and its use appears all over the place in mathematics and physics~\cite{book:bonfigliolifulci}.}
  Note that we have not shown $\exp(X + Y) = (\exp X)(\exp Y)$.
  In fact, this is \emph{false} in general.
  As a matter of fact, $\exp X \exp Y = \exp Z$ where
  \begin{equation}
    Z = X + Y + \frac{1}2 [X,Y] + \frac1{12}[X,[X,Y]] - \frac1{12}[Y,[X,Y]] + \ldots,
  \end{equation}
  with the $\ldots$ indicating terms involving higher commutators of $X$ and $Y$.
\end{remark}

The one-parameter subgroups of $GL(n)$ follow nicely from the results introduced above.
Let $A\in\mathfrak{gl}(n)$. Using its identification with $\fX(GL(n))$ we can think of the matrix $A$ as the left-invariant vector field $A^L$.
That is, the one-parameter subgroup generated by $A$ is the integral curve of $A^L$ on $GL(n)$ starting at $e$.
This is a good place to see where the right shift is coming from.

Let $A = (A^i_j)$ and let $(X^i_j)$ denote the global coordinates on $GL(n)$ given by the matrix entries.
Then the natural isomorphism $T_{\id}GL(n) \simeq \mathfrak{gl}(n)$ is given by the mapping
\begin{equation}
  A^i_j \frac{\partial }{\partial X^i_j}\Big|_{\id} \mapsto (A^i_j).
\end{equation}
Thus, if you remember that $v^L|_g = d(L_g)_e(v)$, the left-invariant vector field $A^L$ is given by
\begin{equation}
  A^L|_X = d(L_X)_{\id}(A) = d(L_X)_{\id}\left(A^i_j \frac{\partial }{\partial X^i_j}\right),
\end{equation}
and, thus, in coordinates, its value at $X\in GL(n)$ is
\begin{equation}
  X^k_i A^i_j \frac{\partial }{\partial X^i_j}.
\end{equation}
Which means that the condition to be an integral curve, in coordinates, is $\frac{d}{dt}\gamma^i_j(t) = \gamma^i_k(t) A^k_j$ or, in matrix notation, $\gamma'(t) = \gamma(t) A$.
By using the expansion $e^{tA} = \sum_{k\geq 0} \frac{A^k}{k!}$ one can verify that such $\gamma$ is exactly the matrix exponential.

We can summarise\footnote{One should also prove convergence, if you are curious about the details you can refer to~\cite[Proposition 20.2]{book:lee}.} this as follows.

\begin{proposition}\label{prop:expmat}
  Let $A\in\mathfrak{gl}(n)$, then the one-parameter subgroup of $GL(n)$ generated by $A$ is
  \begin{equation}
    \exp(tA) = e^{tA} = \sum_{k\geq 0} \frac{A^k}{k!}.
  \end{equation}
\end{proposition}

\begin{corollary}
  Let $G$ be a matrix Lie group with Lie algebra $\fg$.
  Then the exponential map $\exp:\fg\to G$ is given by matrix exponentiation: $\exp A = e^A$.
\end{corollary}

\marginnote{This uncovers the structure behind the structure you encountered in Exercise~\ref{exe:onsubmanifold}.}
\begin{exercise}
  The one-parameter subgroups of $O(n)$ are the maps of the form $e^{tA}$ for arbitrary skew-symmetric matrices $A$.
  In particular, this shows that the exponential of any skew-symmetric matrix is orthogonal.
\end{exercise}

\begin{exercise}
  Let $G$ a Lie group and $\fg$ its Lie algebra.
  For any $X,Y\in\fg$, there show the following results.
  \begin{enumerate}
    \item For some $\epsilon>0$, there is a smooth function $Z:(-\epsilon, \epsilon)\to \fg$  such that, for all $t\in(-\epsilon, \epsilon)$,
          \begin{equation}
            (\exp tX)(\exp tY) = \exp(t (X+Y) + t^2 Z(t)).
          \end{equation}
    \item For some $\epsilon>0$, there is a smooth function $\widetilde Z:(-\epsilon, \epsilon)\to \fg$ such that, for all $t\in(-\epsilon, \epsilon)$,
          \begin{equation}
            (\exp tX)(\exp tY) = \exp\left(t (X+Y) + \frac12 t^2 [X,Y] + t^3 \widetilde Z(t)\right).
          \end{equation}
  \end{enumerate}
  \textit{\small Hint: Taylor expansions and Exercise~\ref{ex:DiffGroupMaps} can help for this exercise.}
\end{exercise}

The study of Lie groups acting on manifolds, opens a whole world of interesting topics, spanning across all fields of mathematics.
We will not enter into the details here, however I want to leave you the main definitions.

\begin{definition}
  Let $G$ a Lie group and let $M$ be a manifold.
  We call \emph{left action of $G$ on $M$} a smooth map $\ell: G\times M \to M$ such that
  \begin{equation}
    \ell(gh, p) = \ell(g,\ell(h,p)), \quad \ell(e, p) = p
  \end{equation}
  for all $g,h\in G$ and $p\in M$.
  For any fixed $g\in G$, the map $p\mapsto \ell(g,p)$ is a diffeomorphism of $M$, which is usually denoted $\ell_g$.
  Analogously, we call \emph{right action of $G$ on $M$} a smooth map $\rho: M\times G \to M$ such that
  \begin{equation}
    \rho(p, gh) = \rho(\rho(p,g), h), \quad \rho(p, e) = p
  \end{equation}
  for all $g,h\in G$ and $p\in M$.
  For any fixed $g\in G$, the map $p\mapsto \rho(p,g)$ is a diffeomorphism of $M$ called \emph{orbit map}, which is usually denoted $\rho^p$ or $(p,g)\mapsto p\cdot g$.
\end{definition}

There are many interesting types of Lie group actions. I am going to mention one here, which occurs when the action of $G$ on $M$ is transitive.
In this case $M$ becomes a \emph{homogeneous space} and is diffeomorphic to the quotient $G/H$ for some Lie subgroup $H \subset G$.

\begin{exercise}
  Let $G$ be a Lie group with Lie algebra $\fg$.
  For each $g\in G$, the differential at the identity of the conjugation map $C_g:=L_g\circ R_{g^{-1}}: G\to G$ is a linear isomorphism $C_{g*}:\fg\to\fg$.
  Hence, $C_{g*}\in GL(\fg)$.
  \begin{enumerate}
    \item Show that the map $\mathrm{Ad}:G\to GL(\fg)$ defined $\mathrm{Ad}(g) = C_{g*}$ and called \emph{adjoint representation of $G$} is a group homomorphism.
    \item Show that $\mathrm{Ad}:G\to GL(\fg)$ is smooth.
  \end{enumerate}
\end{exercise}

We will come back to discuss Lie groups and Lie algebras in the appendix on distribution theory and Frobenius theorem.
