\documentclass[nobib, a4paper]{tufte-book}
\setcounter{secnumdepth}{3}
\setcounter{tocdepth}{2}
\usepackage{microtype, ifluatex, ifxetex}
%Next block avoids bug, from  http://tex.stackexchange.com/a/200725/1913 
\ifx\ifxetex\ifluatex\else % if lua- or xelatex http://tex.stackexchange.com/a/140164/1913
  \usepackage{fontspec}
  \setmainfont[Renderer=Basic, Numbers=OldStyle, Scale = 1.0]{TeX Gyre Pagella}
  \setsansfont[Renderer=Basic, Scale=0.90]{TeX Gyre Heros}
  \setmonofont[Renderer=Basic]{TeX Gyre Cursor}

  % \setmainfont[Mapping=tex-text,Numbers=OldStyle]{Bembo Std}
  % \setsansfont[Mapping=tex-text,Numbers=OldStyle,Scale=MatchLowercase]{Gill Sans}
  % \setmonofont[Mapping=tex-text,Scale=MatchLowercase]{DejaVu Sans Mono}

  \renewcommand{\textls}[2][5]{%
    \begingroup\addfontfeatures{LetterSpace=#1}#2\endgroup
  }
  \renewcommand{\allcapsspacing}[1]{\textls[15]{#1}}
  \renewcommand{\smallcapsspacing}[1]{\textls[10]{#1}}
  \renewcommand{\allcaps}[1]{\textls[15]{\MakeTextUppercase{#1}}}
  \renewcommand{\smallcaps}[1]{\smallcapsspacing{\scshape\MakeTextLowercase{#1}}}
  \renewcommand{\textsc}[1]{\smallcapsspacing{\textsmallcaps{#1}}}
\fi

\usepackage{graphicx} % allow embedded images
  \setkeys{Gin}{width=\linewidth,totalheight=\textheight,keepaspectratio}
  \graphicspath{{images/}} % set of paths to search for images
\usepackage{amsmath,amssymb,amsthm,amsfonts,ulem,tikz}  % extended mathematics
\usepackage{booktabs} % book-quality tables
\usepackage{units}    % non-stacked fractions and better unit spacing
\usepackage{multicol} % multiple column layout facilities
\usepackage{fancyvrb,xcolor} % extended verbatim environments
  \fvset{fontsize=\normalsize}% default font size for fancy-verbatim environments
\usepackage{tikz,tikz-cd,bbm,mathbbol}
\DeclareSymbolFontAlphabet{\mathbbl}{bbold} %let's you use \mathbbl{k} for a field k
\hypersetup{colorlinks} %puts color to hyperlinks
\setcounter{secnumdepth}{2} 
\usepackage{enumerate}
\usepackage{mathabx}
\usepackage{mathtools}
\mathtoolsset{showonlyrefs, mathic}
\usepackage[english]{babel}
\usepackage{hyperref}
\hypersetup{
    colorlinks=true,    
    urlcolor=Cerulean,
    linkcolor = ForestGreen,
}
\usepackage[toc]{appendix}

\newcounter{dummy} %so that \pageref works properly
\usepackage[absolute]{textpos}
\setlength{\TPHorizModule}{\paperwidth} \setlength{\TPVertModule}{\paperheight}


\usetikzlibrary{decorations.pathreplacing,} %for braces with itemize
\newcommand{\tikzmark}[1]{\tikz[baseline={(#1.base)},overlay,remember picture] \node[outer sep=0pt, inner sep=0pt] (#1) {\phantom{A}};}
\usetikzlibrary{cd}

% Standardize command font styles and environments
\newcommand{\doccmd}[1]{\texttt{\textbackslash#1}}% command name -- adds backslash automatically
\newcommand{\docopt}[1]{\ensuremath{\langle}\textrm{\textit{#1}}\ensuremath{\rangle}}% optional command argument
\newcommand{\docarg}[1]{\textrm{\textit{#1}}}% (required) command argument
\newcommand{\docenv}[1]{\textsf{#1}}% environment name
\newcommand{\docpkg}[1]{\texttt{#1}}% package name
\newcommand{\doccls}[1]{\texttt{#1}}% document class name
\newcommand{\docclsopt}[1]{\texttt{#1}}% document class option name
\newenvironment{docspec}{\begin{quote}\noindent}{\end{quote}}% command specification environment

\newcommand{\adj}[4]{\begin{tikzcd}[ampersand replacement=\&, column sep=4ex]
					  	   #1 \colon #2	\ar[yshift=+.6ex]{r}
					  	\& #3 \colon #4	\ar[yshift=-.4ex]{l}
					 \end{tikzcd}}

\usepackage{etoolbox}
\newcommand{\addQEDstyle}[2]{
  \AtBeginEnvironment{#1}{\pushQED{\qed}\renewcommand{\qedsymbol}{#2}}
  \AtEndEnvironment{#1}{\popQED}
}

\theoremstyle{plain}
\newtheorem{theorem}{Theorem}[section]
\newtheorem{corollary}[theorem]{Corollary}
\newtheorem{proposition}[theorem]{Proposition}
\newtheorem{lemma}[theorem]{Lemma}

\theoremstyle{definition}
\newtheorem{definition}[theorem]{Definition}
\addQEDstyle{definition}{$\lozenge$}
\newtheorem{conjecture}[theorem]{Conjecture}
\addQEDstyle{conjecture}{$(\clubsuit)uit$}

\theoremstyle{remark}
\newtheorem{example}[theorem]{Example}
\addQEDstyle{example}{$\lozenge$}
\newtheorem{remark}[theorem]{Remark}
\addQEDstyle{remark}{$\lozenge$}
\newtheorem{notation}[theorem]{Notation}
\addQEDstyle{notation}{$\lozenge$}

\usepackage{oplotsymbl}
\newtheorem{exercise}[theorem]{Exercise}
\addQEDstyle{exercise}{$\starlet$}
\newtheorem{problem}[theorem]{Problem}
\addQEDstyle{problem}{$\starlet$}

\usepackage[
    type={CC},
    modifier={by-nc-sa},
    version={4.0},
]{doclicense}

\usepackage[dvipsnames]{xcolor}
\usepackage[many]{tcolorbox}

%\usepackage[numbers, sort]{natbib}
\usepackage[style=alphabetic,natbib=true,sorting=anyt]{biblatex}
%\setlength{\bibsep}{3pt}
%\renewcommand{\bibfont}{\small}
\usepackage{doi}
\addbibresource{aom.bib}

\newcommand{\cA}{\mathcal{A}}
\newcommand{\cC}{\mathcal{C}}
\newcommand{\cD}{\mathcal{D}}
\newcommand{\cE}{\mathcal{E}}
\newcommand{\cL}{\mathcal{L}}
\newcommand{\cO}{\mathcal{O}}
\newcommand{\cH}{\mathcal{H}}
\newcommand{\cI}{\mathcal{I}}
\newcommand{\cT}{\mathcal{T}}
\newcommand{\cB}{\mathcal{B}}
\newcommand{\cU}{\mathcal{U}}
\newcommand{\cX}{\mathcal{X}}

%\newcommand{\bC}{\mathbb{C}}
\newcommand{\N}{\mathbb{N}}
\newcommand{\Z}{\mathbb{Z}}
\newcommand{\Q}{\mathbb{Q}}
\newcommand{\R}{\mathbb{R}}
\newcommand{\RP}{\mathbb{RP}}
\newcommand{\bS}{\mathbb{S}}
\newcommand{\bT}{\mathbb{T}}

\newcommand{\fX}{\mathfrak{X}}
\newcommand{\fg}{\mathfrak{g}}
\newcommand{\fh}{\mathfrak{h}}

\newcommand{\bx}{\bm{x}}
\newcommand{\bp}{\bm{p}}
\newcommand{\bv}{\bm{v}}

\DeclareFontFamily{U}{MnSymbolC}{}
\DeclareSymbolFont{MnSyC}{U}{MnSymbolC}{m}{n}
\DeclareFontShape{U}{MnSymbolC}{m}{n}{
    <-6>  MnSymbolC5
   <6-7>  MnSymbolC6
   <7-8>  MnSymbolC7
   <8-9>  MnSymbolC8
   <9-10> MnSymbolC9
  <10-12> MnSymbolC10
  <12->   MnSymbolC12}{}
\DeclareMathSymbol{\iprod}{\mathbin}{MnSyC}{'270}

\let\d\relax
\DeclareMathOperator{\d}{d}
\DeclareMathOperator{\D}{D}
\DeclareMathOperator{\diff}{Diff}
\DeclareMathOperator{\Id}{Id}
\DeclareMathOperator{\id}{id}
\DeclareMathOperator{\diag}{diag}
\let\mod\relax
\DeclareMathOperator{\mod}{mod}
\DeclareMathOperator{\curl}{curl}
\DeclareMathOperator{\Vol}{Vol}
\DeclareMathOperator{\eval}{eval}
\DeclareMathOperator{\supp}{supp}
\DeclareMathOperator{\sgn}{sgn}

\DeclareMathOperator{\Alt}{Alt}

\DeclareMathOperator{\ad}{ad}
\DeclareMathOperator{\Ad}{Ad}

\newcommand{\todo}[1]{\footnote{\textcolor{red}{#1}}}
\newcommand{\TODO}{\textcolor{red}{\hrulefill}}

\title{Analysis\\ \noindent
on\\ \noindent
Manifolds
}
\author{Marcello Seri}
\publisher{Bernoulli Institute\\ \noindent
A.Y. 2020--2021\\ \noindent 
\MakeLowercase{\texttt{m.seri@rug.nl}}
}

\begin{document}
\maketitlepage

\newpage

\begin{fullwidth}
    ~\vfill
    \thispagestyle{empty}
    \setlength{\parindent}{0pt}
    \setlength{\parskip}{\baselineskip}
    Copyright \copyright\ \the\year\ \thanklessauthor
    
    \par Version 0.9.8 -- \today

    \vfill
    \small{\doclicenseThis}
\end{fullwidth}
    
\pagenumbering{roman}
\tableofcontents
\cleardoublepage

\pagenumbering{arabic}
\chapter*{Introduction}
\addcontentsline{toc}{chapter}{Introduction}

At the entry for \emph{Mathematical analysis}, our modern source of truth -- Wikipedia -- says

\begin{quotation}
  \emph{Mathematical analysis} is the branch of mathematics dealing with limits and related theories, such as differentiation, integration, measure, infinite series, and analytic functions.

  These theories are usually studied in the context of real numbers and functions. Analysis evolved from calculus, which involves the elementary concepts and techniques of analysis. Analysis may be distinguished from geometry; however, it can be applied to any space of mathematical objects that has a definition of nearness (a topological space) or specific distances between objects (a metric space). 
\end{quotation}

\newthought{In this sense}, our course will focus on generalizing the concepts of differentiation, integration and, up to some extent, differential equations on spaces that are more general than the standard Euclidean space.

This said, the Euclidean space $\R^n$ is \emph{the} prototype of all manifolds: it won't just be our simplest example, we will see that locally every manifold looks like a Euclidean space.

Euclidean spaces, and the Riemannian charts that you encountered in the \href{http://www.rolandvdv.nl/G20/}{Geometry course}, have a very strong property: they can be described with a set of \emph{global} coordinates.
Even though this means that all computations are explicit, it does make it harder to distinguish \emph{intrinsic}\footnote{I.e. independent from the choice of coordinates.} concepts.
Manifolds will force our hand to work in a \emph{coordinate-free} setting.
We will see that this will unleash a surprising power that will allow us to lay the foundation for a lot of the mathematics that will come in the rest of the curriculum.

These notes will focus on fundamental methods of differential geometry, in particular we will discuss manifolds, differential forms, integration, geometry of submanifolds, real and complex vector bundles, connections.
Throughout the course and the text, I will try to give particular emphasis on the usefulness of these topics in the mathematical theories of classical and quantum mechanics.

If the time permits it, we will give a brief tour of Riemannian metrics and the notion of curvature or of distributions and Frobenius theorem, depending on the preferences expressed in class.
This part of the material will not necessarily be part of the lecture notes\footnote{I will update this paragraph, if needed, in due course.}.

The course relies \emph{heavily} on your knowledge of linear and multilinear algebra, multivariable analysis\footnote{Make sure to review the material of \href{http://www.rolandvdv.nl/M19/}{Multivariable Analysis} before the course begins} and dynamical systems.
This should not come as a surprise: differential geometry and classical mechanics were born together as unique discipline, part of mathematical physics, before the various communities started diverging on their own paths.

An old mathematical joke says that
\begin{quote}
  differential geometry is the study of properties that are invariant under change of notation.
\end{quote}
Sadly, this is \emph{funny because it is alarmingly close to the truth}\footnote{Cit. Lee~\cite{book:lee}.}
You will soon see that different references use different notations. I'll try to stick to the ones you used in the past courses when possible, falling back to~\cite{book:lee} and~\cite{book:tu} and to my personal preference when the latter disagree.

\marginnote{In addition to the reference books, these lecture notes have found deep inspiration from~\cite{lectures:merry,lectures:teufel,lectures:hitchin} (all freely downloadable from the respective authors' websites), and from the book~\cite{book:abrahammarsdenratiu}.}
\newthought{These lecture notes} are by no means comprehensive.
As a reference you can use to the former course textbook~\cite{book:tu} or you can refer to~\cite{book:lee}: it is an incredibly good textbook and contains all the material of the course and much more.
I have requested for~\cite{book:tu} book to be freely available via SpringerLink using the university proxy but this will take some time to become active.
However, you can already freely access Lee's book via the University proxy on \href{https://link.springer.com/book/10.1007/978-1-4419-9982-5}{SpringerLink} and it will provide a very good and extensive reference for this and other future courses.
The book~\cite{book:McInerney} is a nice compact companion that develops most of this course concept in the specific case of $\R^n$ and could provide further examples and food for thoughts.
A colleague recently mentioned also~\cite{book:lang}. I don't know this book but from a brief look it seems to follow a similar path as these lecture notes, so might provide an alternative reference after all.

The idea for the cut that I want to give to this course was inspired by the online \href{https://www.video.uni-erlangen.de/course/id/242}{Lectures on the Geometric Anatomy of Theoretical Physics} by Frederic Schuller, by the lecture notes of Stefan Teufel's Classical Mechanics course~\cite{lectures:teufel} (in German), by the classical mechanics book by Arnold~\cite{book:arnold} and by the Analysis of Manifold chapter in~\cite{book:thirring}.
In some sense I would like this course to provide the introduction to geometric analysis that I wish was there when I prepared my \href{https://www.mseri.me/lecture-notes-hamiltonian-mechanics/}{first edition} of the Hamiltonian mechanics course.

I am extremely grateful to Martijn Kluitenberg for his careful reading of the notes and his useful comments and corrections.
Many thanks also to Huub Bouwkamp, Bram Brongers, Mollie Jagoe Brown, Wietze Koops, Nicol\'as Moro, Luuk de Ridder, Lisanne Sibma, Jordan van Ekelenburg, Hanneke van Harten and Dave Verweg for reporting a number of misprints and corrections.

\mainmatter

\chapter*{Einstein summation convention}
\addcontentsline{toc}{chapter}{Einstein summation convention}

As will become clear soon, sums of the type $\sum_i x^i e_i$ are unavoidably appearing all over the place when working on manifolds.
Therefore, throughout these notes we will apply the \emph{Einstein summation convention}: if the same index\footnote{For example, $i$ in the summation $\sum_i x^i e_i$.} appears exactly twice in a monomial term, once in the lower and once in the upper index position, then that term is understood to be summed over all possible values of that index\footnote{Usually from $1$ to the dimension of the space in question.}.

For instance, the expression
\begin{equation}
  a^{ij}b_l^k e_i e_k
\end{equation}
is a shorthand for
\begin{equation}
  \sum_{i,k} a^{ij}b_l^k e_i e_k.
\end{equation}

In general, we will use lower indices for basis of vector spaces\footnote{E.g., $(e_1,\ldots,e_n)$ could be the standard basis of $\R^n$.}, and upper indices for the components of a vector with respect to a basis\footnote{E.g., the $i$th-coordinate $x^i$ of $x\in\R^n$.}.
\marginnote[10pt]{Since the coordinates of a point $x\in\R^n$ are also its components with respect to the standard basis $(e_1, \ldots, e_n)$, for consistency they will be denoted $(x^1, \ldots, x^n)$ with upper indices.}

Note that an upper index ``in the denominator'' is regarded as a lower index, so the following are to be considered equivalent:
\begin{equation}
  \sum_{i} a^i \frac{\partial}{\partial x^i} = a^i \frac{\partial}{\partial x^i}.
\end{equation}
In fact, the expressions below are all equivalent and commonly used in the differential geometry literature:
\begin{equation}
  \sum_{i} a^i \frac{\partial}{\partial x^i} = a^i \frac{\partial}{\partial x^i} = a^i \partial_{x^i} = a^i \partial_i.
\end{equation}

\chapter{Manifolds}\label{ch:manifolds}
\newthought{In the first two years} of your mathematical education, you have become familiar with calculus for functions and vector fields on $\R^n$.
As I mentioned in the introduction, euclidean spaces will be our prototypical example.
However, the generalization of calculus to curved spaces will require us to carefully isolate the mathematical structures associated to the various concepts.
This process will help us to discover the rich geometric structure that lies at the root of derivation and integration, which ultimately is of great mathematical interest and has revolutionized mathematical physics.

If you think carefully, this abstraction step was already in the air. Think about the concept of continuity.

\begin{enumerate}[a)]
  \item (High school) A function $f:\R\to\R$ is \emph{continuous} if you can draw it without lifting your pen from the page.
  Then, the derivative $f'(x)$ of $f$ at a point $x$ is just the slope of the function $f$ at the point $x$.
  
  \item (Analysis) A function is continuous if its left and right limits at each point exist and have the same value.
  Then, $f:\R\to\R$ is \emph{differentiable} at a point $x$ if the limit
  \begin{equation}
    f'(x) := \lim_{h\to0} \frac{f(x+h) - f(x)}{h} 
  \end{equation}
  exists, and is \emph{continuously differentiable} if $x\mapsto f'(x)$ is itself a continuous function.
  
  \item (Multivariable analysis) You generalized the concepts to functions with more than one variable.
  Continuity is practically unchanged but, now, a continuous function $f=(f^1, \ldots, f^m):\R^n\to\R^m$ is differentiable at $x=(x^1,\ldots,x^m)\in\R^n$ if there is a \emph{linear map}\footnote{That is, $T$ is a $m\times n$ matrix.} $T: \R^n\to\R^m$ such that
  \begin{equation}\label{eq:diff}
    \lim_{\|h\|\to 0} \frac{\|f(x+h) - f(x) - T h\|}{\|h\|} = 0.
  \end{equation}
  The map $Df(x) := T$ is the \emph{differential} (or total derivative) of $f$ and is nothing else than the Jacobian matrix of $f$ at the point $x$, that is
  \begin{equation}\label{eq:jacobian}
    Df(x) = \begin{pmatrix}
      \frac{\partial f^1}{\partial x^1}(x) & \cdots & \frac{\partial f^1}{\partial x^n}(x) \\
      \vdots & \ddots & \vdots \\
      \frac{\partial f^m}{\partial x^1}(x) & \cdots & \frac{\partial f^m}{\partial x^n}(x) \\
    \end{pmatrix}.
  \end{equation} 
  The notion of continuous differentiability is unchanged\footnote{Note how the spaces are changing though: since it takes values in the space of $m\times n$ matrices, the differential $x\mapsto Df(x)$ is in fact a mapping of $\R^n \to \R^{mn}$.}, and in fact for $m=n=1$ it coincides with the one you gave for real functions.
  
  \item (Metric and topological spaces) A map $f:X\to Y$ between \emph{topological spaces} is continuous if preimages of open sets under $f$ are open. More explicitly, $f$ is continuous if for every open set $O\subset Y$, $f^{-1}(O)\subset X$ is an open set.

  If $X$ and $Y$ are \emph{metric spaces}, then this reduces to the definition given above.
  But how can we make sense of differentiability in this case? 
  
  If you have taken a course on calculus of variations, you know that you can make sense of \eqref{eq:diff} and give a notion of differentiability in the case $X$ and $Y$ are Banach spaces\footnote{Complete normed vector spaces.}.
  In general, a topological space is \emph{not} a vector space: there is no notion of adding points, least of all one of linearity.
\end{enumerate}

This is where differential geometry comes into play.
The rest of this chapter will be devoted to the introduction of \emph{smooth manifolds}, which are a class of topological spaces on which it is possible to make sense of the notion of differentiation even though they are not necessarily vector spaces.
We will do this in two stages.
First we will introduce \emph{topological manifolds}, which are topological spaces that \emph{locally} look like euclidean spaces.
Then we will endow topological manifolds with a so-called \emph{smooth structure}.
This will allow us to define differentiability and \emph{smooth manifolds}\footnote{These will just be topological manifolds with a smooth structure.}.

Without further ado, let's get started.

\section{Topological manifolds}

\newthought{Since to speak of continuity we need topological spaces}, it may be a good idea to remind you what they are and set some notation.
I will be very brief: if you need a more extensive reminder, you can refer to Appendix A of either \cite{book:tu} or \cite{book:lee}.

\begin{defn}
  Let $M$ be some set and $\cT$ a set of subsets of $M$.
  A pair $(M, \cT)$ is a \emph{topological space}\footnote{In such case the elements $O\in\cT$ of $\cT$ are all subsets of $M$ called \emph{open} subsets and $\cT$ is a topology on $X$.} if
  \begin{enumerate}[(i)]
    \item $M$ and $\emptyset$ are open, i.e., $M\in \cT$ and $\emptyset\in\cT$;
    \item arbitrary unions of families of open subsets are open;
    \item the intersection of finitely many\footnote{It is equivalent to require the intersection of any two open subsets to be open. (Why?)} open subsets is open.
  \end{enumerate}
\end{defn}

With topological spaces at hand, we can give a definition of continuity and introduce a way to compare topological spaces.

\begin{defn}
  A map $f: X \to Y$ between two topological spaces $(X,\cT)$ and $(Y, \cU)$ is called:
  \begin{itemize}
    \item \emph{continuous} if $U\in\cU$ implies that $f^{-1}(U)\in\cT$, that is, preimages of open sets under $f$ are open;
    \item \emph{homeomorphism} if it is bijective\footnote{I.e., one-to-one.} and continuous with continuous inverse.\marginnote{The existence of an homeomorphism between two spaces can be thought as those spaces being equivalent in a loose sense: they can be deformed continuously into each other.}
  \end{itemize}
\end{defn}

\begin{marginfigure}
  \includegraphics{images/1_1-dount-to-cup.pdf}
  \vspace{5pt}
\end{marginfigure}

\begin{defn}
  A topological space $(X, \cT)$ is \emph{Hausdorff} if every two distinct points admit disjoint open neighborhoods. That is, for every pair $x\neq y$ of points in $X$, there exist open subsets $U_x, U_y\in\cT$ such that $x\in U_x$, $y\in U_y$ and $U_x \cap U_y = \emptyset$.
\end{defn}

Topological spaces are extremely general, as such they may have very inconvenient -- someone would say nasty -- properties.
You can see this for yourself with the following exercise.

\begin{exe}
Let $X$ be an arbitrary set. Show that $\cT:=\{\emptyset, X\}$ defines a topology on $X$, called the \emph{trivial topology}. Show that on $(X, \cT)$ any sequence in $X$ converges to every point of $X$, and every map from a topological space into $X$ is continuous.
\end{exe}

Hausdorff spaces are still rather general: in particular, any metric space with the metric topology\footnote{Recall that in a metric space $X$ the \emph{metric topology} is defined in the following way: a set $U\subset X$ is called open if for any $x\in U$ there exists $\epsilon>0$ such that $U$ fully contains the ball of radius $\epsilon$ around $x$.} is Hausdorff.

\begin{defn}
  A topological space $(X, \cT)$ is \emph{second countable} if there exists a countable set $\cB\subset\cT$ such that any open set can be written as a union of sets in $\cB$.
  In such case, $\cB$ is called a (countable) basis for the topology $\cT$.
\end{defn}

\begin{exe}[Euclidean space $\R^n$]\label{exe:rntopsp}
  Let's consider on $\R^n$ the metric topology\footnote{See comment above.} induced by the Euclidean metric $d: \R^n \times \R^n \to [0, +\infty)$, $d(x,y) := \sqrt{\sum_{i=1}^n (x^i-y^i)^2}$.
  The topological space defined on $\R^n$ is Hausdorff and second countable.
\end{exe}

\begin{defn}[Topological manifold]
  A topological space\footnote{From now on, if we say that $X$ is a topological space we are implying that there is a topology $\cT$ defined on $X$.} $M$ is a \emph{topological manifold} of dimension $n$, or topological $n$-manifold, if it has the following properties
  \marginnote[0.5em]{Note that the finite dimensionality is a somewhat artificial restriction: manifolds can be infinitely dimensional. For example, the space of continuous functions between manifolds is a so-called infinite-dimensional Banach manifold.\vspace{1em}}
  \begin{enumerate}[(i)]
    \item $M$ is a Hausdorff space;
    \item $M$ is second countable;
    \item $M$ is \emph{locally euclidean} of dimension $n$, that is\footnote{In words, any point $p\in M$ has a neighborhood that is homeomorphic to an open subset of $\R^n$.}, for any point $x\in M$ there exist an open subset $U\subset M$ with $p\in U$, and open subset $V\in\R^n$ and a homeomorphism $\phi: U\to V$.
  \end{enumerate}
\end{defn}

\begin{ntn}
  Reusing the notation of the definition above, we call \emph{(coordinate) chart} the pair $(U, \phi)$ of a \emph{coordinate neighborhood} $U$ and an associated \emph{coordinate map}\footnote{Or \emph{coordinate system}.} $\phi: U\to V$ onto an open subset $U=\phi(V)\subseteq\R^n$ of $\R^n$.
  Furthermore, we say that a chart is \emph{centered at $p\in U$} if $\phi(p) = 0$.
\end{ntn}

Don't get scared by the first two conditions: they are only needed to make sure that there are not too few open sets (Hausdorff) and not too many (second countable).

\begin{ex}
  With our definition, a countable collections of points with the discrete topology is a $0$-dimensional topological manifolds.
  An uncountable collection of points with the discrete topology, however, is not!
\end{ex}

\begin{ex}
  $\R^n$ is trivially\footnote{Use Exercise \ref{exe:rntopsp} and the \emph{global} chart $(\R^n, \id_{\R^n})$, where $\id_{\R^n}(x) := x$ is the identity on $\R^n$.} a topological manifold of dimension $n$.
  More generally, any $n$-dimensional vector space\footnote{In fact, any subset of a $n$-dimensional vector space.} is a topological $n$-manifold.
\end{ex}

\begin{exe}
	Even though $\R^n$ with the euclidean topology is Hausdorff, being Hausdorff does not follow from being locally euclidean. A famous counterexample is the following\footnote{See also Problem 5.1 in \cite{book:tu}.}.
  \begin{marginfigure}
    \includegraphics{1_ex_1_0_11.pdf}
    \label{fig:hausdorff-not-locally-euclidean}
    \caption{A locally euclidean space which is not Hausdorff.}
  \end{marginfigure}
  Let $A_1, A_2$ be two point not on the real line $\R$ and define $M:= (\R\setminus\{0\})\cup\{A_1,A_2\}$.
  Define the two charts
  \begin{equation}
  \phi_j:(\R\setminus\{0\})\cup\{A_j\} \to \R, \quad
  \phi_j(x) = \begin{cases} x &\mbox{if } x\neq A_j\\ 0 & \mbox{if } x = A_j \end{cases}, \quad
  j = 1,2.
  \end{equation}
  \begin{enumerate}[(a)]
    \item Show that $\phi_1$ and $\phi_2$ are homeomorphisms with respect to the induced topology\footnote{Let $(X, \cT)$ be a topological space and $f: X\to Y$ some map. The induced topology on $Y$ is \begin{equation}\cU_f := \{f^{-1}(U) \;\mid\; U\in\cT\}.\end{equation}}.
    \item Show that $M$ is locally euclidean and second countable but not Hausdorff.
  \end{enumerate}
\end{exe}

\begin{ex}\label{ex:uball}
  The \emph{closed} unit ball $D_1(0)$, where similarly as before
  \begin{equation}
    D_r(x) := \{z\in\R^n \;\mid\; d(z,x) \leq r\},
  \end{equation}
  is \emph{not} a topological manifold of dimension $n$. Can you see why? In fact, this is an example of a more general concept of \emph{manifold with boundary} that we will introduce later in Chapter~\ref{sec:mbnd}.
\end{ex}

\begin{ex}
	Consider the set $M := \{ x\in\R^2 \;\mid\; |x^1| = |x^2| \}$ with the topology induced by $\R^2$:
   this is \emph{not} a topological manifold.
	Since the number of connected component is invariant under homeomorphisms, open connected neighborhoods of $(0,0)\in M$ cannot be\footnote{A drawing of $M$ is worth more than a hundred words.} holomorphically mapped to open connected sets in $\R$.
\end{ex}

\marginnote{There is a caveat, the theorem holds for \emph{connected} components of a Manifold. If you consider two distinct connected components, you can indeed have different dimensions for each of them.}
\newthought{There is still an elephant in the room} in need of a comment.
In our definition of topological manifolds, we are giving for granted that the dimension of the manifold is well defined, that is, if we have two different charts, $\phi_1: U \to \R^n$ and $\phi_2: U \to \R^m$, then necessarily $m=n$. Luckily this is true! The result is called \emph{Invariance Domain Theorem} and, since its proof requires advanced concepts of algebraic topology, we will not pursue it further in the course.

\section{Differentiable manifolds}

Before entering into the details of new definitions, let's recall what will be the most important tools throughout the rest of the course.

\begin{defn}
  A map $f: U \to V$ between open sets $U\subset\R^n$ and $V\subset\R^m$ is in $C^r(U,V)$ or \emph{of class $C^r$}, if it is differentiable $r$-times.
  It is called a $C^r$-diffeomorphism\footnote{With this definition a homeomorphism is a $C^1$-diffeomorphism} if it is bijective and of class $C^r$ with inverse of class $C^r$.
  We say that $f$ is \emph{smooth}, or of class $C^\infty$, if it is of class $C^r$ for every $r \geq 1$.
\end{defn}

\begin{thm}[Chain rule]\label{thm:chainrule}
Let $U\subset\R^n$ and $V\subset\R^k$ be open sets and $f: U \to \R^k$, $g: V\to\R^m$ two continuously differentiable functions such that $f(U)\subset V$.
Then, the following holds.
\begin{enumerate}[(i)]
  \item\label{thm:chainrule1} The function $g\circ f: U\subset\R^n \to\R^m$ is continuously differentiable and its total derivative \eqref{eq:jacobian} at a point $x\in U$ is given by
\begin{equation}
  D(g\circ f)(x) = D(g(f(x)) \circ Df(x).
\end{equation}
\item\label{thm:chainrule2} Denote $x=(x^1, \ldots, x^n)\in\R^n$ and $y=(y^2,\ldots,y^k)\in\R^k$ the coordinates on the respective euclidean spaces and $f=(f^1,\ldots,f^k)$ and $g=(g^1,\ldots,g^m)$ the components of the functions. Then the partial derivatives of $g\circ f$ are given by
\begin{equation}
  \frac{\partial g^i\circ f}{\partial x^j}(x)
  = \sum_{r=1}^k \frac{\partial g^i}{\partial y^r}(f(x)) \frac{\partial f^r}{\partial x^j}(x),
\qquad 1\leq i \leq m,\; 1\leq j\leq n.
\end{equation}
\end{enumerate}
\end{thm}

\begin{rmk}
Theorem~\ref{thm:chainrule} has some very deep consequences.
First of all \ref{thm:chainrule} \ref{thm:chainrule2} tells us that that composition preserves the regularity, that is the composition of functions  of class $C^r$ is itself a function of class $C^r$.
Second, if $f:U\subset\R^n\to V\subset\R^m$ is a diffeomorphism, then \ref{thm:chainrule} (\ref{thm:chainrule1}) implies that $n=m$.
\marginnote[-1.5em]{It follows from the fact that $Df(x)$ is an invertibe matrix and with inverse is $D(f^{-1})(f(x))$.}
\end{rmk}

Since differentiability is a \emph{local} property and topological manifolds are \emph{locally} like euclidean spaces, it seems reasonable to expect that we can lift the definitions directly from $\R^n$.
If we are given a continuous map between two topological manifolds, we can locally view it as a continuous map between two Euclidean spaces.
Generalizing this further, we could conceivably say that our original map is differentiable if the local map is.

\newthought{As usual, the devil is in the details}: a topological manifold is only homeomorphic to a Euclidean space, and a different choice of homeomorphism might affect whether the local map is differentiable or not.
We need to have extra care to ensure that these lifted definitions keep making sense when we use different charts that overlap.

The solution is to introduce a little more structure to the problem.

\begin{defn}\label{def:crcomp}
We say that two charts $(U_1, \phi_1)$ and $(U_2, \phi_2)$ on a topological manifold $M$ are \emph{compatible} if either $U_1 \cap U_2 = \emptyset$ or if the \emph{transition map}\footnote{Both the composition maps $\phi_1 \circ \phi_2^{-1}$ and $\phi_2 \circ \phi_1^{-1}$ are called transition maps. Both maps are necessarily homeomorphisms since $\phi_1$ and $\phi_2$ are.}
\begin{equation}
  \phi_1 \circ \phi_2^{-1} : \phi_2(U_1\cap U_2) \to \phi_1(U_1 \cap U_2)
\end{equation}
is a smooth diffeomorphism.
\end{defn}

\begin{figure*}[htp]
  \centering
  \includegraphics{1_2-compatible-charts.pdf}
  \caption{Charts are compatible if they coincide on the intersections of their coordinate neighborhoods.}
  \label{fig:1.2-compatible-charts}
\end{figure*}

With these at hand, let's jump into the definition of smooth manifolds.

\begin{defn}\label{def:cratlas}
  A \emph{smooth atlas} is a collection
  \begin{equation}
    \cA = \{\phi_\alpha: U_\alpha \to V_\alpha \;\mid\; \alpha\in A\}
  \end{equation}
  of pairwise compatible charts that cover\footnote{I.e. such that $M = \cup_{\alpha\in A} U_\alpha$. One calls the set $\{U_\alpha \;\mid\; \alpha\in A\}$, covering $M$ with open sets, a \emph{open cover} of $M$. Here $A$ is some index set, not necessarily countable.} $M$.

  Two smooth atlases are \emph{equivalent} if their union is also a smooth atlas. That is if any two charts in the atlases are compatible.
\end{defn}

\begin{exe}
  Show that the equivalence of atlases is really an equivalence relation.
\end{exe}

\begin{defn}\label{def:diffstr}
  A \emph{differentiable structure}, or more precisely a \emph{smooth structure}, on a topological manifold is an equivalence class of smooth atlases.
  \marginnote{The union of all atlases in a differentiable structure is the \emph{unique} \emph{maximal} atlas in the equivalence class. There is a one-to-one correspondence between differentiable structures and maximal differentiable atlases: for convenience and to lighten the notation, from now on, we will always regard a differentiable structure as a differentiable maximal atlas without further comments.}
\end{defn}

\begin{ntn}
By a \emph{chart $(U, \phi)$ about $p$} in a manifold $M$ we mean a chart in the differentiable structure of $M$ such that $p\in U$.
\end{ntn}

\begin{defn}\label{def:diffmanifold}
  A \emph{smooth manifold} of dimension $n$ is a pair $(M, \cA)$ of a topological $n$-manifold $M$ and a smooth structure $\cA$ on $M$.
  \marginnote[1em]{There are no preferred coordinate charts on a manifold: all coordinate systems compatible with the differentiable structure are on equal footing.}
\end{defn}

In colloquial language, a differentiable manifold is a space covered by charts with differentiable transition maps.

Whenever possible we will omit the differentiable structure $\cA$ from the notation and just write $M$.
We may write $M^n$ when we want to emphasize the dimension $n$ of $M$.

\begin{exe}
  Show that on a second countable differentiable manifold it is always possible to find a countable atlas.
\end{exe}

\begin{exe}\label{exe:subsetsmanifolds}
  $\R^n$ with the \emph{standard} smooth structure $\cA=(\R^n, \id_{\R^n})$ is trivially a smooth manifold of dimension $n$.

  In fact, any open subset $U\subset\R^n$ can be made into a smooth manifold in a natural way: pick the atlas $\cA=(U, \id_{\R^n}|_U)$.

  In the same way, any open subset $U$ of a smooth manifold $M$ is a smooth manifold.
  Which atlas would you choose?

  More generally, if $V$ is any $n$-dimensional real vector space, then the standard smooth structure on $V$ is the one induced by the smooth atlas consisting of a single chart $(V, T)$ where $T: V \to R^n$ is some linear isomorphism.
  Why is this independent of the choice of the isomorphism $T$?

  This example has a very interesting consequence.
  The space $\mathrm{Mat}(n)$ of $n\times n$-matrices can be identified with $\R^{n^2}$ by writing the elements of the matrix as a $n^2$-vector.
  This gives to $\mathrm{Mat}(n)$ a structure of differentiable manifold.
  The subset of invertible matrices $GL(n) = \{ A \in \mathrm{Mat}(n) \;\mid\; \det A \neq 0\}$, widely known as the \emph{general linear group}, being an open subset of $\mathrm{Mat}(n)$ (why?) is itself a differentiable manifold.
Can you guess if such manifold is connected or not?
\end{exe}

\begin{ntn}\label{ntn:coords}
  We will stick to the notation of \cite{book:tu}.
  In the context of manifolds, denote $r^i: \R^n\to\R$, $1\leq i\leq n$, the standard coordinates on $\R^n$. With this notation, if $e_i$ denotes the $i$th standard basis vector in $\R^n$, then $r^i(e_j) = \delta^i_j$.
  \marginnote[-1em]{The Kronecker delta $\delta_j^i$ is defined by $\delta_j^i = 1$ if $i=j$ and $\delta_j^i = 0$ otherwise.}

  If $(U, \phi:U\to\R^n)$ is a chart of a manifolds, then $x^i = r^i\circ\phi$ will denote the $i$-th component of $\phi$ and denote $\phi = (x^1, \ldots, x^n)$ and, when convenient, $(U,\phi) = (U, x^1, \ldots, x^n)$.

  Thus, for $p\in U$, $(x^1(p), \ldots, x^n(p))$ is a point\footnote{By abuse of notation we sometimes omit the $p$! Thus $(x^1, \ldots, x^n)$ can stand either for local coordinates or a point in $\R^n$: which one it is should be clear from the context.} in $\R^n$. The functions $x^1, \ldots, x^n$  are called \emph{(local) coordinates} on $U$.
\end{ntn}

An advantage of this new notation is that we can talk about coordinates without the need to explicitly reference charts. In other words, we can say
\begin{quote}
  Let $p\in M$ and choose local coordinates $(x^1, \ldots, x^n)$ about $p$...
\end{quote}
or even
\begin{quote}
  Let $x=(x^1, \ldots, x^n)\in M$ be a point in $M$...
\end{quote}
dropping the distinction between $p$ and $x$, both in place of
\begin{quote}
  Let $p \in M$ and $(U, \phi)$ a chart defined on a neighborhood $U$ of $p$.
  Let $x^i = r^i \circ\phi$ denote the components of $\phi$ with respect to the standard euclidean coordinates...
\end{quote}

\begin{ex}\label{ex:S1emb}
  The unit circle
  \begin{equation}
    \bS^1 := \{x\in\R^2 \;\mid\; \|x\|=1\}\subset\R^2
  \end{equation}
  with the relative topology\footnote{Let $(X,\cT)$ be a topological space and $Y\subset X$. The relative topology on $Y$ is \begin{equation}\mathcal{V}:=\{V\subset Y\;\mid\;\exists U\in\cT \mbox{ s.t. } V = U \cap Y\}.\end{equation}} is a $1$-dimensional topological manifold.
  To provide the local homeomorphisms to $\R^n$ and define a smooth structure four $\bS^1$ it is enough to define the following four charts:
  \begin{equation}
    \begin{aligned}
      &V_1 := \{ x^1 > 0 \},\quad \phi_1: V_1 \to (-1, 1), \quad \phi_1(x) := x^2,\\
      &V_2 := \{ x^1 < 0 \},\quad \phi_2: V_2 \to (-1, 1), \quad \phi_2(x) := x^2,\\
      &V_3 := \{ x^2 > 0 \},\quad \phi_3: V_3 \to (-1, 1), \quad \phi_3(x) := x^1,\\
      &V_4 := \{ x^2 < 0 \},\quad \phi_4: V_4 \to (-1, 1), \quad \phi_4(x) := x^1.
    \end{aligned}
  \end{equation}
  What do these charts look like?
\end{ex}

\begin{exe}
  Let $f: \R^n \to \R^m$ be a smooth map.
  Show that its graph
  \begin{equation}
    \Gamma_f := \{(x, f(x)) \;\mid\; x\in\R^n\} \subset\R^{n+k}
  \end{equation}
  is a smooth manifold of dimension $n$.
\end{exe}

\begin{ex}
  The definition of smooth manifold does not require $M$ to be embedded into some ambient space as in the examples above.
  In fact, we can define the differentiable manifold $\bS^1$ by equipping the topological quotient space $\R/\Z$ with the two charts
  \begin{equation}\textstyle
    \phi_1 : \R/\Z \setminus\{[0]\} \to (0,1)
    \quad\mbox{and}\quad
    \phi_2 : \R/\Z \setminus\{[\frac12]\} \to (-\frac12,\frac12)
  \end{equation}
  which map $[x]\in\R/\Z$ to its representation in $(0,1)$ or $[-\frac12, \frac12)$ respectively.
  The manifold obtained in this way is diffeomorphic to the one defined in Example \ref{ex:S1emb}.
\end{ex}

\begin{ex}[Product manifolds]
Given two manifolds $(M_1, \cA_1)$ and $(M_2, \cA_2)$, we can define the \emph{product manifold} $M_1 \times M_2$ by equipping $M_1 \times M_2$ with the product topology\footnote{Open sets in the product are products of open sets from the respective topological spaces.} and cover the space with the atlas $\{ (U_1\times U_2, (\phi_1, \phi_2)) \;\mid\; (U_1, \phi_1)\in\cA_1, (U_2, \phi_2)\in \cA_2\}$.
\end{ex}

Note that smooth manifolds do not yet have a metric structure: distances between the points are not defined.
However, they are \emph{metrizable}\footnote{This property is far more general: all the topological manifolds are metrizable.}: there exists some metric on the manifold that induces the given topology on it.
This allows to always view manifolds as metric spaces.

\begin{rmk}
	There exist examples of topological manifolds without smooth structures.
	It is also known that smooth manifolds of dimension $n < 4$ have exactly one smooth structure while ones of dimension $n > 4$ have finitely many\footnote{A beautiful example of this is the $7$-sphere $\bS^7$ which is known to have 28 non-diffeomorphic smooth structures.}.
	The case $n = 4$ is unknown: if you prove that there is only one smooth structure, you will have shown the smooth Poincar\'e conjecture.
\end{rmk}

Instead of always constructing a topological manifold and then specify a smooth structure, it is often convenient to combine these steps into a single construction.
This is especially useful when the initial set is not equipped with a topology.
In this respect, the following lemma provides a welcome shortcut: in brief it says that given a set with suitable ``charts'' that overlap smoothly, we can use the charts to define both a topology and a smooth structure on the set.

\begin{lem}[Smooth manifold lemma]\label{lem:manifold_chart}
  Let $M$ be a set. Assume that we are given a collection $\{U_\alpha\mid \alpha\in A\}$ of subsets of $M$ together with bijections $\phi_\alpha: U_\alpha\to\phi(U_\alpha)\subseteq\R^n$, where $\phi(U_\alpha)$ is an open subset of $\R^n$. Assume in addition that the following hold:
  \begin{enumerate}
    \item For each $\alpha, \beta \in A$, the sets $\phi_\alpha(U_\alpha \cap U_\beta)$ and $\phi_\beta(U_\alpha \cap U_\beta)$ are open in $\R^n$.
    \item If $U_\alpha \cap U_\beta \neq 0$, the map $\phi_\beta\circ\phi_\alpha^{-1}: \phi_\alpha(U_\alpha \cap U_\beta)\to \phi_\beta(U_\alpha \cap U_\beta)$ is smooth.
    \item Countably many of the sets $U_\alpha$ cover $M$.
    \item If $p\neq q$ are points in $M$, either there exists $\alpha$ such that $p,q\in U_\alpha$ or there exists $\alpha,\beta$ with $U_\alpha\cap U_\beta=\emptyset$ such that $p\in U_\alpha$ and $q\in U_\beta$.
  \end{enumerate}
  Then $M$ has a unique smooth manifold structure such that each $(U_\alpha,\phi_\alpha)$ is a smooth chart.
\end{lem}
\begin{exe}
  Prove Lemma~\ref{lem:manifold_chart}. Hint: declare all the $\phi_\alpha$ to be homeomorphisms and use the hypotheses to check the definition of a smooth manifold.
\end{exe}

\begin{ex}[Grassmannian (Manifold)]
  \todo{Lee, Example 1.36}
\end{ex}

\begin{tcolorbox}
  The general definition of $C^r$-manifolds is mostly a matter of replacing occurrences of ``smooth'' in the text with $C^r$.
  The study of these more general structures is not dissimilar from what we will see in this course, with the exception of analytic and $C^0$-manifolds, but it introduces an unnecessary extra level of verbosity.
  In these notes we will only deal with smooth manifolds.
\end{tcolorbox}

\section{Intermission: exercises}
\TODO

\section{Smooth maps and differentiability}

With a well defined differentiable structure and the idea of compatible charts, we have all the ingredients to lift the definition of differentiable maps from the euclidean world.

Before considering the general definition of a differentiable map, let's look at the simpler example of differentiable functions $f:M\to\R$ between a smooth manifold $M$ and $\R$.

\begin{defn}
  A function $f:M\to\R$ from a smooth manifold $M$ of dimension $n$ to $\R$ is \emph{smooth}, or \emph{of class $C^\infty$}, if for any chart $(\phi, V)$ of $M$ the map $f\circ\phi^{-1}:\phi(V)\subset\R^n \to \R$ is smooth as a euclidean function.
  \begin{marginfigure}
    \includegraphics{1_5-diff-fun-v2.pdf}
    \label{fig:diff-fun}
    \caption{A function is differentiable if it is differentiable as a euclidean function through the magnifying lens provided by the charts.}
  \end{marginfigure}
  We denote the space of smooth functions by $C^\infty(M)$.
\end{defn}

This, colloquially speaking, means that a function is differentiable if it is differentiable as a euclidean function through the magnifying lens (see Figure~\ref{fig:diff-fun}) provided by the charts.

\begin{exe}
  Define on the following operations.
  For any $f,g\in C^\infty(M)$, $c\in\R$,
  \begin{equation}
    (f+g)(x) := f(x) + g(x),\quad
    (fg)(x) := f(x) g(x),\quad
    (cf)(x) := c f(x).
  \end{equation}
  Then, the space $C^\infty(M)$ endowed with the operations above is an \emph{algebra}\footnote{I.e. a vector space where you can also multiply two elements.} over $\R$.
\end{exe}

Which immediately gives away the general definition.

\begin{defn}
  Let $F:M_1\to M_2$ be a continuous map \footnote{Remember: continuity is not a problem since $M_1$ and $M_2$ are topological spaces.} between two smooth manifolds of dimension $n_1$ and $n_2$ respectively.
  We say that $f$ is \emph{smooth}, or \emph{of class $C^\infty$}, if, for any chart $(\phi_1, V_1)$ of $M_1$ and $(\phi_2, V_2)$ of $M_2$, the map
  \begin{align}
    \phi_2 \circ F \circ \phi_1^{-1}: U_1 \to U_2,\\
    U_1 := \phi_1(V_1 \cap f^{-1}(V_2))\subset\R^{n_1},\\
    U_2 := \phi_2(f(V_1) \cap V_2)\subset\R^{n_2},
  \end{align}
  is smooth as a euclidean function.
  \marginnote[-6em]{Differently from your calculus classes, we are defining differentiability \emph{before} we define what the derivative is. Getting to it will require some amount of work, and will have to wait until the next chapter.}

  We denote by $C^\infty(M_1, M_2)$ the set of all functions $F:M_1\to M_2$ of class $C^\infty$.

  The map $\hat F := \phi_2 \circ F \circ \phi_1^{-1}$ is called the \emph{coordinate representation of $F$} with respect to the given coordinates.
\end{defn}

\begin{figure}[htp]
  \centering
  \includegraphics{1_3-diffble_maps.pdf}
  \caption{Maps are differentiable when they are differentiable as maps between euclidean spaces.}
  \label{fig:1.3-differentiable_maps}
\end{figure}

A first observation about our definition of smooth maps is that as one would hope, smooth imply continuity.

\begin{exe}
  Shoa that every smooth map is continuous.
\end{exe}

\begin{defn}
A \emph{diffeomorphism} $F$ between two smooth manifolds $M_1$ and $M_2$ is a bijective map such that $F\in C^\infty(M_1, M_2)$ and $F^{-1}\in C^\infty(M_2, M_1)$.

Two smooth manifolds $M_1$ and $M_2$ are called \emph{diffeomporphic} if there exists a diffeomorphism $F:M_1\to M_2$ between them.
\end{defn}

\begin{ex}
Any chart $(V, \phi)$ of a manifold $M$ is a diffeomorphism between the manifolds $V\subset M$ and $\phi(V)\subset\R^n$.
\end{ex}

\begin{exe}
  Prove the following propositions and aid your reasoning by drawing the relevant figures.
  \begin{prop}
    Let $M$ be a smooth manifold of dimension $n$.
    Then $F:M\to\R^m$ is smooth iff for all charts $(U,\phi)$ of $M$, the function $F\circ\phi^{-1}:\phi(U)\to\R^m$ is smooth.
  \end{prop}
  \begin{prop}
    Let $M$ be a smooth manifold of dimension $n$.
    Then $f:\R^m\to M$ is smooth iff for all charts $(U,\phi)$ of $M$, the function $\phi\circ F:F^{-1}(U)\to\R^m$ is smooth.
  \end{prop}
  \begin{prop}
    Let $M, N, P$ be three smooth manifolds, and suppose that $F:M\to N$ and $G:N\to P$ are smooth.
    Then $G\circ F\in C^\infty(M, P)$.
  \end{prop}
  \marginnote{In Proposition~\ref{prop:uniqdiffeoinclusion} it is not enough to ask that $\iota$ is smooth! As counterexample consider the two manifolds $(\R, \cA_1)$ with $\cA_1 := \{(\R, \id_\R)\}$ and $(\R, \cA_2)$ with $\cA_2 := \{(\R, x\mapsto x^3)\}$. The inclusion of open sets in $\R$ is smooth in both cases but is a diffeomorphism only in one.}
  \begin{prop}\label{prop:uniqdiffeoinclusion}
    Let $M$ be a manifold and $U\subset M$ an open set.
    Then $U$ has a unique differentiable structure such that the inclusion $\iota:U\hookrightarrow M$ is a diffeomorphism.
  \end{prop}
  \begin{prop}[Smoothness is a local property]\label{prop:smoothlocal}
    Let $F:M\to N$ be a continuous function and let $\{U_i\}_{i\in I}$ be an open cover for $M$. Then $F|_{U_i}:U_i \to N$ is smooth for every $i\in I$ iff $F:M\to N$ is smooth.
  \end{prop}
\end{exe}

The following corollary is just a restatement of Proposition~\ref{prop:smoothlocal}, but provides a useful perspective on the construction of smooth maps.

\begin{prop}[Gluing lemma for smooth maps]
  Let $M$ and $N$ be two smooth manifolds and let $\{U_\alpha\mid\alpha\in A\}$ be an open cover of $M$.
  Suppose that for each $\alpha\in A$ we are given a smooth map $F_
  \alpha:U_\alpha\to N$ such that the maps agree on the overlaps: $F_\alpha|_{U_\alpha\cap U_\beta} = F_\beta|_{U_\alpha\cap U_\beta}$ for all $\alpha,\beta\in A$. 
  Then there exists a unique smooth map $F:M\to N$ such that $F|_{U_\alpha} = F_\alpha$ for each $\alpha\in A$.
\end{prop}

\begin{tcolorbox}
From now on, when we write manifold, chart, atlas, etc. we always mean smooth manifold, smooth chart, smooth atlas, etc..
\end{tcolorbox}

\section{Partitions of unity}

\newthought{Cutoff functions} are a class of smooth functions that will be of crucial importance throughout the course, and whose existence cannot be given for granted.
Since their definition and construction does not require more than what we have just seen, let's talk about them now.

First of all, recall that the \emph{support} of a smooth function $f: M \to \R$, denoted by $\supp(f)$, is defined as
\marginnote[2em]{The bar over the set denotes its closure.}
\begin{equation}
  \supp(f) := \overline{\{ p\in M \;\mid\; f(p) \neq 0\}}.
\end{equation}

We will introduce those functions with a proposition, and will spend the rest of this chapter proving it.

\begin{figure}[htp!]
  \includegraphics{1_6-cutoffs.pdf}
\end{figure}

\begin{prop}[Cutoff functions]\label{prop:cutoff}
  Let $M$ be a smooth manifold and $K\subset U\subset M$ two subsets such that $K$ is closed and $U$ is open.
  Then, there exists a smooth function $\chi: M \to\R$, called \emph{cutoff} function, with the following properties
  \begin{enumerate}
    \item $0 \leq \chi \leq 1$ for all $p\in M$;
    \item $\supp(\chi)\subset U$;
    \item $\chi(p) = 1$ for all $p\in K$.
  \end{enumerate}
\end{prop}

The proof of this proposition involves a general result which is quite technical and whose proof will be omitted.
You can refer to \cite{book:lee, book:tu} if you are curious to see the details.

Instead, we will show a special case of Proposition~\ref{prop:cutoff}. The main reason is that it involves an explicit construction of the cutoff which can be convenient to have at hand later on.

\begin{lem}[Cutoff functions, compact case]
  Let $M$ be a smooth manifold and $K\subset U\subset M$ two subsets such that $K$ is compact and $U$ is open.
  Then, there exists a smooth function $\chi: M \to\R$ with the following properties
  \begin{enumerate}
    \item $0 \leq \chi \leq 1$ for all $p\in M$;
    \item $\supp(\chi)\subset U$;
    \item $\chi(p) = 1$ for all $p\in K$.
  \end{enumerate}
\end{lem}
\begin{proof}
  \newthought{Part 1}.
  To warm up, let's do some first year analysis.
  For any pair of real numbers $r < R$ there exists a smooth function $f: \R \to [0,1]$ such that $f(t) = 1$ for $t \leq r$, $f(t) = 0$ for $t \geq R$ and $0<f(t)<1$ for $t\in(r,R)$.
  
  We can construct this explicitly by means of the function
  \begin{equation}
    h:\R\to\R, \quad h(t):= \begin{cases}
      e^{-1/t}, & t>0,\\
      0, & t \leq 0.
    \end{cases}
  \end{equation}

  \begin{exe}
    Prove by induction that for $t>0$ and $k\geq 0$, the $k$th derivative $f^{(k)}(t)$ is of the form $p_{2k}(1/t)e^{-1/t}$ for some polynomial $p_{2k}(x)$ of degree $2k$ in $x$.
    Use this to show that $f\in C^\infty(\R)$ and that $f^{(k)}(0) = 0$ for all $k\geq 0$.
  \end{exe}

  The function $f$ that we are seeking is then\footnote{Exercise: check that such function $f$ satisfies all the desired properties.} given by
  \begin{equation}
    f(t) := \frac{h(R-t)}{h(R-t) + h(t-r)}.
  \end{equation}

  \newthought{Part 2}.
  Let's extend $f$ to $\R^n$.
  Denote $B_r \subset \R^n$ the open ball of radius $r$ around the origin.
  Then, for any $0 < r < R$ we seek a function $g:\R^n\to\R$ such that $g(x) = 1$ for all $x\in \overline{B_r}$, $g(x) = 0$ for all $x\in \R^n\setminus B_R$ and $0< g(x)< 1$ for all $x\in B_R\setminus\overline{B_r}$.
  This is immediately achieved by defining $g(x) := f(\|x\|)$, where $f$ is the function defined in the previous step.

  \newthought{Part 3}.
  Let's now pick a point $p\in M$ and an arbitrary neighborhood $U$ of $p$. Choosing an appropriate chart about $p$, the previous step implies that we can choose a smaller neighborhood $V\subset U$ of $p$ with $\overline V\subset U$ and such that there exists a smooth function $\chi: M \to [0,1]$ satisfying $\chi(p) = 1$ for all $p\in\overline{V}$ and $\chi(p) = 0$ for all $p\in M\setminus U$.
  
  \newthought{Part 4}.
  We are ready to complete the proof.
  For each point $p\in K$, choose two neighborhoods $V_p \subset U_p$ such that $\overline{V_p}\subset K$ and $U_p \subset U$.
  Since $K$ is compact, it admits a finite cover in terms of these sets: i.e. there are finitely many points $p_1, \ldots, p_N \in K$ such that $K \subset \bigcup_{i=1}^N V_{p_i}$.
  For each $i$, choose $\chi_i: M \to [0,1]$ as in the previous step: $\chi_i(p) = 1$ for all $p\in\overline{V_{p_i}}$ and $\chi_i(p) = 0$ for all $p\in M\setminus U_{p_i}$.
  The proof is completed by defining
  \begin{equation}
    \chi := 1 - \prod_{i=1}^N(1 - \chi_i(p)).
  \end{equation}
\end{proof}

To extend this result and prove will still require hard work and a new tool, that will be useful throughout the course an in many courses to come.

\begin{defn}
  Let $M$ be a smooth manifold. A \emph{partition of unity} is a collection $\{\rho_\alpha \mid \alpha\in A\}$ of functions $\rho_\alpha:M\to\R$ such that
  \begin{enumerate}
    \item $0 \leq \rho_\alpha \leq 1$ for all $p\in M$ and $\alpha\in A$;
    \item\label{def:pou.2} the collection $\{\rho_\alpha \mid \alpha\in A\}$ is \emph{locally finite}, that is, for any $p\in M$ there are at most finitely many $\alpha\in A$ such that $p\in\supp(\rho_\alpha)$;
    \item for all $p\in M$ one has $\sum_{\alpha\in A} \rho_\alpha(p) = 1$.
    \marginnote{For any $p$, $\sum_{\alpha\in A} \rho_\alpha(p)$ is a finite sum by \ref{def:pou.2}. Thus, the function defined by the sum $\rho := \sum \rho_\alpha$ is a well define smooth function on $M$. We call such sum a \emph{locally finite} sum.}
  \end{enumerate}
\end{defn}

\begin{rmk}
Note that the existence of a partition of unity is a distinguished feature of differentiable manifolds: stronger structures, like analytic or holomorphic ones, in general fail to have one.
\end{rmk}

Throughout the course we will be mostly interested in partitions of unity $\{\rho_\alpha \mid \alpha\in A\}$ which are \emph{subordinate} to an open cover $\{U_\alpha\mid\alpha\in A\}$, that is, such that $\supp_\alpha(\rho_\alpha) \subset U_\alpha$ for each $\alpha\in A$.

\begin{thm}\label{thm:partitionof1}
  Let $M$ be a smooth manifold. For any open cover $\{U_\alpha\mid\alpha\in A\}$ of $M$, there exists a partition of unity $\{\rho_\alpha \mid \alpha\in A\}$ subordinate to $\{U_\alpha\mid\alpha\in A\}$.
\end{thm}

With this result at hand, Proposition~\ref{prop:cutoff} can be shown very easily.

\begin{proof}[Proof of Proposition~\ref{prop:cutoff}.]
  Consider the open cover of $M$ given by $\cC:=\{M\setminus K, U\}$.
  Then Theorem~\ref{thm:partitionof1} implies that there exists a partition of unity $\{\rho_U, \rho_{M\setminus K}\}$ adapted to $\cC$. The function $\chi := \rho_U$ is our cutoff function.
\end{proof}

\section{Manifolds with boundary}\label{sec:mbnd}

\newthought{The definition of manifolds has a serious limitation}, even though it is perfectly good to describe curves\footnote{E.g. the circle seen in Example~\ref{ex:S1emb}.} and surfaces\footnote{E.g. the $n$-spheres $\bS^n$ in the homework sheet.}, it fails to describe many natural objects like a \emph{closed} interval $[a,b]\in\R$ or the \emph{closed} disk $D_1(0)$ of Example~\ref{ex:uball}.

Note that in both these cases, both the interior and the boundary are smooth manifolds and their dimension differ by one\footnote{In the first case the interior $(a,b)$ is a $1$-manifold and the boundary, the set $\partial[a,b] = \{a,b\}$, is a $0$-manifold. In the second case the interior of $D_1(0)$ is the open unit ball, a $2$-manifold, and the boundary $\partial D_1(0)$ is the $1$-manifold $\bS^1$.}.

Let's do a step back and think about topological manifolds: since both the closed interval and the closed disk are closed sets, we have problems to make them locally euclidean in neighborhoods of their boundaries.
Can we modify our local model to resemble something with a boundary?

Of course this is a rhetorical question.
We can generalize our definition by considering the \emph{closed upper half-spaces}
\begin{marginfigure}[2em]
  \includegraphics{1_4-upper_space.pdf}
\end{marginfigure}
\begin{equation}
  \cH^n = \R_+^n = \{x=(x^1, \ldots, x^n)\in\R^n\mid x^n \geq 0\},
\end{equation}
with its $(n-1)$-dimensional boundary
\begin{equation}
  \partial\cH^n = \{x=(x^1, \ldots, x^n)\in\R^n\mid x^n = 0\}
\end{equation}
and the topology inherited by $\R^n$, as a replacement for our local model $\R^n$.

\begin{defn}
  A topological space $M$ is a \emph{topological manifold with boundary} of dimension $n$, or topological $n$-manifold with boundary, if it has the following properties
  \begin{enumerate}[(i)]
    \item $M$ is a Hausdorff space;
    \item $M$ is second countable;
    \item $M$ is \emph{locally} homeomorphic to $\cH^n$, any point $x\in M$ has a neighborhood that is homeomorphic to a (relatively) open\footnote{Recall that $U\subset\R^n_+$ is relatively open, that is open with respect to the relative topology, if there exist an open set $\tilde U\subset\R^n$ such that $U = \tilde U \cap \R^n_+$.} subset of $\R^n_+$.
  \end{enumerate}

  A \emph{chart} on $M$ is a pair $(U, \phi)$ consisting of an open set $U\subset M$ and a homeomorphism $\phi: U \to \phi(U)\subset \cH^n$.
\end{defn}

\begin{figure}
  \includegraphics{1_4-mfld-w-bdry.pdf}
\end{figure}

We saw in Proposition~\ref{prop:uniqdiffeoinclusion} that differentiability is a local property, which means that is a property defined on open sets.
To clarify what it means to have differentiable structures on manifolds with boundary, we will thus need to clarify what it means for a function defined on $\cH^n$ to be differentiable at points on $\partial\cH^n$.
As it turns out, this is a minor modification of our previous definition that stems directly from the definition of the induced topology.

\begin{defn}
  Let $U\subset\cH^n$ be a relatively open set. A map $f: U\to\R^m$ is \emph{$r$-times continuously differentiable}, or of class $C^r$, if there exists an open set $\tilde U\in\R^n$ and a map $\tilde f\in C^r(\tilde U, \R^m)$ such that $U\subset\tilde U$ and $\tilde f|_U = f$.
  The function $f$ is said to be \emph{smooth}, or of class $C^\infty$, if $f$ is $r$-times continuously differentiable for all $r\geq 1$.
\end{defn}

With such definition at hand, one can define compatibility, smooth atlases and differentiable structures as in Definition~\ref{def:crcomp}, Definition~\ref{def:cratlas} and Definition~\ref{def:diffstr} by considering charts taking value in $\cH^n$.

\begin{defn}\label{def:diffmanifoldwb}
  A \emph{smooth manifold with boundary} of dimension $n$ is a pair $(M, \cA)$ of a topological $n$-manifold with boundary $M$ and a smooth differentiable structure $\cA = \{(U_\alpha, \phi_\alpha) \mid \alpha\in A\}$ on $M$.
  
  \marginnote[-2em]{The boundary $\partial M$ as defined by \eqref{def:bdry} can differ from its topological boundary as a subset of another topological space. For example the boundary $\partial\bS^1$ of the circle as a manifold is empty, but the boundary of the circle $\bS^1$ as a subset of $\R^2$ is $\bS^1$ itself.}
  The \emph{boundary} of $M$ is defined as
  \begin{equation}\label{def:bdry}
    \partial M := \bigcup_{\alpha\in A} \phi_\alpha^{-1}\left(\phi_\alpha(V_\alpha)\cap \partial\cH^n\right).
  \end{equation}
\end{defn}

\begin{prop}\label{prop:bdwelldef}
  The boundary $\partial M$ is well defined\footnote{The same definition holds for topological manifolds, but showing that it is well defined is much more complicated and will be omitted.}.
\end{prop}
\begin{proof}
  The statement follows if we show that the transition maps send boundary pieces to boundary pieces.
  It turns out that this fact is more general: for any diffeomorphism $f:U \to V$, where $U,V \subset\cH^n$ are relatively open, it holds that $x\in U\cap\partial\cH^n$ if and only if $f(x)\in V\cap\partial\cH^n$.

  Indeed, let $x\in U\cap(\cH^n\setminus\partial\cH^n)$ be a point in the interior of $U$. Expanding $f$ in Taylor series up to the first order, we have
  \begin{equation}
    f(x+h) = f(x) + Df|_x h + O(\|h\|).
  \end{equation}
  Since the differential $D f$ at $x$ is an isomorphism, there exist an open neighborhood $\cO$ of $x$ such that $f(\cO)$ is open in $\R^n$ and thus $f(x)\in V\cap(\cH^n\cap\partial\cH^n)$.
\end{proof}

\begin{ex}
  Let's go back to the closed interval $M=[a,b]\subset\R$. 
  With the atlas
  \begin{equation}
    \cA=\big\{
      \big([a,b), \; x\mapsto x-a\big),
      \big((a,b], \; x\mapsto b-x\big)
    \big\}
  \end{equation}
  it is a differentiable $1$-manifold with boundary $\partial M = \{a\} \cup \{b\} = \{a, b\}$.
\end{ex}

Let's go back to our observation at the beginning of this section.
We started by observing that some objects seemed to be the ``sum'' of a boundary manifold and an interior manifold.
Can we make sense of such observation using our newly introduced definition?

\begin{prop}
  Let $M$ be a differentiable $n$-manifold with boundary.
  Then $M\setminus\partial M$ and $\partial M$ inherit the structure of manifolds (without boundary) of dimensions $\dim(M\setminus\partial M)=n$ and $\dim(\partial M) = n-1$.
\end{prop}
\begin{proof}
  Let $\cA = \{(U_\alpha,\phi_\alpha) \mid \alpha\in A\}$ be an atlas for $M$. 
  Then
  \begin{equation}
    \cA_\circ := \left\{
      \left(
        U_\alpha \cap (M\setminus\partial M),
        \phi_\alpha|_{U_\alpha \cap (M\setminus\partial M)}
        \right) \mid \alpha\in A
      \right\}
  \end{equation}
  is an atlas for $M\setminus\partial M$ where none of the charts contain points in $\partial\cH^n$.

  In a similar vein, an atlas for $\partial M$ is given by
  \begin{equation}
    \cA_\partial := \left\{
      \left(
        U_\alpha \cap \partial M,
        \phi_\alpha|_{U_\alpha \cap \partial M}
        \right) \mid \alpha\in A
      \right\},
  \end{equation}
  where
  \begin{equation}
    \phi_\alpha|_{U_\alpha \cap \partial M} : (U_\alpha \cap \partial M) \to \partial\cH^n\simeq\R^{n-1}
  \end{equation}
  by the proof of Proposition~\ref{prop:bdwelldef}.
\end{proof}

\begin{tcolorbox}
  Differentiable manifolds without boundary (cfr. Definition~\ref{def:diffmanifold}) can be thought as a special case of differentiable manifolds with boundary (cfr. Definition~\ref{def:diffmanifoldwb}) where the boundary happens to be empty.
  Therefore, with the exception of the beginning of Chapter~\ref{ch:2}, we will no-longer distinguish the two concepts: from now on, a manifold may have or may not have a boundary.
\end{tcolorbox}

\chapter{Tangent bundle}\label{ch:2}

\newthought{Let's the fun begin!}
We are left to define what derivatives of functions between manifolds are.
And, since we saw that euclidean spaces are manifolds, we better find a definition that coincides with the one you saw in your analysis courses.

\begin{marginfigure}[7em]
  \includegraphics{2_1-embedded-sphere-tangent.pdf}
  \label{fig:tan-embedded-sphere}
  \caption{Tangent space to a point of a sphere $\bS^2$ embedded into the ambient space $\R^3$.}
\end{marginfigure}
In this chapter we will see how to associate to an $n$-dimensional smooth manifold $M$ an $n$-dimensional vector space, denoted by $T_x M$, to each point $x\in M$.
Such vector space is called \emph{tangent space to $M$ at $x$} and, for a manifold embedded into a euclidean ambient space, it will coincide with the intuitive understanding of a tangent hyperplane to the point on the manifold, see also Figure~\ref{fig:tan-embedded-sphere}.
As we will see, there are various different definition of tangent space but, in the end, they all turn out to be equivalent.

Due to the amount of freedom and the many equivalent definitions, there is no unique way of introducing tangent spaces.
Just to give you an idea, all the following approaches lead to equivalent definitions (see also \cite{book:lee}):
\begin{enumerate}
  \item equivalence classes of curves through a point;
  \item transformation laws of the components of vectors with respect to different charts;
  \item generalization of linear approximation into the idea of an abstract derivation;
  \item derivations in the category of germs of functions;
\end{enumerate}

It is also possible to ``flip'' the whole construction around, constructing differentials and cotangent spaces and using them to introduce the tangent spaces.
This is the approach taken by \cite{lectures:hitchin} and it is at least worth of a look if you want to see a different perspective.

To avoid diverging from the book too much, we will stick to derivations on the space of germs, which emphasizes the locality of derivations to an extreme.

The equivalence between such approach and the one based on speeds of curves and on derivations will be left as homeworks.

% \subsection{Partial derivatives}

% \newthought{Being locally euclidean seems to be very powerful}.
% Even though we are not yet in the position to define total derivatives of functions, we can already use the euclidean analogy in order to define partial derivatives.

% For simplicity, consider a smooth manifold $M$ of dimension $n$ and a function $f\in\cC^\infty(M,\R^n)$.

% Let $(U, \phi$) be a chart of $M$. 
% We already said that as a function on $\R^n$, $\phi$ has $n$ components $x^1, \ldots, x^n$ defined in terms of the standard euclidean coordinates $r^1, \ldots, r^n$ as $\phi^i = r^i \circ \phi$, $1\leq i\leq n$.
% For $p\in U$, the \emph{partial derivate $\frac{\partial f}{\partal x^i}$ of $f$ with respect to $x^i$ at $p$} can be defined as
% \begin{equation}
%   \frac{\partial}{\partial x^i}\big|_p f := 
%   \frac{\partial f}{\partial x^i} (p) :=
%   \frac{\partial f \circ \phi^{-1}}{\partial r^i} (\phi(p)) :=
% \end{equation}

\subsection{Directional derivatives}

Suppose that $f: U\subset\R^n\to\R^k$ is a smooth map defined on an open subset $U\subset \R^n$.
In multivariable calculus you have seen that if $x\in U$ and $v\in\R^n$, then the vector $Df(x) v$ can be interpreted as the directional derivative\footnote{Sometimes this is denoted $D_v f(x)$ instead.} of $f$:
\begin{equation}
    Df(x) v = \lim_{t\to0}\frac{f(x+tv) - f(x)}{t}.
\end{equation}
Then, the partial derivative is obtained as the particular case
\begin{equation}
    D_jf(x) := Df(x) e_j = \lim_{t\to0} \frac{f(x+te_j) - f(x)}{t}.
\end{equation}
Of course, we can also look at the derivative by using the standard euclidean coordinates $r^1, \ldots, r^n$, in that case we would be deriving $r^i \circ f : \R^n \to \R$.

Let's take it slow, and compare all these various derivatives next to each other.
For $f:U\subset\R^n\to\R^k$ and $x\in U$, we have
\begin{marginfigure}[3.5cm]
    \includegraphics{2_3-ederivs.pdf}
\end{marginfigure}
\begin{itemize}
    \item $Df(x)$, the Jacobian matrix, which is a $k\times n$ matrix;
    \item $D_j f(x)$, the $j$th column of the matrix $Df(x)$, which is an element of $\R^k$;
    \item $D(r^i\circ f)(x)$, a linear function from $\R^n to \R$, which one can think as the $i$th row of the matrix $Df(x)$;
    \item $D_j(r^i\circ f)(x) = \frac{\partial f^i}{\partial x^j}(x)$, a number in $\R$, which corresponds to the element $(i,j)$ of the matrix $Df(x)$.
\end{itemize}

This notation using $D$ instead of spelling out the partial derivatives, comes with an important advantage.
Let's use it to rewrite the chain rule form Proposition~\ref{thm:chainrule}({thm:chainrule2}):
\begin{equation}
    D_j(u^i\circ g \circ f) (x) = \sum_{r=1}^k D_r(u^i\circ g)(f(x))\; D_j(u^r \circ f)(x),
    \qquad
    1\leq i\leq m, 1\leq j \leq n.
\end{equation}
As you cen see, we did not need to spell out explicitly the coordinate systems on $\R^n$ or $\R^k$.

\subsection{Speeds of curves}

A map $\gamma\in\cC^1(I, M)$ from an open interval $I\subset\R$ to a smooth manifold $M$ is called \emph{$\cC^1$-curve} on $M$.
\begin{marginfigure}
  \includegraphics{2_2-curve-on-M.pdf}
\end{marginfigure}
F
\subsection{Via charts}
\subsection{As derivation}

\subsection{A summary}

\subsection{The tangent bundle}

\chapter{Vector fields}\label{ch:vf}
We continue with our quest of generalizing multivariable calculus.
The next familiar object waiting to be questioned are vector fields.
In the euclidean settings these are simply continuous maps that attach a vector to each point in their domain.

The step to abstract manifold is rather intuitive in this case: a vector field will be a map that, at each point of a manifold, picks a tangent vector at that point in a smooth way.

\begin{defn}
    A $C^p$-map $X: M \to TM$ with $\pi\circ X = \id_M$, or equivalently $X_p\in T_pM$ for all $p\in M$, is called \emph{$C^p$-vector field}.
    We denote\footnote{The reason will become clear in a future chapter, but it boils down to the fact that tangent vector fields are just tensor fields of type $(1,0)$.} the set of smooth vector fields by $\cT_0^1(M)$.
\end{defn}

\chapter{Lie groups and Lie algebras}
In the previous chapter we have briefly touched upon the notion of Lie algebras.
A strictly related notion, we will see in which sense, is the notion of Lie group.
There are mathematical objects that are pervasive in mathematics, even outside the realm of differential geometry, and in physics, where they play an important role in classical mechanics\footnote{You may have heard of the celebrated Noether's theorem, which states that every smooth symmetry has a corresponding conservation law}, and in high-energy physics\footnote{Does gauge theory ring any bell?}.

The theory of Lie groups and Lie algebras is vast, and in these lectures we will just briefly scratch the surface.

\begin{definition}
  A \emph{Lie group $G$} is a smooth manifold (without boundary) that is also an algebraic group, with the property that the multiplication map $G\times G \to G$, $(g,h)\mapsto gh$, and the inversion map $G\to G$, $g\mapsto g^{-1}$ are smooth.
\end{definition}

\begin{example}
  \begin{enumerate}
    \item $\R^n$ is a Lie group under addition.
    \item $\R^n\setminus\{0\}$ is a Lie group under multiplication.
    \item A manifold can be equipped with different Lie group structures. For example, the following map
    \begin{equation}
      m(x,y) = (x^1+y^1,\, x^2+y^2,\, x^3+y^+x^1y^2)
    \end{equation}
    induces\footnote{To see that this defines a group structure, identify $\R^3$ with upper triangular $3\times3$ matrices via
    \begin{equation}
      x = (x^1, x^2, x^3) \mapsto \begin{pmatrix}
        1 & x^1 & x^3\\
        0 & 1 & x^2 \\
        0 & - & 1
      \end{pmatrix}
    \end{equation}
    and observe that $m$ becomes the standard matrix multiplication.} an alternative structure of Lie group on $\R^n$ called \emph{Heisenberg group}.
    \item The set $GL(n)$ of invertible $n\times n$ matrices is a Lie group under matrix multiplication. Indeed, it is a manifold of dimension $n^2$, the product is smooth since each matrix entry is given a polynomial and the inversion is smooth thanks to Cramer's rule.
    \item The $n$-torus $\bT^n = \R^n/\Z^n$ is an abelian Lie group with the group structure induced by addition on $\R^n$.
    \item Given Lie groups $(G_1, \ldots, G_k)$, their direct product is the product manifold $G_1\times \cdots\times G_k$ with the group structure given by
    \begin{equation}
      (g_1, \ldots, g_k)(h_1,\ldots,h_k) = (g_1h_1, \ldots g_kh_k)
    \end{equation}
    is a Lie group (why?).
    \item Not all smooth manifolds can be equipped with a Lie group structure: for example, $\bS^n$ admits a Lie group structure only for $n=0,1,3$.
  \end{enumerate}
\end{example}

\begin{definition}
  A \emph{Lie group homomorphism} $F:G\to H$ is a smooth map which is also a group homomorphism. It is called \emph{Lie group isomorphism} if it is also a diffeomorphism, which implies that it has an inverse that is also a Lie group homomorphism. In this case we call $G$ and $H$ isomorphic Lie groups.
\end{definition}

\begin{example}
  It turns out that you know plenty of examples of Lie group homomorpisms.
  \begin{enumerate}
    \item The map $\exp:\R\to\R\setminus\{0\}$ is a Lie group homomorphism. The image of $\exp$ is the open subgroup $\R+$ and $\exp:\R\to\R_+$ is a Lie group isomorphim with inverse $\log:\R_+\to\R$.
    \item The map $\epsilon:\R\to\bS^1$ defined by $\epsilon(t)=e^{2\pi i t}$ is a Lie group homomorphism whose kernel is $\Z$. Similarly, the map $\epsilon^n:\R^n\to\bT^n$ defined by $\epsilon^n(x^1, \ldots, x^n)=(e^{2\pi i x^1}, \ldots, e^{2\pi i x^n})$ is a Lie group homomorphism whose kernel is $\Z^n$.
    \item The determinant function $\det:GL(n)\to\R\setminus\{0\}$ is smooth since $\det$ is a polynomial in the entries of the matrix and it is a Lie group homomorphism since $\det (AB) = \det(A) \det(B)$.
  \end{enumerate}
\end{example}

\begin{definition}
  If $G$ is a Lie group, for any element $g\in G$, we denote by $L_g:G\to G$ the \emph{left translation} and by $R_g:G\to G$ the \emph{right translation}, respectively defined
  \begin{equation}
    L_g(h) = gh
    \quad\mbox{and}\quad
    R_g(h) = hg.
  \end{equation}
\end{definition}

\marginnote{The fact that translations are diffeomorphisms of the groups onto itself is crucial, it implies that the group looks the same around any point. Indeed, they are \emph{homogeneous spaces}. To study the local structure of a Lie group, as we will see soon, it is enough to examine a neighbourhood of the identity element.}
These are both diffeomorphisms, since they can be described by a composition of smooth maps. For instance,
\begin{equation}
  \underset{h}{G} \underset{\mapsto}{\to} \underset{(g,h)}{G\times G} \underset{\mapsto}{\to} \underset{gh}{G}.
\end{equation}
Moreover, $L_{g^{-1}}$ is the inverse of $L_g$.
Similarly for $R_g$.

\begin{remark}
  For convenience, we will only consider left translations.
  There is nothing wrong with right translations and, in fact, you can reformulate all the results that follow in terms of them.
\end{remark}

The next theorem is important for understanding many of the properties of Lie group homomorphisms.

\begin{theorem}
  Every Lie group homomorphism has constant rank.
\end{theorem}
\begin{proof}
  Let $F:G\to H$ be a Lie group homomorphism and let $e$ denote the identity element of $G$.
  
  Fix $g\in G$.
  We will show that $F$ has the same rank at $g$ as its rank at $e$.
  Since $F$ is a homomorphism, for all $h\in G$ we have
  \begin{equation}
    F(L_g(h)) = F(gh) = F(g)F(h) = L_{F(g)}(F(h)),
  \end{equation}
  that is,
  \begin{equation}
    F\circ L_g = L_{F(g)}\circ F.
  \end{equation}
  Differentiating both sides at $e$ and using the chain rule, this reads
  \begin{equation}
    dF_g\circ d(L_g)_e = d(L_{F(g)})_{F(e)}\circ dF_e.
  \end{equation}
  Since the left translation is a diffeomorphism, both $d(L_g)_e$ and $d(L_{F(g)})_{F(e)}$ are isomorphisms, and as such they preserve the rank.
  From this, it follows that $dF_g$ and $dF_e$ have the same rank.
\end{proof}

The global rank theorem then immediately implies the following corollary.
\begin{corollary}
  A Lie group homomorphism is a Lie group isomorphism if and only if it is bijective.
\end{corollary}

\begin{definition}
  Let $G$ be a Lie group. A \emph{Lie subgroup} of $G$ is a subgroup $H\subset G$ endowed with a topology and a smooth structure that make it at the same time a Lie group and an immersed submanifold of $G$.
\end{definition}

\begin{example}
  This means for example that the set $GL^+(n)$ of invertible matrices with positive determinant is a Lie group.
\end{example}

It turns out that embedded submanifolds are automatically Lie groups. In fact more than that.

\begin{theorem}[Closed subgroup theorem]
  Let $G$ be a Lie group and suppose $H$ is any subgroup of $G$.
  The following are equivalent:
  \begin{enumerate}
    \item $H$ is a closed subgroup\footnote{That is, $H$ is a closed subset of $G$.};
    \item $H$ is an embedded submanifold of $G$;
    \item $H$ is an embedded Lie subgroup of $G$.
  \end{enumerate}
\end{theorem}

The proof of this theorem is not hard, but especially proving the equivalence of the first two claims is rather involved, so we will skip it.
For a proof, look at the cooresponding section in \cite[Chapter 20]{book:lee}.

\begin{example}
  \marginnote{Since the closed subgroups of $GL(n)$ play a special role in Lie groups theory, they have their own name: they are the called \emph{matrix Lie group}.}
  Let $O(n)\subset GL(n)$ denote the set of orthogonal matrices\footnote{That is, $A$ such that $AA^T = I$.}, then $O(n)$ is closed in $GL(n)$ and by the previous theorem is a Lie subgroup.
  You have proven this when you solved Exercise~\ref{exe:onsubmanifold}.
\end{example}

\section{Lie algebras}

We are finally ready to see how Lie groups and Lie algebras ended up being related.

\begin{definition}
  Let $G$ be a Lie group.
  We define the \emph{Lie algebra} of $G$, usually denoted $\fg$, as the tangent space to $G$ at the identity element $e$:
  \begin{equation}
    \fg := T_e G.
  \end{equation}
\end{definition}

Of course, for this definition not to be completely insane, the Lie algebra of a Lie group better be a Lie algebra also in the sense of Definition~\ref{def:LieAlgebra}.
We are going to prove this very soon, but let's first look at some examples.

\begin{example}
  \begin{enumerate}
    \item The Lie algebra of $GL(n)$ is $\mathfrak{gl}(n)\simeq \mathrm{Mat}(n)$.
    \item The Lie algebra of $O(n)$ is $\mathfrak{o}{n} = \{A \in \mathfrak{gl}(n) \mid A+A^T = 0\}$. You have shown it in Exercise~\ref{exe:onsubmanifold}.
  \end{enumerate}
\end{example}

\begin{exercise}
  The Lie algebra of $\bT^n$ is $\R^n$.
  
  \textit{\small Hint: using the fact that $T(M\times N) \simeq T(M)\times T(N)$ and look at what happens in the case $n=1$.}
\end{exercise}

\chapter{Cotangent bundle}\label{cg:ctb}
\section{The cotangent space}

\newthought{The dual of a vector space} should be a well-known concept from linear algebra. We recall it here just for the sake of convenience.

\begin{definition}
  Let $V$ a vector space of dimension $n\in \N$.
  Its \emph{dual space} $V^* := \cL(V, \R)$ is the $n$-dimensional real vector space of linear maps $\omega:V \to R$.
  The elements of $V^*$ are usually called \emph{linear functionals} and for $\omega\in V^*$ and $v\in V$ it is common to write
  \begin{equation}
    \omega(v) =: (\omega, v) =: (\omega \mid v).
  \end{equation}
  The notation can be deceiving: the \emph{dual pairing} $(\omega \mid v)$ is \emph{not} a scalar product.
\end{definition}

\begin{remark}\label{rmk:identification}
  Note that a scalar product $\langle\cdot, \cdot\rangle :  V\times V \to \R$ on a vector space $V$ provides a natural identification of $V$ and $V^*$ via the map $V\ni v \mapsto \langle v, \cdot \rangle =: \omega(\cdot) \in V^*$.
  Even though $\dim V = \dim V^*$ in any case, without the scalar product there is no such canonical\footnote{The keywords \emph{natural} and \emph{canonical} here are crucial: we are saying that in general there is no ``coordinate-free'' isomorphism, i.e., one that is independent of the choice of basis on $V$.} isomorphism.
\end{remark}

In the previous chapter we defined the tangent space as a special vector space over each point in a manifold, which nicely fits in the requirements above.

\begin{definition}
  Let $M$ be a differentiable manifold and $p\in M$.
  The dual space $T_p^*M := (T_pM)^*$ of the tangent space $T_pM$ is called the \emph{cotangent space} of $M$ at $p$.
  The elements of $T^*_pM$ are called \emph{cotangent vectors}, \emph{covectors} or \emph{(differential) $1$-forms} at $p$.
\end{definition}

For a function $f:\R^n\to\R$, we usually consider the gradient $\nabla f(x)$ at a point $x$ to be a vector.
On a manifold however things a slightly different.

\begin{example}[The differential of a function]
  Let $f\in C^\infty(M)$.
  Let's look carefully at its differential:
  \begin{equation}
    df_p: T_p M \to T_{f(p)}\R \simeq \R
  \end{equation}
  is a linear function from the tangent space to $\R$.
  In other words, $df_p \in T_p^* M$.
\end{example}

Whereas tangent vectors give us a coordinate-free interpretation of derivatives (of curves), it turns out that derivatives of real-valued functions on a manifold are most naturally interpreted as cotangent vectors.

Indeed, we saw that the action of $df_p$ on a tangent vector $v\in T_p M$ can be interpreted as the directional derivative of $f$ at $p$ in the direction $v$ and, by using Definition~\ref{def:tg:ascurvespeed}, we have
\begin{equation}
  df_p(v) = \frac{d}{dt}f(\gamma(t))\Big|_{t=0}
\end{equation}
for some curve $\gamma$ with $\gamma(0) = p$ and $\gamma'(0)=v$.
We also know that the equation above can be rewritten by thinking of $v$ as a derivation, giving
\begin{equation}
  df_p(v) = v(f).
\end{equation}
That is, we can think of the dual pairing $(df\mid v)$ in a twofold way:
\begin{itemize}
  \item as a linear action of the covector $df$ on the vector $v$;
  \item as the linear action of the vector $v$ as a derivation operating on the function $f$.
\end{itemize}

\begin{notation}
  In analogy to the notation $\frac{\partial}{\partial x}\big|_p$ that we used for tangent vectors, when convenient we may write $df|_p$ instead of $df_p$.
\end{notation}

To look more concretely at differential forms, let's compute its coordinate representation.
Let $(U,(x^i))$ be a chart on $M^n$.
Since the coordinate functions $x^i\in C^\infty(U)$ are smooth real valued functions, we can define the corresponding coordinate $1$-forms $dx^i|_p \in T_p^* M$.
Their action on the coordinate vector fields, then, is immediately computed as
\begin{equation}
  \left(dx^i|_p ,\; \frac{\partial}{\partial x^j}\Big|_p\right) =
  dx^i|_p \left(\frac{\partial}{\partial x^j}\Big|_p\right)
  = \frac{\partial}{\partial x^j}\Big|_p x^i
  = \delta^i_j.
\end{equation}
Which proves the following statement.

\begin{proposition}
  Let $(x^i)$ be local coordinates on an open subset $U\subseteq M^n$.
  At each point $p\in U$, the covectors $\left\{dx^1|_p, \ldots, dx^n|_p\right\}$ form a basis for the cotangent space $T_p^* M$ which is dual to the basis $\left\{\frac{\partial}{\partial x^1}\Big|_p, \ldots, \frac{\partial}{\partial x^n}\Big|_p\right\}$ for the tangent space $T_p M$.
\end{proposition}

That is, any $1$-form $\omega$ can be locally written as a linear combination
\begin{equation}
  \omega = \omega_i\; dx^i
\end{equation}
where $\omega_i:U\to\R$.
In particular, if $f\in C^\infty(M)$, the restriction $df$ to points in $U$ should have the same form.
Evaluating it on a coordinate vector field gives, for all $p\in U$,
\begin{align}
   & \underset{\shortparallel}{df_p} \left(\frac{\partial}{\partial x^j}\Big|_p\right)
  = \omega_i\; dx^i|_p \left(\frac{\partial}{\partial x^j}\Big|_p\right) =
  \omega_i \delta^i_j = \omega_j                                                       \\
   & \frac{\partial f}{\partial x^j}(p).
\end{align}

That is, the local expression for $df$ is
\begin{equation}\label{eq:localformofthedifferential}
  df =  \frac{\partial f}{\partial x^i} dx^i.
\end{equation}

\begin{remark}
  In calculus 1 you have probably been told that you can cancel out differentials when applying solving differential equations.
  This was probably accompanied by a warning that it is just a formal thing, a computational convenience.
  We can finally make sense of that in a general context: in one dimension,~\eqref{eq:localformofthedifferential}, reads as
  \begin{equation}
    d f = \frac{df}{dt}dt.
  \end{equation}
\end{remark}

\begin{example}\label{ex:diff1}
  If $f(x,y) = x y^2 e^{3x}$ on $\R^2$, then $df$ is given by the formula
  \begin{align}
    df
     & = \frac{\partial (x y^2 e^{3x})}{\partial x} dx + \frac{\partial (x y^2 e^{3x})}{\partial y} dy \\
     & = (y^2 e^{3x} +3xy^2 e^{3x}) dx + 2xy e^{3x} dy.
  \end{align}
\end{example}

With the local basis, computing with covectors becomes much easier.
Given a covector $\omega = \omega_j\; dx^j$ and a vector $v = v^i \frac{\partial}{\partial x^i}$ expressed in the respective coordinate bases for the local coordinates $(x^i)$, by linearity in both arguments the dual pairing takes the form
\begin{equation}\label{eq:localdualpairing}
  (\omega \mid v) =
  \left(\omega_j\; dx^j \;\Big |\; v^i \frac{\partial}{\partial x^i} \right) =
  \omega_j v^i \left(dx^j \;\Big |\; \frac{\partial}{\partial x^i} \right) =
  \omega_j v^j.
\end{equation}

\begin{example}
  Let, now, $v = 7 \frac{\partial}{\partial x}\Big|_{(1,2)} + 3 \frac{\partial}{\partial y}\Big|_{(1,2)}\in T_{(1,2)}\R^2$, and $f$ from Example~\ref{ex:diff1}.
  We have
  \begin{align}
    (df|_{(1,2)}, v)
     & = \left((y^2 e^{3x} +3xy^2 e^{3x}) dx + 2xy e^{3x} dy, 7 \frac{\partial}{\partial x} + 3 \frac{\partial}{\partial y} \right)\Big|_{(1,2)} \\
     & = 7(y^2 e^{3x} +3xy^2 e^{3x})|_{(1,2)} + 6 xy e^{3x}|_{(1,2)}                                                                             \\
     & = 7(4 e^3 + 12 e^3) + 12 e^3 = 52e^3.
  \end{align}
\end{example}

\begin{exercise}
  Let $M$ be a smooth manifold and let $f,g\in C^\infty(M)$. Show that the following properties hold:
  \begin{enumerate}
    \item $d(\alpha f + \beta g) = \alpha df + \beta dg$ for $\alpha,\beta\in\R$;
    \item $d(fg) = f dg + g df$;
    \item $d(f/g) = (g df - f dg)/g^2$ on the set where $g\neq 0$;
    \item if $J\subseteq\R$ contains the image of $f$ and $h:J\to\R$ is smooth, then $d(h\circ f) = (h'\circ f) df$;
    \item if $f$ is constant, then $df= 0$.
  \end{enumerate}
\end{exercise}

\begin{remark}[The double dual]
  We said in Remark~\ref{rmk:identification} that unless we have an inner product, there is no canonical identification of a vector space with its dual.
  This is true also for the tangent and cotangent spaces.
  However, the situation is different for the double dual $T^{**}_pM := (T^*_pM)^*$.

  For $v\in T_p M$, the map
  \begin{equation}
    i_v: T_p^*M\to \R, \qquad
    \omega \mapsto i_v(\omega) := (\omega\mid v)
  \end{equation}
  is linear and therefore $i_v\in T^{**}_pM$.

  Furthermore the map $i : T_pM \to T_p^{**}M$, $v\mapsto i_v$, is a vector space isomorphism. Indeed, it is injective since $\ker(i) = \{0\}$ and since $\dim T_p M = \dim T^{**}_p M$ also surjective\footnote{Why? If \emph{rank-nullity theorem} does not ring a bell, make sure to look it up. It is, e.g.,~\cite[Corollary B.21]{book:lee}}.

  That is, $T^{**}_pM$ can be canonically identified with $T_p M$.

  So, to add up to our list of interpretations of geometric objects, we now have seen that
  \begin{itemize}
    \item a covector can act as a linear functional on vectors;
    \item a vector can act as a linear functional on covectors.
  \end{itemize}
\end{remark}

This should start giving you an idea of what is behind the following famous quote by Henri Poincar\'e:
\begin{quote}
  Mathematics is the art of giving the same name to different things.
\end{quote}

\section{Change of coordinates}

In Remark~\ref{rmk:chg_coords} we have seen that if we have two different charts with local coordinates $(x^i)$ and $(y^i)$ on a smooth manifold $M$,
\marginnote{Let's denote the two charts respectively by $\varphi$ and $\psi$, then if $\phi = \psi\circ\varphi^{-1}$ is the corresponding transition map, one has \begin{equation}
    \frac{\partial y^j}{\partial x^i}(p) \frac{\partial}{\partial y^j}\Big|_p = (D\phi(p))_i^j \frac{\partial}{\partial y^j}\Big|_p,
  \end{equation}
  where $D\phi$ is the Jacobian matrix of the transition map.}
\begin{equation}
  \frac{\partial}{\partial x^i}\Big|_p = \frac{\partial y^j}{\partial x^i}(p) \frac{\partial}{\partial y^j}\Big|_p.
\end{equation}
Thus, if $v\in T_pM$ has local form $v = v^i \frac{\partial}{\partial x^i}\big|_p = \widetilde v^j \frac{\partial}{\partial y^j}\big|_p$, we get
\begin{align}
   & \underset{\shortparallel}{v^i \frac{\partial}{\partial x^i}\Big|_p} = v^i \frac{\partial y^j}{\partial x^i}(p) \frac{\partial}{\partial y^j}\Big|_p \\
   & \widetilde v^j \frac{\partial}{\partial y^j}\Big|_p,
\end{align}
or, reading off the basis elements,
\begin{equation}\label{eq:contravariant}
  \widetilde v^j = \frac{\partial y^j}{\partial x^i}(p) v^i.
\end{equation}

Let now $\omega\in T_p^*M$ with local form $\omega = \omega_i dx^i|_p = \widetilde \omega_j dy^j|_p$.
In analogy to our previous computations we get
\begin{equation}
  \omega_i
  = \omega\left(\frac{\partial}{\partial x^i}\Big|_p\right)
  = \omega\left(\frac{\partial y^j}{\partial x^i}(p) \frac{\partial}{\partial y^j}\Big|_p\right)
  = \frac{\partial y^j}{\partial x^i}(p) \widetilde\omega_j.
\end{equation}
That is,
\begin{equation}\label{eq:covariant}
  \omega_i = \frac{\partial y^j}{\partial x^i}(p) \widetilde\omega_j.
\end{equation}

There is an important difference\footnote{I have borrowed this explanation from~\cite[Chapter 11]{book:lee}.} between~\eqref{eq:covariant} and~\eqref{eq:contravariant}.
For covectors,~\eqref{eq:covariant} shows that their components transform in the same way as (``vary with'') the coordinate partial derivatives: the Jacobian of the change of variables $\frac{\partial y^j}{\partial x^i}(p)$ multiplies the objects associated to the ``new'' coordinates $y^j$ to obtain the objects associated to the ``old'' coordinates $x^i$.
For this reason covectors are said to be \emph{covariant vectors}.
Analogously, tangent vectors are said to be \emph{contravariant vectors}, since~\eqref{eq:contravariant} shows that their components transform in the opposite way.

The difference in the way vector and covector transform is reflected also in the way they are transformed by smooth maps between manifolds.
As we have seen, the differential of a smooth map yields a linear map between tangent spaces that pushes vectors from one space to the other.
Its dual is going to be a map that pulls vector form one covector space to another.

\begin{definition}\label{def:pullback:oneform}
  Let $F:M\to N$ be a smooth map between smooth manifolds, let $\omega\in T^*_{F(p)}N$ for some $p\in M$.
  The \emph{pullback} of covectors by $F$ at the point $F(p)$, is the dual linear map of the differential
  \marginnote[-1.5em]{Which is also the reason why, some authors, use the notation $F_*$ to denote the differential of maps between manifolds and other call pushforward the differential.}
  \begin{equation}
    dF^*_p : T^*_{F(p)} N \to T^*_p M, \quad \omega \mapsto dF^*\omega,
  \end{equation}
  defined by duality in the following way\footnote{Or, omitting the point of application, $\left(dF^*\omega, v\right) := \left(\omega, dF(v)\right)$.}:
  \begin{equation}
    \left(dF^*_p\omega \mid v\right) := \left(\omega \mid dF_p(v)\right),\quad
    \forall v\in T_pM,\; \forall \omega\in T^*_{F(p)}N.
  \end{equation}
\end{definition}
\marginnote{Equations are getting more and more tricky: this kind of dimensional analysis is extremely useful to check that you are doing the right thing.}
\noindent Let's check that the definition above makes sense: $dF^*_p \omega \in T^*_p M$ so $v\in T_p M$, but $\omega\in T^*_{F(p)}N$ so $dF_p(v)\in T_{F(p)}N$ since $dF_p: T_pM\to T_{F(p)}N$.

\section{One-forms and the cotangent bundle}

In analogy to Chapter~\ref{sec:tangentbundle} we can glue the cotangent space together into a vector bundle on $M$.

\begin{definition}
  The \emph{cotangent bundle} $T^*M$ of $M$ is the disjoint union of cotangent spaces
  \begin{equation}
    T^*M := \bigsqcup_{p\in M}\left(\{p\}\times T^*_pM\right)
    = \{(p,\omega) \;\mid\; p\in M,\, \omega\in T^*_pM\}.
  \end{equation}
\end{definition}

The cotangent bundle is a vector bundle of rank $n$ with projection $\pi:T^*M\to M$, $(p,\omega)\mapsto p$.
The cotangent spaces are the fibres of the cotangent bundle.

\begin{theorem}\label{thm:starmbld}
  Let $M$ be a smooth $n$-manifold.
  The smooth structure on $M$ naturally induces a smooth structure on $T^*M$, making $T^*M$ into a smooth manifold of dimension $2n$.
\end{theorem}
%\begin{proof}
% The proof is analogous to that of Theorem~\ref{thm:tgbdlsmoothmfld}, so we will not do it again.
% The atlas is obtained by an atlas $\{(U_i, \varphi_i)\}$ of $M$ by defining the new atlas
% \begin{equation}
%   \left\{ T^* U_i, \left(\varphi_i, d(\varphi_i^{-1})^*\right)\right\}
% \end{equation}
% where
% \begin{align}
%   \left(\varphi_i, (\varphi_i^{-1})^*\right) : T^* U_i &\to T^*\varphi(U_i),\\
%   (p, \omega) &\mapsto \left(\varphi_i(p), d(\varphi_i^{-1})^*)_p\omega \right).
% \end{align}
\begin{exercise}\label{exe:prooftstarmbld}
  Mimicking what we did for Theorem~\ref{thm:tgbdlsmoothmfld}, complete the proof of Theorem~\ref{thm:starmbld}.
\end{exercise}
%\end{proof}

In fact, the cotangent bundle is a specific example of a dual bundle.
In the exercise below, we will construct it using Theorem~\ref{thm:bundle_chart_thm}.
\begin{exercise}[The dual bundle]
Let $M$ be a smooth manifold and $\pi:E\to M$ a smooth $k$-vector bundle over $M$.
The \emph{dual bundle} to $E$ over $M$ is the bundle $E^* \to M$, where $E= \bigsqcup_{p\in m} E_p^*$ is the disjoint union of the duals to the fibers of $E$ with the projection $E_p^* \mapsto p$.
\begin{enumerate}
  \item Show that the dual bundle is a smooth vector bundle of rank $k$.
  \item Show that the transition functions are given in terms of the transition functions $\tau : U\to GL(\R,k)$ of $E$ by $\tau^*(p) = (\tau(p)^{-1})^T$.
\end{enumerate}
\end{exercise}

\begin{definition}\label{def:covfield}
  A \emph{covector field} or a \emph{(differential) $1$-form} on $M$ is a smooth section of $T^*M$.
  That is, a $1$-form $\omega\in\Gamma(T^*M)$ is a smooth map $\omega: p \to \omega_p \in T_p^*M$ that assigns to each point $p\in M$ a cotangent vector at $p$.
  Again, a covector field is smooth if its component functions w.r.t. all charts are smooth.

  We denote the space of all smooth covector fields on $M$ by $\fX^*(M)$.
  \marginnote[-1em]{For reasons related to tensor fields that we will understand soon, this is sometimes denoted $\cT_1^0(M)$.}

  Of course, one can also define \emph{$C^p$-covector fields} as the $C^p$-maps $\omega:M\to T^*M$ such that $\pi\circ\omega = \id_M$.
\end{definition}

\begin{remark}
  In Exercise~\ref{exe:prooftstarmbld} you have shown that coordinate covector fields are smooth local sections for the cotangent bundle.
\end{remark}

In analogy with the previous chapters, for covector fields $\omega\in\fX^*(M)$ we will often make the identification of its value $\omega(p) = \omega_p \in \{p\}\times T^*_p M$ at $p\in M$ with its part in $T_p^*M$ without necessarily making this explicit in the notation by projecting on the second factor.

\begin{example}
  Let $f\in C^\infty(M)$, then the map
  \begin{equation}
    df : M \to T^*M, \quad p \mapsto df|_p \in T^*_p M
  \end{equation}
  defines a $1$-form $df\in\fX^*(M)$.
\end{example}

As smooth sections of a vector bundle, covector fields can be multiplied by smooth functions: if $f\in C^\infty(M)$ and $\omega\in\fX^*(M)$, the covector field $f\omega$ is defined by
\begin{equation}
  (f\omega)_p = f(p)\omega_p.
\end{equation}
Also in this case, $\fX^*(M)$ is a module over $C^\infty(M)$.

Since differential 1-forms are dual objects to tangent vectors, the action of a form $\omega$ on $X\in\fX(M)$ is well--defined and pointwise defines a function
\begin{equation}
  (\omega \mid X) : p \mapsto (\omega_p \mid X_p).
\end{equation}

\begin{exercise}
  The differential form $\omega$ is smooth if and only if, for every smooth vector field $X\in\fX(M)$, the function $(\omega \mid X)\in C^\infty(M)$.
  \textit{\small Hint: write it down in local coordinates.}
\end{exercise}

\begin{definition}\label{def:pullback1f}
  The pullback of covectors immediately extends to covector fields.
  The \emph{pullback} is the map
  \begin{equation}
    F^*: \fX^*(N) \to \fX^*(M), \quad \omega \mapsto F^* \omega
  \end{equation}
  defined by
  \begin{equation}
    (F^*\omega)_p := dF_p^*(\omega_{F(p)}).
  \end{equation}
  By definition, this acts on vectors $v\in T_p M$ by
  \begin{equation}
    ((F^*\omega)_p, v) = (\omega_{F(p)}, dF_p(v)) = \omega_{F(p)}(dF_p(v)).
  \end{equation}
\end{definition}

\begin{exercise}\label{ex:propdiff}
  Let $F:M\to N$ be a smooth map between smooth manifolds.
  Suppose $f$ is a continuous real valued function on $N$ and $\omega\in\fX^*(N)$ is a covector field on $N$.
  \begin{enumerate}
    \item Show that
          \begin{equation}
            F^*(f\omega) = (f\circ F)F^*\omega := F^* f\; F^*\omega,
          \end{equation}
          where we introduced the \emph{pullback} of a smooth function as $F^*g := g\circ F$.
    \item If in addition $f\in C^\infty(N)$, show that
          \begin{equation}
            F^* df = d (f\circ F) = d (F^* f).
          \end{equation}
  \end{enumerate}
  \textit{\small Hint: apply the equations at a point $p\in M$ and keep rewriting the equations in different forms.}
\end{exercise}

\begin{exercise}
  Let $F:M\to N$ smooth map between smooth manifolds.
  For $p\in M$, denote $(V, (y^i))$ a chart on $N$ around $F(p)$ and let $U=F^{-1}(N)$.
  If $\omega = \omega_j dy^j \in\fX^*(N)$, apply twice Exercise~\ref{ex:propdiff} to show that in $U$
  \begin{equation}
    F^*\omega = (\omega_j\circ F) d(y^j \circ F).
  \end{equation}

  Let $F:\R^3\to\R^2$ be the map $(u,v) = F(x,y,z) = (x y^2, y \sin z)$.
  Let $\omega\in\fX^*(\R^2)$ denote the covector field $\omega(u,v) = u dv - v du$.
  Compute $F^* \omega$.
\end{exercise}

Exercise~\ref{ex:propdiff} is particularly interesting if we look at it in relation to the pushforward.
\begin{proposition}
  Let $F:M\to N$ be a diffeomorphism and $X\in\fX(M)$.
  Then, for any $f\in C^\infty(N)$,
  \begin{equation}
    X(F^* f) = F_*X(f) \circ F.
  \end{equation}
\end{proposition}
\begin{proof}
  Indeed, for any $p\in M$,
  \begin{align}
    F_*X(f) \circ F(p) & = ((F_*X) f) (F(p)) = (F_*X)_{F(p)} f                  \\
                       & = (dF \circ X \circ F^{-1})(F(p)) f = (dF\circ X)(p) f \\
                       & = dF_p(X_p) f,                                         \\
    X (F^*f)(p)        & = X(f\circ F)(p) = X_p(f\circ F) = dF_p(X_p) f         \\
    % &= d (f\circ F) \circ X = df \circ dF \circ X \\
    % & = df \circ F_* X \circ F = F_* X (f) \circ F.
  \end{align}
\end{proof}
In this case you often say that the vector fields are $F$-related\footnote{This is a definition that can be properly formalized, but we will not spend any time on it in during the course.} or that they behave naturally: you can either pull back the function $f$ to $M$ or push forward the vector field $X$ to $N$.

\begin{exercise}
  Let $\{\rho_\alpha\}$ denote a partition of unity on a manifold $M$ subordinate to an open cover $\{U_\alpha\}$.
  Let $F:N\to M$ denote a smooth map between smooth manifolds.
  With the definition of pullback of functions given above, prove that
  \begin{enumerate}
    \item the collection of supports $\{\supp F^*\rho_\alpha\}$ is locally finite;
    \item the collection of functions $\{F^*\rho_\alpha\}$ is a partition of unity on $N$ subordinate to the open cover $\{F^{-1}(U_\alpha)\}$ of $N$.
  \end{enumerate}
\end{exercise}

When we discussed vector fields, we observed that pushforwards of vector fields under smooth maps are defined only in the special case of diffeomorphisms.
The surprising thing about covectors is that covector fields always pull back to covector fields.

\begin{example}[Polar coordinates on $\R^2$]
  We can define polar coordinates in $\R^2$ via the map
  \begin{align}
    \psi : \R_+\times (-\pi, \pi) & \to \R^2\setminus\{x\in\R^2\mid x^2=0 \mbox{ and } x^1 \leq 0\} \\
    (r, \theta)                   & \mapsto (r\cos\theta, r\sin\theta).
  \end{align}
  It is immediate to check that $\psi$ is a diffeomorphism between open subsets of $\R^2$, and we can think of $\psi^{-1}$ as local coordinates for a part of $\R^2$.

  On the image of $\psi$ we have the coordinate basis $\{dx^1, dx^2\}$. In order to express them in terms of the coordinate basis $\{dr,d\theta\}$, we can apply Exercise~\ref{ex:propdiff}, the properties of differentials and the formulas for the change of coordinates to get
  \begin{align}
    \psi^*(d x^1) & = d(x^1\circ \psi) = d(r\cos\theta)                                            \\
                  & = \cos\theta \,dr +r\,d(\cos\theta) = \cos\theta \,dr -r\,\sin\theta\,d\theta  \\
    \psi^*(d x^2) & = d(x^2\circ \psi) = d(r\sin\theta)                                            \\
                  & = \sin\theta \,dr +r\,d(\sin\theta) = \sin\theta \,dr +r\,\cos\theta\,d\theta.
  \end{align}
\end{example}

\begin{example}[Tautological one-form]
  On $T^*M$ there is a $1$-form, called\footnote{As usual there are different names: two other common ones are \emph{Liouville form} or \emph{Poincar\'e form}, but don't be suprised if you find more.} \emph{tautological one-form}, defined as follows.

  A point in $T^*M$ is a covector $\omega_p\in T^*_p M$ at some point $p\in M$. If $X_{\omega_p}\in T_{\omega_p}(T^*M)$ is a tangent vector to $T^*M$ at $\omega_p$. Let $\pi:T^*M \to M$, then the pushforward $\pi_*(X_{\omega_p})\in T_p M$ is a tangent vector to $M$ at $p$.
  Therefore, one can pair $\omega_p$ and $\pi_*(X_{\omega_p})$ to obtain a real number $\left(\omega_p\;\big|\;\pi_*(X_{\omega_p})\right)$.
  The tautological one-form $\theta\in\fX^*(T^*M)$ is then defined as
  \begin{equation}
    \theta_{\omega_p}(X_{\omega_p}) := \left(\omega_p\;\Big|\;\pi_*(X_{\omega_p})\right).
  \end{equation}

  This is a very important concept in symplectic and contact geometry and in the mathematical theory of classical mechanics.
\end{example}

The pullback is a rather pervasive concept, and does provide us a new way to explore vector bundles.

\begin{example}[The pullback bundle]
  Let $F:M\to N$ be a smooth map between manifolds. Suppose that $\pi: E \to N$ be a vector bundle of rank $r$ over $N$.
  Then $M\times E$ is a trivial bundle over $M$ with constant fibre $E$.
  You may think that this is yet another trivial example, but it allows us to define the \emph{pullback bundle $F^* E$}: let
  \begin{equation}
    F^* E := \left\lbrace (p, v) \in M\times E \mid F(p) = \pi(v)\right\rbrace,
  \end{equation}
  with the projection $\Pi_1:F^* E \to M$.
  The fibre of $F^*E$ over $p\in M$, then, is $\{p\}\times E_{F(p)}$, which under $\Pi_2:F^* E \to E$ is diffeomorphic to $E_{F(p)}$.
  If $\varphi : \pi^{-1}(U) \to U\times\R^r$ is a bundle diffeomorphism for $E$, then $\varphi\circ\Pi_2: \Pi_1^{-1}(F^{-1}(U)) \to U\times\R^r$ is a bundle diffeomorphism for $F^*E$.
  This $F^*E$ is a vector bundle of rank $r$ over $M$.
  In summary, the following diagram commutes:
  \begin{equation}
    \begin{tikzcd}[row sep=huge, column sep=huge]
      F^* E \arrow[r, "\Pi_2"] \arrow[d, "\Pi_1"]
      & E \arrow[d, "\pi"] \\
      M \arrow[r, "F"]
      & N
    \end{tikzcd}.
  \end{equation}
\end{example}

\begin{exercise}
  Prove that the Whitney Sum\footnote{See Exercise~\ref{ex:whitney}.} 
  of two vector bundles $\pi_1 : E_1 \to M$ and $\pi_2 : E_2 \to M$
  is the pullback $\Delta^*(E_1 \times E_2)$ of their product bundle by the diagonal map
  $\Delta : M \to M \times M$, $\Delta(x) = (x, x)$.
\end{exercise}

\section{Line integrals}

An important direct feature of $1$-forms is that they are the natural geometric objects that can be integrated along $1$-dimensional (oriented) submanifolds, i.e. along curves.
In this sense they provide a coordinate-free way to define line integrals.
We will not see this in too many details yet, but it is worth taking the time to give the definition and see a few properties.

The idea is to use the pullback to pull back the 1-form to the parameter space $\R$ and interpret the integral there as a usual Riemann integral.

\begin{definition}
  Let $M$ be a smooth manifold, $\gamma: I = [a,b]\subset \R \to M$ a smooth curve and $\omega\in\fX^*(M)$ a $1$-form.
  The \emph{(line) integral of $\omega$ along $\gamma$} is the number
  \begin{equation}
    \int_\gamma \omega :=
    \int_I \gamma^*\omega :=
    \int_a^b \left(\gamma^*\omega \mid \frac{\partial}{\partial t}\right)(t)\, dt
  \end{equation}
  where $\gamma^*\omega$ is the pullback of $\omega$ to $I$ by $\gamma$ and $\frac{\partial}{\partial t}: I \to TI$ is the unit vector field on $I$.
  The pointwise dual pairing $\left(\gamma^*\omega \mid \frac{\partial}{\partial t}\right)\in C^\infty(I)$ and is integrated in the usual Riemannian sense.
\end{definition}

\begin{example}\label{ex:li}
  Let $M=\R^2\setminus\{0\}$. Let $\omega$ be the one-form
  \begin{equation}
    \omega = \frac{x dy - y dx}{x^2 + y^2}
  \end{equation}
  and let $\gamma:[0,2\pi]\to M$ be the curve segment defined by $\gamma(t) = (\cos t, \sin t)$.

  We already saw that thanks to covariance, $\gamma^*\omega$ is immediately computed with the substitution $x=\cos t$ and $y=\sin t$ in the definition of $\omega$, so we get
  \begin{align}
    \int_\gamma \omega
     & = \int_{[0,2\pi]} \frac{\cos t\, d(\sin t) - \sin t \, d(\cos t)}{\sin^2 t + \cos^2 t} \\
     & = \int_{[0,2\pi]} (\cos t\, \cos t\, dt - \sin t \, (-\sin t)\, dt)                    \\
     & = \int_0^{2\pi} dt = 2\pi.
  \end{align}
\end{example}

\begin{exercise}\label{exe:FTC}
  Let $M$ be a smooth manifold, $\gamma: I = [a,b]\subset \R \to M$ a smooth curve and $\omega\in\fX^*(M)$ a $1$-form.
  Show the following properties.
  \begin{enumerate}
    \item Show that with the definition above
          \begin{equation}\label{eq:lineIntCurve}
            \int_\gamma \omega = \int_a^b \omega_{\gamma(t)}(\gamma'(t))\, dt.
          \end{equation}
    \item Let $J\subset\R$ be an open interval  and $F: J\to I$ a diffeomorphism with $F'(t) > 0$.
          If $\delta : J \to M$ denotes the reparametrisation of $\gamma$ defined by $\delta(t) := F^*\gamma(t) = (\gamma\circ F)(t)$, show that
          \marginnote{This shows that line integrals are independent of the parametrization.}
          \begin{equation}
            \int_\delta \omega = \int_\gamma \omega.
          \end{equation}
          \textit{\small Hint: use the chain rule to get $\delta'(t) = \gamma'(F(t))F'(t)$ and then apply~\eqref{eq:lineIntCurve}.}
    \item Let $f\in C^\infty(M)$. Prove the fundamental theorem of calculus:
          \begin{equation}
            \int_\gamma df = f(\gamma(b)) - f(\gamma(a)).
          \end{equation}
          \textit{\small Hint: justify that $df_{\gamma(t)}(\gamma'(t)) = \frac{d}{ds}f(\gamma(s))\big|_{s=t}$ and then use the usual fundamental theorem of calculus on $\R$.}
  \end{enumerate}
\end{exercise}

\begin{example}[One-forms in thermodynamics]
  Consider a physical system composed of a fixed number of particles.
  The thermal equilibrium state of the system can be characterised in terms of its entropy $S\in\R_+$ and its volume $V\in\R_+$.
  If we think at the thermodynamic state space $M=\R_+^2 \subset \R^2$ as a smooth manifold, we can define the energy of the system as a function $E = E(S,V): M \to\R$ on the space of equilibrium states.

  Show that the differential $dE\in\fX^*(M)$ has the following representation with respect to the coordinate basis $\{dS,dV\}$:
  \begin{equation}
    dE = \frac{\partial E}{\partial S}dS + \frac{\partial E}{\partial V} dV =: T dS - p dV.
  \end{equation}
  Here $T$ and $p$ are the two functions denoting respectively the temperature of the system and its pressure.
  The $1$-form $TdS$ is called the \emph{heat} absorbed by the system while $-p dV$ is the \emph{work} performed by the system.

  Differently from the other properties of the system, these are not functions and thanks to this it makes sense to ask how much heat has been transferred or how much work has been performed: these are just the integrals of those one-forms over curves in the space of equilibrium states.

  Note that since the energy is the differential of a function, its integral over a closed curve is just the difference between initial and final energy and, thus, it vanishes.
  However, work and heat are usually \emph{not} the differential of a function, which makes their integral dependent on the specific path taken and usually not vanish on closed loops.
  This peculiar property is what makes possible to construct heat engines.
\end{example}


\chapter{Tensor fields}\label{cg:tf}
\marginnote{For a brief and \emph{concrete} explanation of tensors, I warmly recommend the following \href{https://youtu.be/f5liqUk0ZTw}{youtube video by Dan Fleisch} and \cite[Chapter XIV]{book:lieber}.}
Many of the spaces that we have encountered so far are particular examples of a much larger class of objects.
In this chapter we are going to introduce all the necessary algebraic concepts.

We have seen that covectors in $V^*$ are real linear maps $V\to\R$ from the underlying space $V$ while, through the double dual, vectors can be understood as real linear maps $V^*\to\R$ from the dual space $V^*$.
In practice, \emph{tensors} are just multilinear real-valued maps on cartesian products of the form $V^*\times \cdots \times V^* \times V \times \cdot \times V$.
We have already encountered some examples; covectors, inner products and even determinants are examples of tensors:
\begin{itemize}
  \item a scalar product is a bilinear map $\langle\cdot,\cdot\rangle:V\times V\to \R$;
  \item the signed area spanned by two vectors is a bilinear map $\R^2\times\R^2\to\R$ defined by $\mathrm{area}(u,v) := u\wedge v = u^1v^2-u^2v^1$;
  \item the determinant\footnote{In fact, the signed area is the determinant of the $2\times 2$ matrix $(u \, v)$...} of a square matrix in $\mathrm{Mat}(n)$, viewed as a function $\det: \LaTeXunderbrace{\R^n\times\cdots\times\R^n}_{n\mbox{ times}}\to\R$ is a $n$-linear map.
\end{itemize}

So functions of several vectors or covectors that are linear in each argument are also called multilinear forms or tensors.
It should not come as a surprise that multilinear functions of tangent vectors and covectors to manifolds appear naturally in different geometrical and physical contexts.
In this chapter we are going to discuss the general definitions and notions that interest us, some of which may be just refreshing what you have seen in multivariable analysis in the context of general vector spaces $V$.
Keep in mind, that at a certain point, we will replace $V$ with the tangent spaces $T_pM$ of a smooth manifold $M$.

\section{Tensors}

\begin{definition}
  Let $V$ be a $n$-dimensional vector space and $V^*$ its dual.
  Let
  \begin{equation}
    \mathrm{Mult}(V_1, \ldots, V_k)
  \end{equation}
  denote the space of multilinear maps $V_1\times\cdots\times V_k\to\R$.

  A multilinear map
  \begin{equation}
    \tau : \LaTeXoverbrace{V^*\times \cdots \times V^*}^{r\mbox{ times}} \times \LaTeXunderbrace{V \times \cdots \times V}_{s\mbox{ times}} \to \R
  \end{equation}
  is called \emph{tensor of type $(r,s)$}, $r$-contravariant $s$-covariant tensor, or $(r,s)$-tensor.
  Similarly as we did for the dual pairing, when convenient we define the pairing
  \begin{equation}
    \tau\left(\omega^1, \ldots, \omega^r; v_1, \ldots, v_s\right) 
    =: \left(\tau \mid \omega^1, \ldots, \omega^r; v_1, \ldots, v_s \right).
  \end{equation}

  For tensors $\tau_1$ and $\tau_2$ of the same type $(r,s)$ and $\alpha_1, \alpha_2\in\R$ we define
  \begin{equation}
    \left(\alpha_1\tau_1 + \alpha_2\tau_2 | \ldots \right) := \alpha_1\left(\tau_1 | \ldots \right) + \alpha_2 \left(\tau_2 | \ldots \right).
  \end{equation}
  This equips the space 
  \begin{equation}
    T^r_s(V) := \mathrm{Mult}(\LaTeXoverbrace{V^*,\ldots,V^*}^{r \mbox{ times}}, \LaTeXunderbrace{V, \ldots, V}_{s \mbox{ times}})
  \end{equation}
  of tensors of type $(r,s)$ with the structure of a real vector space\footnote{Be careful when reading books and papers, for tensor spaces the literature is wild: there are so many different conventions and notations that there is not enough space on this margin to mention them all. Note that the book of Lee inverts the order of superscripts and subscripts in $T^r_s$.}. %of dimension $(\dim V)^{r+s}$.
  In particular, $V^* = T_1^0(V)$ and $V=T_0^1(V)$.
\end{definition}

\begin{example}
  \begin{itemize}
    \item An inner product on $V$, e.g. the scalar product in $\R^n$, is a $(0,2)$-tensor.
    This means, for example that the aforementioned scalar product is an element of $T^0_2(\R^n)$.
    \item The determinant, thought as a function of $n$ vectors, is a tensor in $T_n^0(\R^n)$.
    \item Covectors are elements of $T_1^0(T_pM)$ while tangent vectors are elements of $T_0^1(T_pM)$.
  \end{itemize}
\end{example}

Take now, for example, two covectors $\omega^1, \omega^2 \in V^*$. We can define the bilinear map
\begin{equation}
  \omega^1\otimes \omega^2 : V\times V \to \R,\quad
  \omega^1\otimes \omega^2(v_1, v_2) = \omega_1(v_1)\omega_2(v_2),
\end{equation}
called the tensor product of $\omega^1$ and $\omega^2$.
This can be generalized immediately to general tensors in order to define new higher order tensors.

\begin{definition}
  Let $V$ an $n$-dimensional vector space, $\tau_1\in T_s^r(V)$, $\tau_2\in T_{s'}^{r'}(V)$.
  We define the \emph{tensor product} $\tau_1\otimes\tau_2$ as the $(r+r', s+s')$-tensor defined by
  \begin{align}
    &\tau_1\otimes\tau_2(\omega^1,\ldots,\omega^{r+r'}, v_1,\ldots,v_{s+s'}) \\
    &= \tau_1(\omega^1,\ldots,\omega^{r}, v_1,\ldots,v_{s}) \tau_2(\omega^{r+1},\ldots,\omega^{r+r'}, v_{s+1},\ldots,v_{s+s'}).
  \end{align}
\end{definition}

This definition immediately implies that the map
\begin{equation}
  \otimes :  T_s^r(V)\times T_{s'}^{r'}(V) \to T_{s+s'}^{r+r'}(V)
\end{equation}
is associative and distributive but not commutative (why?).

\begin{exercise}
  Give a tensor in $T^2_0$ which is a linear combination of tensor products but cannot be written as a tensor product.
  Justify your answer.\\
 \textit{\small Hint: one of the examples at the beginning of the chapter can help.}
\end{exercise}

In fact, this is a general fact.

\marginnote{A more general approach to this proposition is by proving the universal property of tensor spaces. See for instance~\cite[Propositions 12.5, 12.7 and 12.8]{book:lee}.}
\begin{proposition}\label{prop:tensorbasis}
  Let $V$ be an $n$-dimensional vector space.
  Let $\{e_j\}$ and $\{\varepsilon^i\}$ respectively denote the bases of $V=T_0^1(V)$ and $V^*=T_1^0(V)$ respectively.
  Then, every $\tau\in V_s^r$ can be uniquely written as the linear combination\marginnote{\textit{Exercise}: expand Einstein's notation to write the full sum on the left with the relevant indices.}% In the expression, all indices $j_1,\ldots,j_r$, $i_1,\ldots,i_s$ run from $1$ to $n$.}
  \begin{equation}\label{eq:tensor:decomposition}
    \tau = \tau^{j_1\cdots j_r}_{i_1\cdots i_s} \, e_{j_1}\otimes\cdots\otimes e_{j_r}\otimes \varepsilon^{i_1}\otimes \cdots\otimes \varepsilon^{i_s},
  \end{equation}
  where the coefficients $\tau^{j_1\cdots j_r}_{i_1\cdots i_n}\in\R$.
  %
  Thus the $n^{r+s}$ tensor products
  \begin{equation}\label{eq:tensor:decompo}
    e_{j_1}\otimes\cdots\otimes e_{j_r}\otimes \varepsilon^{i_1}\otimes \cdots\otimes \varepsilon^{i_s}, \quad j_1,\ldots,j_r, i_1,\ldots,i_s = 1,\ldots,n,
  \end{equation}
  form a basis of $T_s^r(V)$, and $T_s^r(V)$ has dimension $n^{r+s}$.
\end{proposition}

\begin{proof}
  Let $\{\beta^j\}$ and $\{b_i\}$ denote the bases of $V^*$ and $V$ that are dual to $\{e_j\}$ and $\{\varepsilon^i\}$, that is,
  \marginnote{A linear map is uniquely specified by its action on a basis, which in particular means that these dual bases are unique.}
  \begin{equation}
    (\beta^j\mid e_i) = \delta^j_i = (\varepsilon^j \mid b_i).
  \end{equation}
  Define
  \begin{equation}
    \tau^{j_1\cdots j_r}_{i_1\cdots i_s} := \tau(\beta^{j_1}, \ldots, \beta^{j_r}, b_{i_1}, \ldots, b_{i_s}).
  \end{equation}
  Then, on any element of the form $(\beta^{j_1}, \ldots, \beta^{j_r}, b_{i_1}, \ldots, b_{i_s})$, we trivially have the decomposition~\eqref{eq:tensor:decomposition}. By multilinearity of all the terms involved,~\eqref{eq:tensor:decomposition} holds for any element $(\omega^1, \ldots, \omega^r, v_1, \ldots, v_s)$ after decomposing it on the basis.

  Uniqueness follows from the linear independence of the tensor products $e_{j_1}\otimes\cdots\otimes e_{j_r}\otimes \varepsilon^{i_1}\otimes \cdots\otimes \varepsilon^{i_s}$ proceeding by contradiction.
\end{proof}

\begin{exercise}
Formalize in details the last step of the proof: uniqueness follows from the linear independence of the tensor products.
\end{exercise}

\begin{remark}
  There is a canonical isomorphism such that
  \begin{equation}
    T^r_s(V) \simeq \LaTeXoverbrace{V\otimes \cdots \otimes V}^{r\mbox{ times}} \otimes \LaTeXunderbrace{V^*\otimes \cdots \otimes V^*}_{s\mbox{ times}}.
  \end{equation}
  This allows us to choose whichever interpretation is more convenient for the problem at hand: being it a multilinear map on a cross product of spaces or an element of the tensor product of spaces.
\end{remark}


Let's go back for a moment to the example of inner products.

\begin{definition}\label{def:metric}
  We call \emph{pseudo-metric tensor}, any tensor $g\in T_2^0(V)$ that is
  \begin{enumerate}
    \item symmetric, i.e. $g(v,w) = g(v,w)$ for all $v,w\in T_0^1(V)$;
    \item positive definite, i.e. $g(v,v)\geq0$ for all $v\neq 0$.
  \end{enumerate}
  
  We call \emph{non-degenerate} any tensor $g\in T_2^0(V)$ such that
  \begin{equation}
    g(v,w) = 0 \quad\forall w\in V \qquad\Longrightarrow\qquad v=0.
  \end{equation}
  
  A \emph{metric tensor} or \emph{scalar product} is a non-degenerate pseudo-metric tensor.
  The Riemannian metric is a metric tensor on the tangent bundle of a manifold.
  An example of non-degenerate tensor which is not a metric is the so-called \emph{symplectic form}: a skew-symmetric non-degenerate $(0,2)$-tensor, which is fundamental in classical mechanics and the study of Hamiltonian systems.
\end{definition}

\begin{example}\label{ex:musicaliso}
  Let $V$ be a $n$-dimensional real vector space with an inner product $g(\cdot, \cdot)$.
  %
  Denote $\{e_1, \ldots, e_n\}$ the basis for $V$ and $\{e^1, \ldots, e^n\}$ the basis for its dual $V^*$.
  As a bilinear map on $V\times V$, the inner product is uniquely associated to a matrix $[g_{ij}]$ by $g_{ij} = (g e_i, e_j)$.

  We already mentioned that in this case we can canonically identify $V$ with $V^*$.
  Indeed, the inner product defines the isomorphisms\footnote{Often called \emph{musical isomorphisms} or index raising and index lowering operators.}
  \begin{equation}
    {}^\flat: V \to V^*,\; v\mapsto g(v, \cdot),
    \quad\mbox{and its inverse}\quad
    {}^\sharp: V^*\to V.
  \end{equation}
  \marginnote{That ${}^\flat$ is an isomorphism follows immediately from the linearity and the fact that non-degeneracy implies that its kernel contains only the zero vector.}
  The matrix of ${}^\flat$, by definition, is $[g_{ij}]$, that is,
  \begin{equation}
    (v^\flat)_i = g_{ij} v^j,
  \end{equation}
  where the $v^j$ are the components of $v$.
  Therefore, the matrix of ${}^\sharp$ is the inverse\footnote{Using lower indices for matrix entries and upper indices for the entries of the inverse is very common. It turns out to be an especially convenient notation, which simplifies many formulas in general relativity and classical mechanics.} $[g^{ij}]$ of the inner product matrix, that is, 
  \begin{equation}
    (\omega^\sharp)^i = g^{ij}\omega_j,
  \end{equation}
  where the $\omega_j$ are the components of $\omega$.

  \marginnote[2em]{To add to the confusion: in the physics literature, for $v\in V$, the components $v_j$ of $v^\flat$ are often called covariant components of $v$ while the components $v^j$ of $v$ are called its contravariant components.}
  Note that, in general, $e^\flat_i\neq e^i$: indeed, by definition $e^\flat_i = g_{ij}e^j$.

  It turns out that these operators can be applied to tensors to produce new tensors.
  For example, if $\tau$ is a $(0,2)$-tensor we can define an associated tensor $\tau'$ of type $(1,1)$ by $\tau(\omega, v) = \tau(\omega^\sharp, v)$.
  Its components are $(\tau')_i^j = g^{jk}\tau_{ik}$.
\end{example}

\begin{exercise}
  Let $V$ be a vector space with an inner product.
  \begin{enumerate}
    \item Show that the space $T^1_1(V)$ is canonically isomorphic to the space of endomorphisms of $V$, that is, of linear maps $L:V\to V$.

    \item If $\ell\in T^1_1(V)$ is the tensor associated to $A$, show that its components $\ell_i^j$ are just the matrix entries of $A$ seen as a matrix.
  
    \item Of course, given the previous example, $T^1_1(V)$ is also canonically isomorphic to the space of endomorphisms of $V^*$, that is, of linear maps $\Lambda:V^*\to V^*$.
    Prove the claim by explicitly constructing the mapping $\ell \leftrightarrow \Lambda$.
  \end{enumerate}
  \textit{\small Hint: definitions can look rather tautological when dealing with tensors... think carefully about domains and codomains, remember the musical isomorphisms and the tensor pairing.}
\end{exercise}

We are now in a good place to discuss how tensors are affected by changes of basis.
Let $L: V\to V$ be an isomorphism, then we can define a new basis $\{\widetilde e_i\}$ of $V$ by $\widetilde e_i := L e_i$. For convenience, we denote its dual basis by $\{\widetilde e^i\}$ in contrast with our initial notation.

Thinking in linear algebraic terms, the linear map $\Lambda:V^*\to V^*$ that relates the dual bases is determined by
\begin{align}
  &\delta^i_j = (\widetilde e^i \mid \widetilde e_j) = (\Lambda e^i \mid L e_j ) =: (L^* \Lambda e^i \mid e_j)\\
  &\mbox{that is, } \Lambda = (L^*)^{-1}.
\end{align}
Indeed, if $[l_i^j]$ is the matrix associated to $L$, that is, $\widetilde e_i = L e_i = l_i^j e_j$, then $\widetilde e^j = L^* e^j = l_i^j e^i$. Since the matrix $[\lambda_i^j]$ of $\Lambda$, that is, $\Lambda e^j = \lambda_k^j e^k$, must satisfy $\lambda_k^j l_i^k= \delta_i^j$, \emph{as matrices}, $[\lambda_i^j]$ is the inverse of $[l_i^j]$. However, don't forget that $[l_i^j]$ is the matrix of the endomorphism $L:V\to V$ while $[\lambda_i^j]$ is the matrix of the endomorphism $\Lambda: V^*\to V^*$.

We can transport this fact to general tensors to obtain that the components of an arbitrary tensor $\tau\in T^r_s(V)$ transform as follows.
Since
\begin{align}
  \tau
  &=
  \tau^{j_1\cdots j_r}_{i_1\cdots i_s} \, e_{j_1}\otimes\cdots\otimes e_{j_r}\otimes \varepsilon^{i_1}\otimes \cdots\otimes \varepsilon^{i_s} \\
  &= \widetilde\tau^{k_1\cdots k_r}_{h_1\cdots h_s} \, \widetilde e_{k_1}\otimes\cdots\otimes \widetilde e_{k_r}\otimes \widetilde\varepsilon^{h_1}\otimes \cdots\otimes \widetilde\varepsilon^{h_s},
\end{align}
applying the previous reasoning and comparing term by term we get
\begin{align}
  &\tau^{j_1\cdots j_r}_{i_1\cdots i_s} = \widetilde\tau^{k_1\cdots k_r}_{h_1\cdots h_s} l_{k_1}^{j_1}\cdots l_{k_r}^{j_r} \lambda_{i_1}^{h_1}\cdots \lambda_{i_s}^{h_s}\\
  &\mbox{or}\\
  &\widetilde\tau^{k_1\cdots k_r}_{h_1\cdots h_s} = \tau^{j_1\cdots j_r}_{i_1\cdots i_s} \lambda_{j_1}^{k_1}\cdots \lambda_{j_r}^{k_r}\cdots l_{h_1}^{i_1}\cdots l_{h_s}^{i_s}.
\end{align}

\begin{remark}
  An important consequence of this fact is that we can use a metric tensor, and the associated musical isomorphisms ${}^\flat$ and ${}^\sharp$, to canonically identify a tensor space $T_s^r(V)$ with $T_r^s(V)$, $T_0^{r+s}(V)$ and $T_{t+s}^0(V)$ by concatenating the correct number of maps, for example
  \begin{align}
    \cI = \cI_g : T_s^r (V) \to T_r^s(V) \\
    \cI : \tau \mapsto \tau \circ (\LaTeXunderbrace{{\cdot}^\flat, \ldots, {\cdot}^\flat}_{r\mbox{ times}}, \LaTeXoverbrace{{\cdot}^\sharp, \ldots, {\cdot}^\sharp}^{s\mbox{ times}}).
  \end{align}
  In general, one can use the metric to raise or lower arbitrary indices, changing the tensor type from $(r,s)$ to $(r+1, s-1)$ or $(r-1, s+1)$.

  A neat application of this is showing that a non-degenerate bilinear map $g\in T_2^0(V)$ can be lifted to a non-degenerate bilinear map on arbitrary tensors, that is
  \begin{equation}
    G: T_s^r(V)\times T_s^r(V) \to \R,
    \quad 
    G(\tau, \widetilde\tau) := (\cI_g(\tau)\mid \widetilde\tau).
  \end{equation}
  In particular, if $g$ is a metric on $V$, then $G$ is a metric on $T_s^r(V)$.
\end{remark}

\begin{exercise}
  What do the canonical identifications of $T_s^r(V)$ with $T_0^{r+s}$ and $T_{t+s}^0$ look like?
\end{exercise}

\begin{remark}
  Interestingly, even though each of the tensor spaces $T_s^r(V)$ is generally not an algebra, the map $\otimes$ transforms the collection of all tensor spaces
  \marginnote{This is a so-called \emph{graded algebra} since $\otimes :  T_s^r(V)\times T_{s'}^{r'}(V) \to T_{s+s'}^{r+r'}(V)$ in some sense moves along the structure of the indices.}
  \begin{equation}
    T(V) := \bigoplus_{r,s\geq 0} T_s^r(V), \qquad T_0^0(V):= \R,
  \end{equation}
  to an algebra, called \emph{tensor algebra}.
  Here, for $r=s=0$ we define the tensor multiplication with a scalar as the standard multiplication: $r\otimes v = r v$ for $r\in T_0^0(V)=\R$ and $v\in T^1_0(V)=V$.
\end{remark}

Before moving on, there is an important operation on tensors that will come back later on and is worth to introducte in its generality.

\begin{definition}
Let $V$ be a vector space and fix $r,s\geq0$.
For $h\leq r$ and $k\leq s$, we define the \emph{$(h,k)$-contraction} of a tensor as the linear mapping $T_s^r(V)\to T_{s-1}^{r-1}(V)$ defined through
\begin{align}
  v_1&\otimes\cdots\otimes v_r\otimes\omega^1\otimes\cdots\otimes\omega^s \\
  &\mapsto \omega^k(v_h)\, v_1\otimes\cdots\otimes v_{h-1}\otimes v_{h+1}\cdots\otimes v_r\otimes\omega^1\otimes\cdots\otimes\omega^{k-1}\otimes\omega^{k+1}\cdots\otimes\omega^s
\end{align}
and then extended by linearity, thus mapping $\tau \mapsto \widetilde\tau$ where
\begin{align}
  \widetilde\tau&(\nu^1,\ldots,\nu^{r-1}, v_1,\ldots,v_{s-1}) \\
  &= \tau(\nu^1,\ldots,\LaTeXunderbrace{e^i}_{h\mbox{th index}},\ldots,\nu^{r-1},w_1,\ldots,\LaTeXunderbrace{e_i}_{k\mbox{th index}},\ldots,w_{s-1}).
\end{align}
\end{definition}

\begin{example}
  To understand why the two equations in the definition are equivalent it is worth looking at an example over a decomposable element.
  For simplicity, assume $(r,s) = (2,3)$ and $\tau = v_1\otimes v_2\otimes\omega^1\otimes\omega^2\otimes\omega^3$.
  Then $\tau$ corresponds to a multilinear function
  \begin{equation}
    \tau(\nu^1,\nu^2,w_1,w_2,w_3) = \nu^1(v_1)\nu^2(v_2)\omega^1(w_1)\omega^2(w_2)\omega^3(w_3).
  \end{equation}
  By definition, the $(1,2)$-contraction is
  \begin{align}
    \widetilde\tau(\nu^1,w_1,w_2) &= \tau(e^i,\nu^1,w_1,e_i,w_2) \\
    &= e^i(v_1)\;\nu^1(v_2)\omega^1(w_1)\omega^2(e_i)\omega^3(w_2) \\
    &= \LaTeXunderbrace{e^i(v_1)\omega^2(e_i)}_{=\omega^2_i e^i (v_1) =\omega^2(v_1)}\nu^1(v_2)\omega^1(w_1)\omega^3(w_2) \\
    &= \omega^2(v_1)\; v_2\otimes\omega^1\otimes\omega^3 (\nu^1,w_1,w_2).
  \end{align}
\end{example}

\begin{example}
  For $a\in T_1^1(V)$, the contraction $\mathrm{tr} (a) := a^1_1$ is called the trace of $a$ and is the usual trace of the corresponding endomorphism $A:V\to V$.
\end{example}

\section{Tensor bundles}

It is time to leave the abstract world of vector spaces and start getting closer to our main focus: manifolds.
In the previous chapters we have shown that the tangent bundle and the cotangent bundle are families of vector spaces built over $M$ that are dual to each other.
We now have the tools to further extend this idea and define tensor bundles as families of tensor spaces build on top of the fibres of the tangent bundle.

\begin{definition}
  The \emph{$(r,s)$-tensor bundle over $M$} as the bundle
  \begin{equation}
    T_s^r M = \bigsqcup_{p\in M}\left(\{p\}\times T_s^r(T_p M)\right)
  \end{equation}
  of tensors of type $(r,s)$, with the projection on the first component $\pi:T_s^r M\to M$.
\end{definition}

Here pullback and differential\footnote{Now you see why somebody calls it pushforward...} turn out to be life-saviours:
any atlas $\{(U_i, \varphi_i)\}$ of $M$ can be naturally mapped to an atlas  on $T_s^r M$ via $\{(T_s^r U_i, \widetilde\varphi_i)\}$ where
\begin{align}
  \widetilde\varphi_i : T_s^r(U_i) \to T_s^r\varphi(U_i)
\end{align}
is defined by linearity on the fibres via
\marginnote{Study hint: look carefully at the domains and codomains of all the maps involved and make sure that you understand how this is defined.}
\begin{align}
  \widetilde\varphi&(p, e_{j_1}\otimes\cdots\otimes e_{j_r}\otimes \varepsilon^{k_1}\otimes \cdots\otimes \varepsilon^{k_s})\\
  &:=(\varphi(p), d\varphi_p e_{j_1}\otimes\cdots\otimes d\varphi_p e_{j_r}\otimes (\varphi^{-1})^*\varepsilon^{i_1}\otimes \cdots\otimes (\varphi^{-1})^*\varepsilon^{i_s}).
\end{align}

In analogy to the definition of vector fields, we can introduce tensor fields: these will just be local assignments of tensors to points.

\begin{definition}
  A section $\gamma(T_s^r M)$ of $T_s^r M$, that is, a smooth map $\tau : M \to T_s^r M$ such that $\pi\circ \tau = \id_M$, is called a \emph{tensor field} of type $(r,s)$.
  We denote the space of tensor fields of type $(r,s)$ by $\cT_s^r(M)$ and define $\cT_0^0(M) := C^\infty(M)$.
\end{definition}

\begin{example}
  With the definition above we have that $\fX(M) = \cT_0^1(M)$ and $\fX^*(M) = \cT_1^0(M)$.
\end{example}

Locally, we can express any tensor field in terms of the coordinate bases.
On a chart for $M$ with local coordinates $(x^i)$, our analysis of the change of basis tells us that $\tau\in\cT_s^r(M)$ has the form
\begin{equation}
  \tau(p) = \tau^{j_1\cdots j_r}_{i_1\cdots i_s}(p)\; \frac{\partial}{\partial x^{j_1}}\otimes\cdots\otimes\frac{\partial}{\partial x^{j_r}}\otimes dx^{i_1}\otimes\cdots\otimes dx^{i_s},
\end{equation}
where $\tau^{j_1\cdots j_r}_{i_1\cdots i_s}\in C^\infty(M)$.

\begin{example}
  A non-degenerate symmetric bilinear form $g\in \cT_2^0(M)$ is a \emph{pseudo-Riemannian metric} and the pair $(M,g)$ a \emph{pseudo-Riemannian manifold}\footnote{Also called \emph{semi-Riemannian manifold}.}.
  If $g$ is also fibre-wise positive definite, then $g$ is a \emph{Riemannian metric} and $(M,g)$ is a \emph{Riemannian manifold}.
  From this you see that the Riemannian metric is just an inner product on the tangent bundle of the manifold.

  \begin{enumerate}
    \item The euclidean space $\R^n$ is a Riemannian manifold with the usual scalar product, which we can represent as $g = \sum_{i=1}^n dx^i\otimes dx^i$ (What is its matrix form?).
    \item If $M=\R^4$, an example of pseudo-Riemannian metric is the Minkowski metric $g = g_{ij} dx^i\otimes dx^j$ where $[g_{ij}] = {\left(\begin{smallmatrix} -1 & 0 & 0 & 0\\ 0 & 1 & 0 & 0 \\ 0 & 0 & 1 & 0 \\ 0 & 0 & 0 & 1 \end{smallmatrix}\right)}$. The pseudo-Riemannian manifold $(M, g)$ is the space-time manifold of special relativity, with $x^1 = t$ is the time and $(x^2, x^3, x^4) = (x,y,z)$ is the space.
  \end{enumerate}
\end{example}

\begin{definition}
  The \emph{support} of a tensor field $\tau\in\cT_s^r(M)$ is defined as the set
  \begin{equation}
    \supp \tau := \overline{\{p\in M\mid \tau(p) \neq 0\}} \subset M.
  \end{equation}
  We sat that $\tau\in\cT_s^r(M)$ is \emph{compactly supported} if $\supp\tau$ is a compact set.
\end{definition}

Again in analogy with what we saw on tangent and cotangent bundles, we can provide a general definition of pullback and pushforward on tensor bundles.
This will be extremely useful soon, when we start dealing with differential forms.

\begin{definition}\label{def:pullback0s}
  Let $F:M\to N$ be a smooth map between smooth manifolds and let $\omega \in \cT_s^0(N)$ be a $(0,s)$-tensor field on $N$. We define the \emph{pullback of $\omega$ by $F$} as the $(0,s)$-tensor field $F^*\omega \in \cT_s^0(M)$ on $M$ defined for any $p\in M$ by
  \begin{align}
    &F^* : \cT_s^0(N) \to \cT_s^0(M),\\
    &F^*\omega|_p := dF^*_p(\omega|_{F(p)}) \quad \forall p\in M,
  \end{align}
  where
  \begin{equation}
    dF^*_p(\omega|_{F(p)})(v_1, \ldots, v_s) := \omega|_{F(p)} (dF_p v_1, \ldots, dF_p v_s), \quad\forall v_1, \ldots, v_s \in T_p M.
  \end{equation}
\end{definition}

To be consistent with this definition, if $f\in C^\infty(M) = \cT_0^0(M)$ and $\omega \in\cT_s^0(N)$, then we define $f\otimes \omega := f\omega$ and $F^* f := f\circ F$.

\begin{exercise}
  Show that the tensor pullback satisfies the following properties.
  Let $F:M\to N$ and $G:N\to P$ be smooth maps and $\nu, \omega \in\cT_s^0(N)$ and $f\in C^\infty(N)$, then the following hold
  \begin{enumerate}
    \item $F^*(f\otimes\omega) = F^*(f \omega) = (f\circ F) F^*\omega = (F^* f)(F^*\omega)$;
    \item $F^*(\omega\otimes\nu) = F^*\omega\otimes F^*\nu$;
    \item $F^*(\omega + \nu) = F^*\omega + F^*\nu$;
    \item $(G\circ F)^*\omega = F^*(G^* \omega)$;
    \item $(\id_N)^* \omega = \omega$.
  \end{enumerate}
\end{exercise}

We can use the pullback to construct a diffeomorphism of tensor bundles of the same type out of a diffeomorphism $\varphi:M\to N$ between manifolds.

\begin{proposition}
  Let $\varphi:M\to N$ be a diffeomorphism between smooth manifolds.
  Then $\varphi$ induces a diffeomorphism $T_s^rM \to T_s^r N$.
\end{proposition}
\begin{proof}
\newthought{Step I}.
We know that the pullback induces on the fibres a diffeomorphism of cotangent bundles. Let $p\in M$.
We have already seen that on the fibres the pullback is a diffeomorphism:
\begin{equation}
  T^*N \to T^* M, \quad
  (q,\omega) \mapsto (\varphi^*\omega)_q = \left(\varphi^{-1}(q), d\varphi^*\omega|_{\varphi^{-1}(q)}\right).
\end{equation}
This can be inverted giving rise to the so-called \emph{cotangent lift}
\begin{equation}
  d\varphi^\dagger :=(d\varphi)^\dagger := (\varphi^{-1})^*: T^*M \to T^*N.
\end{equation}
\marginnote{An aid to understand this map is the following commuting diagram:
\begin{equation}\nonumber
  \begin{tikzcd}[row sep=normal, column sep=normal, ampersand replacement=\&]
    T_p^* M \arrow[d, "\pi_M" left] \arrow[r, "d\varphi^\dagger"] \& T_{\varphi(p)}N \arrow[d, "\pi_N"] \\
    M \arrow[r, "\varphi" below] \& N
  \end{tikzcd}
\end{equation}}
For any $\omega\in T_p^* M$ and any $v\in T_pM$, we have
\begin{align}
  (d\varphi^\dagger_p \omega \;\mid\; d\varphi_p v)_{\varphi(p)} &=  d(\varphi^{-1})^*(\omega|_{\varphi^{-1}\circ\varphi(p)} (d\varphi_p v) )\\
  &= \omega_p(d\varphi^{-1}_{\varphi(p)} \circ d\varphi_p v ) \\
  &= \omega_p(v) = (\omega \mid v)_p.
\end{align}

\newthought{Step II}.
Chaining $d$ and $d^\dagger$ on the appropriate components of the tensor, we obtain a diffeomorphism of arbitrary tensor bundles:
\begin{equation}
  d\varphi \otimes\cdots\otimes d\varphi \otimes d\varphi^\dagger \otimes\cdots\otimes d\varphi^\dagger  : T_s^rM \to T_s^r N,
\end{equation}
defined on the product elements as
\begin{align}
  d\varphi &\otimes\cdots\otimes d\varphi \otimes d\varphi^\dagger \otimes\cdots\otimes d\varphi^\dagger  (p, v_1 \otimes \cdots\otimes v_r \otimes \omega^1\otimes\cdots\otimes\omega^s) \\
  &:= (\varphi(p), d\varphi\; v_1 \otimes \cdots\otimes d\varphi\; v_r \otimes d\varphi^\dagger \omega^1\otimes\cdots\otimes d\varphi^\dagger \omega^s),
\end{align}
which extends to the whole fibres by linearity.
\end{proof}

With this diffeomorphism at hand, we can finally define the pushforward.

\begin{definition}
  Let $F:M\to N$ be a diffeomorphism between smooth manifolds.
  We define \emph{pushforward of $(r,s)$-tensor fields} by $F$ as the map $F_*: \cT_s^r(M) \to \cT_s^r(N)$ for which the following diagram commutes:
  \begin{equation}
    \begin{tikzcd}[row sep=huge, column sep=huge]
      M \arrow[r, "F"] \arrow[d, "\tau" left]
      &[10em] N \arrow[d, "F_* \tau" cyan] \\
      T_s^r(M) \arrow[r, "dF \otimes\cdots\otimes dF \otimes dF^\dagger \otimes\cdots\otimes dF^\dagger "]
      & T_s^r(N)
    \end{tikzcd}.
  \end{equation}
  That is, for $\tau\in\cT_s^r(M)$ we define
  \begin{equation}
    F_*\tau = \LaTeXunderbrace{dF \otimes\cdots\otimes dF}_{r\mbox{ times}} \otimes \LaTeXoverbrace{dF^\dagger \otimes\cdots\otimes dF^\dagger }^{s\mbox{ times}} \circ \tau \circ F^{-1}.
  \end{equation}
\end{definition}

\begin{example}
  Let $f\in\cT_0^0(M)$, then $F_* f = f\circ F^{-1}$.
  Similarly, for $X\in\cT_0^1(M)$ we have the pushforward $F_* X = dF\circ X \circ F^{-1}$, in line with the definition of pushforward of vector fields that we gave in the previous chapter.
  An interesting, not really surprising though (right?), property is the following: $F_* df = d(F_* f)$.
\end{example}

\begin{exercise}[\textit{[homework 3]}]
  Let $F:M\to N$ and $G:N\to P$ two diffeomorphisms of smooth manifolds.
  \begin{enumerate}
    \item Show that the chain rule $(G\circ F)_* = G_* \circ F_*$ holds.
    \item Show that our previous definition\footnote{That is, Definition~\ref{def:pullback0s} -- which includes the pullback from Definition~\ref{def:pullback1f}.} of pullback is a particular case of the following general definition of a \emph{pullback of $(r,s)$-tensor fields by $F$}:
    \begin{equation}
      F^* := (F^{-1})_* : \cT_s^r(N) \to \cT_s^r(M).
    \end{equation}
  \end{enumerate}
  \textit{\small Hint: always work on a product tensor and extend by linearity.}
\end{exercise}

Note that thanks to this duality between pullback and pushforward, the dual pairing is always invariant under diffeomorphisms:
\marginnote[-2em]{In general, \eqref{eq:pairdualitypull} is not true for scalar products: one has to require that the diffeomorphism leaves the metric invariant, i.e. $g_N(F_* v, F_* w)\circ F = g_M(v,w)$ where $g_M\in T_2^0(M)$ and $g_N\in T_2^0(N)$.
You encounter this if you study isometries for pseudo-Riemannian metrics or canonical transformations in classical mechanics.}
\begin{equation}\label{eq:pairdualitypull}
  (F_* \omega \mid F_* v) = (\omega \mid v).
\end{equation}
Can you show why?

\begin{example}[Change of coordinates for tensor fields]
  Let, as usual, $(U,\varphi)$ be a chart on $M$ with local coordinates $(x^i)$.
  If $\{e^i:\R^n\to\R\}$ are the standard euclidean coordinates\footnote{Have a look at Notation~\ref{ntn:coords} if you don't remember what I am talking about.
  Here we are using the notation $e^i \equiv r^i$ since now we know that $\{e^i\}$ is just the dual basis to $\{e_i\}$. } and $\{e_i\}$ are the standard basis vectors in $\R^n$, then the coordinate $1$-forms and the coordinate vector fields on $U\subset M$ are given by
  \begin{equation}
    dx^i = \varphi^* de^i
    \quad\mbox{and}\quad
    \frac{\partial}{\partial x^i} = (\varphi^{-1})_* e_i.
  \end{equation}
  This immediately exposes the transformation laws for the change of coordinates: let $(U, \psi)$ be another chart on $U$ with local coordinates $(y^i)$, then $dy^i = \psi^* de^i$ and $\frac{\partial}{\partial y^i} = (\psi^{-1})_* e_i$. If we denote $\phi = \psi\circ\varphi^{-1}$ the transition map in $\R^n$, we get
  \begin{align}
    \frac{\partial}{\partial x^i} &= (\varphi^{-1})_* e_i \\
    &= (\varphi^{-1})_* \LaTeXunderbrace{(\phi^{-1})_*\phi_*}_{\id} e_i \\
    &= (\varphi^{-1}\circ \phi^{-1})_* (\phi_* e_i) \\
    &= (\psi^{-1})_* ((D\phi)_i^j e_j) \\
    &= (D\phi)_i^j \frac{\partial}{\partial y^j},
  \end{align}
  which may be easier to think about in terms of the following diagram
  \begin{equation}\nonumber
    \begin{tikzcd}[row sep=large, column sep=tiny]
      & \frac{\partial}{\partial x^i} \in \cT_0^1(U) \ni (D\phi)_j^i \frac{\partial}{\partial y^j} \arrow[dl, "\varphi_*" description] \arrow[dr, "\psi*" description] & \\
      e_i \in \cT_0^1(V) \arrow[rr, "\phi_*" description] & & \cT_0^1(W) \ni \LaTeXunderbrace{\phi_* e_i}_{= (D\phi)_i^j e_j}
    \end{tikzcd}
  \end{equation}
  where $V = \varphi(U)$ and $W = \psi(U)$.

  From this, we immediately get $dy^j = (D\phi)_i^j dx^i$ and, therefore, $dx^i =  (D\phi^{-1})_j^i dy^j$.
\end{example}

\begin{exercise}
  Let $F:N\to M$ be a smooth map between smooth manifolds.
  Show that a function $f\in C^\infty(M)$ is constant on $F(N)\subset M$ if and only if $F^* df \equiv 0$.\\
  \textit{\small Hint: if you get stuck start by looking at a simple example, like $N=M$ and $F=\id_M$.}
\end{exercise}


\chapter{Differential Forms}
In the rest of the course we will focus on a particular class of tensors, which generalizes the differential one-forms that we studied on the cotangent bundle.
It should not be surprising then, that these will be called differential $k$-forms and that they will be alternating $(0,k)$-tensors, that is, skew-symmetric in all arguments.

Geometrically, they are not dissimilar from the forms you may have seen in multivariable calculus: a $k$-form takes $k$ vectors as arguments and computes the $k$-dimensional volume spanned by these $k$-vectors.
In this sense, they will be the key elements to define integration over $k$-dimensional manifolds, in the same way as one-forms and line integrals.

\begin{definition}
  Let $V$ be a real $n$-dimensional vector space.
  Let $S_k$ denote the \emph{symmetric group on $k$ elements}, that is, the group of permutations of the set $\{1,\ldots,k\}$.
  Recall that for any permutation $\sigma\in S_k$, the \emph{sign of $\sigma$}, denoted $\sgn(\sigma)$, is equal to $+1$ if $\sigma$ is even\footnote{It can be written as a composition of an even number of transpositions} and $-1$ is $\sigma$ is odd\footnote{It can be written as a composition of an odd number of transpositions}.

  \marginnote{In particular, exchanging two arguments changes the sign of $\omega$.}
  A tensor $\omega\in T_k^0(V)$, $0\leq k\leq n$, is called \emph{alternating} (or \emph{antisymmetric} or \emph{skew-symmetric}), if it changes sign whenever two of its arguments are interchanged, that is,
  for all $v_1, \ldots, v_k\in V$ and for any permutation $\sigma\in S_k$ it holds that
  \begin{equation}
    \omega(v_{\sigma(1)}, \ldots, v_{\sigma(k)}) = \sgn(\sigma) \omega(v_1, \ldots, v_k).
  \end{equation}
  The subspace of alternating tensors in $T_k^0(V)$ is denoted by $\Lambda_k \equiv \Lambda_k(V)$ and its elements are called \emph{exterior forms}, \emph{alternating $k$-forms} or just  \emph{$k$-forms}.
  For $k=0$, we define $\Lambda_0 := T_0^0(V) := \R$.
\end{definition}

\section{The exterior derivative}
\section{Poincar\'e lemma}

\chapter{Integration of forms}
We finally have all the main ingredients to generalize our line integral detour and discuss integration of $n$-forms over $n$-dimensional manifolds.

\section{Orientation}

\newthought{We know from calculus one}, or our line integral examples, that the direction in which we traverse the interval, or a curve, can actually make a difference.
Indeed, the sign of the integral of a differential $n$-form is only fixed after choosing an orientation of the manifold.

If for a curve an orientation is simply a choice of a direction along it, so we can make sense of it in terms of clockwise or counter-clockwise, generalising the concept will require an extra abstraction step.
Not just that, you have seen already that in $\R^n$ there is a standard orientation, but in other vector spaces we may need to make arbitrary choices.
For manifolds, the situation is much more complicated: for example, on a M\"obius strip\footnote{Cf. Example~\ref{ex:mobius}.} it is impossible to make any such choice, as it turns out, it is non-orientable.

Let's get there step by step.

\begin{definition}
  Let $V$ be a one-dimensional vector space. Then $V\setminus\{0\}$ has two components.
  An \emph{orientation} of $V$ is a choice of one of these components, which one then labels as ``positive'' and ``negative''.
  A \emph{positive basis} of $V$ then is a choice of any non-zero vector belonging to the positive component, while a \emph{negative basis} of $V$ is a choice of any non-zero vector belonging to the negative component.
\end{definition}

\begin{example}
  The standard orientation of $\R$ is give by declaring that the positive numbers are the positive components of $\R\setminus\{0\}$.
  A common choice as positive basis for $\R$ is $\{e_1 \equiv 1\}$ while a negative basis could be $\{-e_1\}$.
\end{example}

Let $V$ be a $n$-dimensional vector space.
How can we generalize in a meaningful way the definition above?

By Proposition~\eqref{prop:dimLkV}, the space $\Lambda^n(V)$ is a one-dimensional vector space.
Moreover, if $\{e_1,\ldots,e_n\}$ is a basis for $V$, then $e_1\wedge\cdots\wedge e_n$ is a basis for $\Lambda^n(V)$.

Looks like we are getting somewhere.

\begin{definition}
  Let $V$ be a $n$-dimensional vector space.
  An \emph{orientation} on $V$ is a choice of orientation on $\Lambda^n(V)$.
  \marginnote{It should be clear from this that the orientation is, in fact, an equivalence class of ordered bases.}
  Therefore there are exactly two orientations: we say that a basis $\{e_1,\ldots,e_n\}$ of $V$ is \emph{positive} (or positively oriented) if $e_1\wedge\cdots\wedge e_n$ is a positive basis of $\Lambda^n(V)$ and \emph{negative} (or negatively oriented) otherwise.
\end{definition}

\begin{example}
  If $e_i$ is the standard $i$th basis vector in $\R^n$, the standard orientation of $\R^n$ is given by declaring that $e_1\wedge\cdots e_n$ is a positive basis of $\Lambda^n(\R^n)$ and thus that $\{e_1,\ldots,e_n\}$ is a positive basis of $\R^n$.
\end{example}

The key in the preservation of orientation now resides only in the way different bases are transformed by $n$-forms, as the following lemma shows.

\begin{lemma}\label{lemma:orient}
  Let $V$ be a $n$-dimensional vector space and let $0\neq \omega\in\Lambda^n(V)$.
  Then, all bases $\{v_1, \ldots, v_n\}$ for which $\omega(v_1,\ldots v_n) > 0$ give the same orientation for $V$.
\end{lemma}
\begin{proof}
  Let $\{v_1, \ldots, v_n\}$ and $\{w_1, \ldots, w_n\}$ denote two different basis for $V$, then there exists a linear isomorphism $\varphi$ such that $v = \varphi w$, that is $v_i = \varphi_{i}^j w_j$.
  By definition and by multilinearity we then have
  \begin{equation}\label{eq:posorie}
    \omega(v_1,\ldots v_n) = \omega(\varphi w_1,\ldots \varphi w_n) = \det(\varphi)\omega(w_1,\ldots w_n) > 0,
  \end{equation} that is the positivity of $\omega$ on the bases characterize the set of bases.
\end{proof}

\begin{exercise}
  Let $V$ be a $n$-dimensional vector space, prove that two nonzero $n$-forms on $V$ determine the same orientation if and only if each is a positive multiple of the other.
\end{exercise}

\begin{remark}
  Of course, if $V$ is a vector space, then an orientation on $V$ canonically determines an orientation on the dual space $V^*$ by declaring that the basis dual to a positive basis is itself positive.
\end{remark}

We are almost there.
The tangent space is a vector space and $n$-forms act naturally on tangent vectors, this seems likely to be the right place to define an orientation for a manifold, at least pointwise.
As usual, one does need to make sure that all the local orientations just defined on the tangent bundle are gluing together coherently.

\begin{remark}
  If we look at a single chart $(U,\varphi)$, by Lemma~\ref{lemma:orient} each chart in the atlas determines an orientation at each point of its domain, which will be positive if $\det(D\varphi)>0$ and negative otherwise.
  This procedure can be repeated for each chart in an atlas for $M$.
  Thus, in order to get a globally consistent ordering, we need to worry about the overlaps between charts.
\end{remark}

\begin{definition}
  We call an atlas $\cA = \{(U_i,\varphi_i)\}$ \emph{oriented} if all the charts have the same orientation, that is, if $\det(D\varphi_{ij}) > 0$ for all the transition functions $\varphi_{ij} := \varphi_i\circ\varphi_j^{-1}$.

  A manifold $M$ with an oriented atlas is called \emph{oriented manifold}.
  If an orientation exists, we say tht $M$ is \emph{orientable}, in this case we call the equivalence class of atlases with the same orientation an \emph{orientation}.
  Otherwise we say that the manifold is \emph{nonorientable}.
\end{definition}

An immediate consequence of Lemma~\ref{lemma:orient} is that if a manifold is orientable, there are exactly two different orientations.

\begin{definition}
  Given an orientation on a manifold, we say that any chart from the same equivalence class of atlases is \emph{positively oriented}, while we call all other charts \emph{negatively oriented}.
\end{definition}

If $M$ is connected, as for vector spaces, an orientation on $\Lambda^n(M)$ determines the orientation of the manifold.
\marginnote{If it is not connected, then we need to deal with each connected component separately.}

\begin{theorem}
  Let $M$ be a $n$-dimensional smooth manifold.
  A nowhere-vanishing $n$-form $\omega\in\Omega^n(M)$ uniquely determines an orientation.
  For this reason, nowhere vanishing $n$-forms on a smooth $n$-manifold are called \emph{volume forms}.
\end{theorem}
\begin{proof}
  Let $\varphi$ and $\psi$ be two different charts with overlapping domains (otherwise there is nothing to check) and with local coordinates $(x^i)$ and $(y^i)$ respectively.
  Define the transition map $\varphi := \psi\circ\varphi^{-1}$, so that $(y^1,\ldots,y^n) = \varphi(x)$.
  Since $d y^j = (D\varphi)_i^j dx^i$, we have that locally
  \begin{align}
    \omega &= \widetilde \omega (y) dy^1\wedge\cdots\wedge dy^n \\
    &= (\widetilde\omega \circ \varphi)(x) \det(\varphi|_x) dx^{1}\wedge\cdots\wedge dx^{n} \\
    &= \omega(x) dx^{1}\wedge\cdots\wedge dx^{n},
  \end{align}
  where we used Proposition~\ref{prop:wedgeToJDet} and Theorem~\ref{thm:pullbacksdifferentialforms}.
  Thus, $\omega(x)$ and $\widetilde\omega(y)$ have the same sign if and only if $\det(D\varphi|_x) > 0$.
\end{proof}

\begin{definition}
  Let $M$ be a $n$-dimensional smooth manifold.
  If $(U,\varphi)$ is a chart with local coordinates $(x^i)$ such that, in the coordinate representation, $\omega = \omega(x) dx^1\wedge\cdots\wedge dx^n$ with $\omega(x) > 0$, then we say that the chart $\varphi$ is \emph{positively oriented} with respect to $\omega$, otherwise we say that it is \emph{negatively oriented}.  
\end{definition}

\begin{remark}
  In fact, this definition can be immediately extended to vector bundles.
  Given a real vector bundle $\pi: E \to M$, an orientation of $E$ means that for each fiber $E_p$, there is an orientation of the vector space $E_p$ such that each trivialization map
  \begin{equation}
    \varphi_{U}:\pi^{-1}(U)\to U\times \R^{n},
  \end{equation}
  with $\R^n$ equipped with its standard orientation, is fiberwise orientation-preserving.
  \marginnote[-2em]{Otherwise said, we can cover the manifold by (continuous) local frames whose local trivializations are orientation preserving.}

  With this definition, the orientability of $M$ coincides with the orientability of the bundle $M\to TM$.
\end{remark}

\begin{example}
  The euclidean space $\R^n$ is orientable with orientation given by the continuous global frame $\frac{\partial}{\partial r^i},\ldots,\frac{\partial}{\partial r^n}$.
\end{example}

\begin{example}\label{exe:orientsphere}
  Let $M = \bS^1\subset \R^2$.
  This is an orientable manifold and we can find an orientation using the stereographic projections from Exercise~\ref{ex:stereo}.
  Let $U_1 = \bS^1\setminus\{N\}$ and $U_2 = \bS^1\setminus\{S\}$, with the associated diffeomorphisms
  \begin{equation}
    \varphi_1(p) = \frac{2p^1}{1-p^2}
    \quad\mbox{and}\quad
    \varphi_2(p) = \frac{2p^1}{1+p^2}.
  \end{equation}
  Let's pick a pointwise orientation by choosing as basis $X_p\in T_pM$ given by $X_p = -p^2 \frac{\partial}{\partial p^1} + p^1 \frac{\partial}{\partial p^2}$.
  Then, on $U_1$,
  \begin{align}
    (\varphi_1)_*(X) &= (d\varphi_1)_p(X) \\
    &= \left(\begin{smallmatrix}
      \frac{2}{1-p^2} & \frac{2p^1}{(1-p^2)^2}
    \end{smallmatrix}\right)
    \left(\begin{smallmatrix}
      -p^2 \\ p^1
    \end{smallmatrix}\right) \frac{\partial}{\partial x}\Big|_{\varphi_1(p)}\\
    &= \frac{2}{1-p^2} \frac{\partial}{\partial x}\Big|_{\varphi_1(p)},
  \end{align}
  and $\frac{2}{1-p^2}>0$.
  If we perform the same computation on $U_2$, however, we obtain $(\varphi_2)_*(X) = -\frac{2}{1+p^2}\frac{\partial}{\partial x}\Big|_{\varphi_2(p)}$, with the negative coefficient $-\frac{2}{1+p^2} < 0$ (check!), corresponding to the opposite orientation on $U_2$.
  Of course, in this case, not all is lost: by choosing $\widetilde\varphi_2(p) = \varphi_2(-p^1, p^2)$ we obtain $(\widetilde\varphi_2)_*(X) = \frac{2}{1+p^2} \frac{\partial}{\partial x}\Big|_{\widetilde\varphi_2(p)}$ with the positive coefficient $\frac{2}{1+p^2} > 0$ (check!), which shows that $X_p$ defines an orientation on the whole $\bS^1$.
\end{example}

\begin{exercise}
  Check that the Jacobian determinant $\det(D(\varphi_2\circ \varphi_1^{-1}))$ of the transition chart from Exercise~\ref{exe:orientsphere} is negative, while $\det(D(\widetilde\varphi_2\circ \varphi_1^{-1}))$ is positive.
\end{exercise}

\begin{marginfigure}
  \includegraphics{7_1-mobius_strip.pdf}
\end{marginfigure}

\begin{exercise}
  Consider the open M\"obius strip $M$, a variation of Example~\ref{ex:mobius} defined as the quotient of $\R\times(-1,1)$ via the identification $(x,y) \sim (x+1, -y)$, and denote $\pi: \R\times(-1,1)\to M$ the corresponding projection map.
  The M\"obius strip inherits the differentiable structure from $\R^2$, so we need to show that there is no orientable atlas which is also compatible with the differentiable structure on $M$.
  \begin{enumerate}
    \item Define the map $\sigma:\R\times(-1,1)\to\R\times(-1,1)$ by $\sigma(x,y) = (x+1, -y)$ and show that $\pi\circ\sigma = \sigma$.
    \item If $\nu\in\Omega^2(M)$ define $f$ by $\pi^* \nu = f \omega$ where $\omega$ is an area\footnote{I.e. a volume $2$-form.} form on $\R\times(-1,1)$.
    Show that $f(x+1, y) = - f(x,y)$.
    \item Conclude that $f$ must vanish at some point of $\R\times(-1,1)$, which implies that $M$ is nonorientable.
  \end{enumerate}
\end{exercise}

What about orientation on the boundaries?

Let's first look at the tangent space. 

Let $M$ be a smooth $n$-manifold with boundary and $p\in \partial M$.
Then, we have three types of possible vectors:
\begin{enumerate}
  \item tangent boundary vectors: $X\in T_p(\partial M)\subset T_p M$ tangent to the boundary, forming an $(n-1)$-dimensional subspace of $T_p M$;
  \item inward pointing vectors: $X\in T_pM$ such that $X = \varphi^{-1}_*(Y)$ where $\varphi^{-1}: V\subset \cH^n \to M$ and $Y$ is some vector $Y = (Y^1, \ldots, Y_n)$ with $Y_n > 0$;
  \item outward pointing vectors: $X\in T_pM$ such that $-X$ is inward pointing.
\end{enumerate}
Thus, a vector field along $\partial M$ is a function $X:\partial M\to T_pM$ (not to $T_p\partial M$).

\begin{proposition}
  On a smooth manifold $M$ with boundary, there is a smooth outward pointing vector field along $\partial M$.  
\end{proposition}
\begin{proof}
  Pick an open cover of $\partial M$ with coordinate charts $\{(U_\alpha, (x^1_\alpha,\ldots,x^n_\alpha) \mid \alpha\in I\}$. Then $X_\alpha = -\frac{\partial}{\partial x^n_\alpha}$ on $U_\alpha\cap \partial M$ is smooth and outward pointing.
  Choose a partition of unity $\{\rho_\alpha \mid \alpha\in I\}$ on $\partial M$ subordinate to the open cover $\{U_\alpha\cap \partial M \mid \alpha\in I\}$.
  Then $X:= \sum_{\alpha\in I}\rho_\alpha U_\alpha$ is a smooth ouwtard pointing vector field along $\partial M$.
\end{proof}

We can use this to introduce a notion of induced orientation on $\partial M$.

\begin{proposition}
  Let $M$ be an oriented $n$-manifold with boundary.
  If $\omega$ is a volume form on $M$ and $X$ a smooth outward-pointing vector field on $\partial M$, then $\iota_X\omega$ is a smooth nowhere-vanishing $(n-1)$-form on $\partial M$ and, thus, $M$ is orientable.
\end{proposition}
\begin{proof}
  Since both $\omega$ and $X$ are smooth, the contraction $\iota_X\omega$ is also smooth.
  We need to check that it cannot vanish.

  Assume that $\iota_X\omega$ does vanish at some point $p\in\partial M$, that is, $(\iota_X)(v_1, \ldots, v_{n-1}) = 0$ for all $v_1, \ldots, v_{n-1}\in T_p(\partial M)$.
  Let $\{e_1,\ldots,e_{n-1}\}$ be a basis for $T_p(\partial M)$.
  Then $\{X_p,e_1,\ldots,e_{n-1}\}$ is a basis for $T_p M$ such that
  \begin{equation}
    \omega_p(X_p, e_1, \ldots, e_{n-1}) = (\iota_X\omega)_p(e_1, \ldots, e_{n-1}) = 0.
  \end{equation}
  Then, by Exercise~\ref{ex:zeroform}, $\omega_p\equiv0$ reaching a contradiction.
  
  Therefore, $\iota_X\omega$ is non-vanishing on $\partial M$ which means that $\partial M$ is orientable.
\end{proof}

\begin{exercise}
  Let $M$ be an oriented manifold with boundary, $\omega$ an orientation for $M$ and $X$ a smooth outward pointing vector field along $\partial M$.
  Prove the following statements.
  \begin{enumerate}
    \item It $\sigma$ is another orientation form on $M$, then $\sigma = f\omega$ for some everywhere positive $f\in C^\infty(M)$. Prove that  $\iota_X\sigma = f\iota_X \omega$ on $\partial M$.
    \item Show that if $Y$ is another smooth outward pointing vector field along $\partial M$, then there is an everywhere positive $f\in C^\infty(M)$ such that $\iota_Y\sigma = f \iota_X \omega$ on $\partial M$.
  \end{enumerate}
\end{exercise}

Note that if $(U_i, \varphi_i)$ is a positively oriented atlas on $M$, then $(U_i|_{\partial M}, \varphi_i|_{\partial M})$ can be negatively oriented. 
Let $\omega = dx^1\wedge\cdots\wedge dx^n$ be a positive volume form on $M$ on one of the charts, then $-\frac{\partial}{\partial x^n}$ is an outward pointing on $\partial \cH^n$ and we have
\begin{align}
  \iota_{-{\partial}/\!{\partial x^n}} (dx^1\wedge\cdots\wedge dx^n)
  &= -\iota_{{\partial}/\!{\partial x^n}} (dx^1\wedge\cdots\wedge dx^n) \\
  &= -(-1)^{n-1} dx^1\wedge\cdots\wedge dx^{n-1}\wedge \iota_{{\partial}/\!{\partial x^n}} (dx^n) \\
  &= (-1)^n dx^1\wedge\cdots\wedge dx^{n-1}.
\end{align}
Thus, for example, the boundary orientation on $\partial \cH^1 = \{0\}$ is $-1$, the one on $\partial \cH^2$ is the standard orientation on $\R$ given by $dx^1$ and the one on $\partial \cH^3$ is $-dx^1\wedge dx^2$, which is the clockwise orientation in the $(x^1, x^2)$-plane, etc.

\begin{example}\label{ex:int:bdryo}
  The closed interval $[a,b]\subset\R$ with standard euclidean coordinate $x$ has a standard orientation given by the vector field $\frac{\partial}{\partial x}$. 
  Therefore, the boundary orientation at $b$ is $\iota_{\frac{\partial}{\partial x}}(dx) = +1$ and the one at $a$ is $\iota_{-\frac{\partial}{\partial x}}(dx) = -1$.
\end{example}

\begin{exercise}
  \begin{enumerate}
    \item Prove that $\bS^n$ is orientable.
    \item Prove that any Lie group is orientable.
    \item Prove that $\RP^n$ is orientable if and only if $n$ is odd. \\
      \textit{\small Hint: the antipodal map $x\mapsto -x$ on $\bS^n$ can help.}
  \end{enumerate}
\end{exercise}

\section{Integrals on manifolds}

To avoid unnecessary complications, we will only integrate $n$-forms with compact support.
Armed with our experience with line integrals, fond memories of multivariable analysis and our recent discoveries, we are finally ready to talk about integrals.
Let's keep things simple and go one step at a time.

\begin{definition}\label{def:intnform:chart}
  Let $M$ be a smooth $n$-manifold and $(U,\varphi)$ be a chart from an oriented atlas of $M$.
  If $\omega\in\Omega^n(M)$ be a $n$-form with compact support in $U$, we define the integral of $\omega$ as
  \begin{equation}
    \int_M \omega = \int_U \omega := \int_{\varphi(U)} \varphi_*\omega := \int_{\cH^n} \omega(p) d^n p,
  \end{equation}
  where $d^n p$ denotes the $n$-dimensional Lebesgue measure on $\R^n$ and on the chart
  \begin{equation}
    \varphi_*\omega = \omega(p) e^1\wedge \cdots\wedge e^n\in\Omega^n(\cH^n).
  \end{equation}

  If $M$ is an oriented $0$-dimensional manifold and $f$ is a $0$-form (that is, a smooth function) than we defined the integral to be the sum
  \begin{equation}
    \int_M f = \sum_{p\in M} \pm f(p),
  \end{equation}
  where we take the positive sign at points where the orientation is positive and the negative sign at points where it is negative.
  The compactness assumption here implies that there are only finitely many nonzero terms in the sum.
\end{definition}

To make sure that this definition makes sense, let's show that the integral is well-defined, that is, up to orientation it does not depend on the chosen chart.

\begin{lemma}\label{lemma:intindep:chart}
  Suppose $\omega\in\Omega^n(M)$ with compact support $\supp \omega \subset U\cap V$, where $(U, \varphi)$ and $(V, \psi)$ are two positively oriented charts on the oriented manifold $M$.
  Then, the value of the integral $\int_M\omega$ is independent on the chosen chart.
\end{lemma}
\begin{proof}
  Let $\varphi$ and $\psi$ be two charts on $U$ with the same orientation and local coordinates $p$ and $q$, let $\varphi = \psi\circ\varphi^{-1}$ be the corresponding transition map.
  Then,
  \begin{align}
    \int_{\varphi(U)} \varphi_*\omega &= \int \omega(p)\, d^n p \\
    &\overset{(\star)}{=} \int (\widetilde\omega \circ \varphi)(p) \det(D\varphi|_p) d^n p \\
    &\overset{(*)}{=} \int \widetilde\omega(q) d^n q\\
    &= \int_{\psi(U)} \psi_*\omega,
  \end{align}
  where $\omega(p)$ and $\widetilde\omega(q)$ are the local expressions for $\omega$ in the two coordinate charts, in $(\star)$ we applied Proposition~\ref{prop:wedgeToJDet} and in $(*)$ we used the classical euclidean change of variables.
\end{proof}

To be able to integrate charts which are not supported in the domain of a single chart, we now need the help of a partition of unity.

\begin{definition}
  Let $M$ be a smooth oriented manifold and $\cA = \{(U_i,\varphi_i)\}$ a positively oriented atlas.
  If $\omega \in \Omega^n(M)$ has compact support, then the \emph{integral of $\omega$} is defined as
  \begin{equation}\label{eq:intnform}
    \int_M \omega := \sum_{j=1}^N \int_{U_j}\rho_j\omega,
  \end{equation}
  where $\{\rho_j\mid j=1,\ldots, N\}$ is a partition of unity subordinate to a finite cover of $\supp \omega$ by charts $\{U_j\}$ and such that $\sum_{j=1}^N \rho_j(p) = 1$ for $p\in\supp\omega$.
  \marginnote{The terms on the right hand side of \eqref{eq:intnform} are all integrals as in Definition~\ref{def:intnform:chart}.}
\end{definition}

The definition above makes sense only if the value of the integral is independent of the chosen partition, but with the help of the previous lemma this is easily checked.

\begin{lemma}\label{lemma:intinman}
  The value of $\int_M\omega$ is independent from the choice of the atlas and the choice of partition of unity.
\end{lemma}
\begin{proof}
  The independence from the choice of the charts was demonstrated in Lemma~\ref{lemma:intindep:chart}.
  Let $\{\widetilde\rho_j\}$ be another partition of unity adapted to a (possibly different) finite cover by charts $\{(V_j, psi_j)\}$ with $\sum \widetilde\rho_j(p) = 1$ for $p\in\supp\omega$.
  Then we have,
  \begin{align}
    \sum_j \int_{\varphi_j(U_j)} (\varphi_j)_*\left(\rho_j \omega\right)
    &= \sum_j \int_{\varphi_j(U_j)} (\varphi_j)_*\left(\rho_j \sum_k \widetilde\rho_k\omega\right) \\
    &= \sum_{j,k} \int_{\phi_j(U_j\cap V_k)} (\varphi_j)_* \left(\rho_j \widetilde\rho_k\omega\right) \\
    &\overset{(!)}{=} \sum_{j,k} \int_{\psi_k(U_j\cap V_k)} (\psi_k)_* \left(\rho_j \widetilde\rho_k\omega\right) \\
    &= \sum_k \int_{\psi_k(U_j\cap V_k)} (\psi_k)_*\left(\rho_j \widetilde\rho_k \sum_j\rho_j \omega\right) \\
    &= \sum_k \int_{\psi_k(V_k)} (\psi_k)_*\left( \widetilde\rho_k \omega\right),
  \end{align}
  where in $(!)$ we used Lemma~\ref{lemma:intindep:chart}.
\end{proof}

This result can be nicely formalized as follows.
\begin{theorem}[Global change of variable]
  Suppose $M$ and $N$ are oriented $n$-manifolds and $F:M\to N$ is an orientation preserving diffeomorphism.
  If $\omega\in\Omega^n(N)$ has compact support, then $F^*\omega$ has compact support and the following holds
  \begin{equation}
    \int_N \omega = \int_M F^* \omega.
  \end{equation}
\end{theorem}
\begin{proof}
  First of all, observe that $\supp(F^*\omega) = F^{-1}(\supp(\omega))$ which is compact since manifolds are Hausdorff spaces and $F$ is continuous.

  Let now $\{(U_i,\varphi_i)\}$ be an atlas of a positively oriented chart on $M$ and $\{\rho_i\}$ a subordinate partition of unity.
  Then, $\{(F(U_i),\varphi_i\circ F^{-1})\}$ is an atlas of positively oriented charts for $N$ and $\{\rho_i \circ F^{-1}\}$ is a partition of unity subordinate to the covering $\{(F(U_i)\}$.
  By Lemma~\ref{lemma:intinman} we have,
  \begin{align}
    \int_M F^*\omega &= \sum_i \int_M \rho_i F^*\omega \\
    &= \sum_i \int_{\R^n}(\varphi_i)_*(\rho_i F^*\omega)\\
    &= \sum_i \int_{\R^n}(\varphi_i)_*(F^{-1})_*(\rho_i \circ F^{-1}) \omega \\
    &= \sum_i \int_{\R^n}(\varphi_i \circ F^{-1})_*(\rho_i \circ F^{-1}) \omega \\
    &= \int_N\omega,
  \end{align}
  which shows the commutativity of the following diagram
  \begin{equation}
    \begin{tikzcd}
      \Omega^n(M) \arrow[rr, "F^*", yshift=0.5ex] \arrow[dr, "\int_M"] & & \Omega^n(N)\arrow[ll, "F_*", yshift=0.5ex] \arrow[dl, "\int_N"] \\
      & \R &
    \end{tikzcd}
  \end{equation}
  and concludes the proof.
\end{proof}

This justifies the following definition.

\begin{definition}[Integral on submanifolds]
  Let $M$ a smooth $m$-manifold, $N$ an oriented smooth $n$-manifold and $J:N\to M$ a smooth map\footnote{If $N\subset M$ is a submanifold, then $J:N\hookrightarrow M$ is just the inclusion map.}.
  If $\omega\in\Omega^m(M)$ has compact support, we define
  \begin{equation}
    \int_N \omega := \int_N J^*\omega.
  \end{equation}
  In particular, if $M$ is compact, oriented, smooth $m$-manifold, $\omega$ is a $(m-1)$-form on $M$ and $i:\partial M\hookrightarrow M$ is the inclusion of the boundary in $M$, we can interpret unambiguously
  \begin{equation}
    \int_{\partial M} \omega := \int_{\partial M} i^* \omega,
  \end{equation}
  where $partial M$ is understood to have the induced orientation.
\end{definition}

\begin{example}
  Let $M=[a,b]\subset \R$ equipped with the canonical global atlas $\{(M, \id_\R|_M)\}$ and $f\in C^\infty_0(M)$, i.e., smooth with compact support\footnote{Which does not mean $f(a) = f(b)=0$ since $[a,b]$ is itself compact.}.
  Then, $df\in\Omega^1(M)$ and $\supp df \subset \supp f$ is compact as well and we have
  \begin{equation}
    \int_M df = \int_a^b \frac{\partial f}{\partial x} dx = f(b)- f(a) = \int_{\partial M} f.
  \end{equation}
\end{example}

\begin{exercise}[Fubini's theorem]
  
\end{exercise}

\section{Stokes' Theorem}

Theorem, corollary, a few remarks and comparisons, proof, exercises and examples.

Invariance under diffeotopies and diffeomorphisms.

More exercises and examples.

\begin{appendices}

\chapter{Frobenius theorem}
Integrable and nonintegrable distributions, Contact geometry, Frobenius theorem.

\chapter{Vector bundles and connections}
Hopefully we can cut on differential forms since they were treated in multivariable analysis and get to this.

\end{appendices}

% \begin{appendices}
%   \chapter{Solution to selected exercises}
%   \section{Chapter~\ref{ch:manifolds}}
  
%   \newthought{Exercise~\ref{exe:rntopsp}.}
%   \begin{enumerate}
%     \item[] Hausdorff. For $x\neq y\in\R^n$, let $\epsilon = d(x,y)/3$.
%     Then the two balls $B_x(\epsilon) := \{z\in X \;\mid\; d(z,x)<\epsilon\}$ and $B_y(\epsilon)$ are disjoint open sets containing $x$ and $y$ respectively.
%     \item[] Second countable. As countable basis for the topology we can take the open balls $B_\epsilon(x)$ with rational radii $\epsilon\in\Q$ and centers $x\in\Q^n$.
%   \end{enumerate}
  
% \end{appendices}

\printbibliography
\addcontentsline{toc}{chapter}{Bibliography}
\end{document}